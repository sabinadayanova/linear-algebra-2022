\documentclass[10pt, a4paper]{extarticle}

%% Язык
\usepackage{cmap} % Поиск в PDF
\usepackage{mathtext} % Кириллица в формулах
\usepackage[T2A]{fontenc} % Кодировка
\usepackage[utf8]{inputenc} % Кодировка
\usepackage[english,russian]{babel} % Локализация, переносы

\pagestyle{empty} \textwidth=19.0cm \oddsidemargin=-1.3cm
\textheight=26cm \topmargin=-3.0cm

%% Математика
\usepackage{amsmath, amsfonts, amssymb, amsthm, mathtools}
\usepackage{icomma}

% Операторы
\DeclareMathOperator{\tr}{tr}
\renewcommand{\le}{\leqslant}
\renewcommand{\ge}{\geqslant}
\renewcommand{\leq}{\leqslant}
\renewcommand{\geq}{\geqslant}

% Множества
\def \R{\mathbb{R}}
\def \N{\mathbb{N}}
\def \Z{\mathbb{Z}}

\def \a{\alpha}
\def \be{\beta}

% Другое

\newcommand{\const}{\mathrm{const}}
\theoremstyle{definition}
\newtheorem{Task}{Задача}
%\newtheorem*{Taskn}{Задача {#1}}
\newtheorem*{Sol}{Решение}
\usepackage[dvipsnames]{xcolor}

\newcommand{\heart}{\begin{center}
\textcolor{RoyalPurple}{\ensuremath\heartsuit} 
\end{center}}
\usepackage{mathtools}
\usepackage{nicefrac}

%% Гиперссылки
\usepackage{xcolor}
\usepackage{hyperref}
\definecolor{linkcolor}{HTML}{8b00ff}
\hypersetup{colorlinks = true,
			linkcolor = linkcolor,
			urlcolor = linkcolor,
			citecolor = linkcolor}

%% Выравнивание
% \setlength{\parskip}{0.5em} % Расстояние между абзацами
\usepackage{geometry} % Поля
\geometry{
	a4paper,
	left=12mm,
	top=10mm,
	right=12mm}
% \setlength{\parindent}{0cm} % Отступ (красная строка)
% \linespread{1.1} % Интерлиньяж
\usepackage[many]{tcolorbox}  

%% Оформление

% Код
\newcommand{\code}[1]{{\tt #1}}
\setcounter{MaxMatrixCols}{20}

\newenvironment{amatrix}[2]{%
    \left(\begin{array}{@{}*{#1}{c}|*{#2}{c}@{}}
}{%
    \end{array}\right)
}

\begin{document}

\begin{center}
\small
\noindent\makebox[\textwidth]{Линейная алгебра и геометрия \hfill ФКН НИУ ВШЭ, 2022/2023 учебный год, 1-й курс ОП ПМИ, группа 2212}
\end{center}

\large

\begin{center}
\textbf{Семинар 5 (4.10.2022)}
\end{center}

\textbf{Краткое содержание}

Новая тема -- перестановки. Обсудили, как представлять перестановки визуально.
Обсудили инверсии, знак и чётность, произведение перестановок, тождественную перестановку и обратную перестановку.
Обсудили, как решать уравнения вида $\sigma x=\rho, x\sigma=\rho, \sigma x \tau=\rho$ относительно неизвестной перестановки~$x$.
(В этих случаях искомая перестановка выражается формулами $x = \sigma^{-1}\rho, x = \rho\sigma^{-1}, x=\sigma^{-1}\rho\tau^{-1}$ соответственно.)
На доске перемножили две перестановки $\begin{pmatrix} 1 & 2 & 3 & 4 & 5 \\ 4 & 3 & 5 & 1 & 2 \end{pmatrix}, \begin{pmatrix} 1 & 2 & 3 & 4 & 5 \\ 3& 4 & 5 & 1 & 2  \end{pmatrix}$
в обоих порядках, для первой нашли обратную.

Проговорили, что такое циклы.
На примере перестановки
\[
\begin{pmatrix} 1 & 2 & 3 & 4 & 5 & 6 & 7 & 8 & 9 \\ 5 & 8 & 9 & 2 & 1 & 4 & 3 & 6 & 7 \end{pmatrix}
\]
обсудили разложение перестановок в произведение \textbf{независимых} циклов. Поскольку циклы в таком разложении коммутируют, $m$-я степень перестановки есть просто произведение $m$-х степеней всех циклов, и тогда задача возведения произвольной перестановки в степень сводится к той же задаче для циклов.
Если цикл имеет длину~$k$, то его $k$-я степень есть $id$ (тождественная перестановка), поэтому для вычисления его $m$-й степени достаточно возвести этот цикл в степень $m \ \mathrm{mod}\, k$.
Важно помнить, что нулевая степень любой перестановки есть~$id$.
Нашли в явном виде 100-ю, 101-ю и 102-ю степени указанной выше перестановки.

\textbf{Важное дополнение к семинару!!!}

Изначально вычисление знака перестановки определялось, как $sgn(\sigma) =(-1)^{\text{\#инверсий}}$. Когда перестановки большие, число инверсий искать тяжелее -- нужно перебирать много ($ \frac{n(n-1)}{2}$) пар элементов. При разложении перестановки на независимые циклы, подсчет знака сильно упрощается. Знак определеятся как $sgn(\sigma) = (-1)^{dec(\sigma)}$, где $dec(\sigma)$ -- декремент подстановки. Он равен $n - \text{\#циклов} - \text{\#неподвижных точек}$. Если вы разложили перестановку на независимые циклы, декремент и, следовательно, знак считать очень легко.

Пример: перестановка $\sigma = \begin{pmatrix} 1 & 2 & 3 & 4 & 5 & 6 & 7 & 8 & 9 \\ 4 & 1 & 2 & 3 & 7 & 5 & 6 & 8 & 9 \end{pmatrix}$ раскладывается на независимые циклы \\
$(1432)(576)(8)(9)$. 8 и 9 -- это неподвижные точки, так как они все время переходят сами в себя. А цикла в этой перестановке всего два -- (1432) и (576). Таким образом, $dec(\sigma) = 9 - 2 - 2 = 5$. Значит, $sgn(\sigma) = (-1)^5 = -1 \implies \sigma$ -- нечетная.

\heart

\textbf{Домашнее задание к семинару 6. Дедлайн 12.10.2022}

Номера с пометкой П даны по задачнику Проскурякова, с пометкой К -- Кострикина.

\begin{enumerate}
    \item К3.1(б,в)

    \item К3.2(а--г), К3.3(а,б)
    
    \item Вычислить число инверсий в следующих перестановках:\\
    1) $\begin{pmatrix} 
    1 & 2 & 3 & 4 & \dots & n & n+1 & n+2 & n+3 & n+4 &  \dots & 2n \\
    1 & 3 & 5 & 7 & \dots & 2n-1 & 2 & 4 & 6 & 8 & \dots & 2n\end{pmatrix}$\\
    2) $\begin{pmatrix}
    1 & 2  & \dots & n-k+1 & n-k+2 & n-k+3& \dots & n\\
    k & k+1 & \dots & n & 1 & 2 & \dots & k-1
    \end{pmatrix}$.
    
    \item К3.6(а--г)

    \item Решите следующее уравнение относительно неизвестной перестановки $X$:
    \[
    \left(
    \begin{array}{cccccccc}
    1 & 2 & 3 & 4 & 5 & 6 & 7 \\
    6 & 7 & 4 & 2 & 1 & 3 & 5
    \end {array}
    \right)
    \cdot X \cdot
    \left(
    \begin{array}{cccccccc}
    1 & 2 & 3 & 4 & 5 & 6 & 7 \\
    3 & 4 & 6 & 7 & 2 & 1 & 5
    \end{array}
    \right)
    =
    \left(
    \begin{array}{cccccccc}
    1 & 2 & 3 & 4 & 5 & 6 & 7 \\
    7 & 1 & 4 & 6 & 3 & 5 & 2
    \end {array}
    \right).
    \]
    
    \item Вычислите $\sigma^{77}$, $\sigma^{90}$, $\sigma^{119}$, $\sigma^{148}$, если $\sigma = \begin{pmatrix} 1 & 2 & 3 & 4 & 5 & 6 & 7 & 8 & 9 & 10 & 11 & 12 \\ 5 & 9 & 11 & 3 & 8 & 10 & 2 & 12 & 7 & 4 & 6 & 1 \end{pmatrix}$.
    
    \item На сколько может измениться число инверсий в перестановке $\sigma = \begin{pmatrix} 1 & 2 & 3 & \ldots & n \\ \sigma(1) & \sigma(2) & \sigma(3) & \ldots & \sigma(n) \end{pmatrix}$, если в её нижней строке поменять местами два соседних элемента? А как при этом изменится её знак?

    \item Пусть $\rho$~--- перестановка, полученная из перестановки $\sigma$ предыдущей задачи путём перестановки элементов $\sigma(i)$ и $\sigma(j)$ в нижней строке. Опишите перестановку $\sigma^{-1} \rho$.
\end{enumerate}
\heart
    
\end{document}
