\documentclass[10pt, a4paper]{extarticle}

%% Язык
\usepackage{cmap} % Поиск в PDF
\usepackage{mathtext} % Кириллица в формулах
\usepackage[T2A]{fontenc} % Кодировка
\usepackage[utf8]{inputenc} % Кодировка
\usepackage[english,russian]{babel} % Локализация, переносы
\usepackage{bbold} % для ажурных буковок

\pagestyle{empty} \textwidth=19.0cm \oddsidemargin=-1.3cm
\textheight=26cm \topmargin=-3.0cm

%% Математика
\usepackage{amsmath, amsfonts, amssymb, amsthm, mathtools}
\usepackage{icomma}


% Операторы
\DeclareMathOperator{\tr}{tr}
\renewcommand{\le}{\leqslant}
\renewcommand{\ge}{\geqslant}
\renewcommand{\leq}{\leqslant}
\renewcommand{\geq}{\geqslant}

% Множества
\def \R{\mathbb{R}}
\def \N{\mathbb{N}}
\def \Z{\mathbb{Z}}
\newcommand{\rk}{\operatorname{\mathrm{rk}}}
\newcommand{\Ker}{\mathop{\mathrm{Ker}}}
\renewcommand{\Im}{\mathop{\mathrm{Im}}}

\def \a{\alpha}
\def \be{\beta}

% Другое

\newcommand{\const}{\mathrm{const}}
\theoremstyle{definition}
\newtheorem*{proposal}{Предложение}
\newtheorem{Task}{Задача}
%\newtheorem*{Taskn}{Задача {#1}}
\newtheorem*{Sol}{Решение}
\usepackage[dvipsnames]{xcolor}

\newcommand{\heart}{\begin{center}
\textcolor{RoyalPurple}{\ensuremath\heartsuit} 
\end{center}}
\usepackage{mathtools}
\usepackage{nicefrac}

%% Гиперссылки
\usepackage{xcolor}
\usepackage{hyperref}
\definecolor{linkcolor}{HTML}{8b00ff}
\hypersetup{colorlinks = true,
			linkcolor = linkcolor,
			urlcolor = linkcolor,
			citecolor = linkcolor}

%% Выравнивание
% \setlength{\parskip}{0.5em} % Расстояние между абзацами
\usepackage{geometry} % Поля
\geometry{
	a4paper,
	left=12mm,
	top=10mm,
    bottom=20mm,
	right=12mm}
% \setlength{\parindent}{0cm} % Отступ (красная строка)
% \linespread{1.1} % Интерлиньяж
\usepackage[many]{tcolorbox}  
\usepackage{enumitem}

%% Оформление

% Код
\newcommand{\code}[1]{{\tt #1}}

\newenvironment{amatrix}[2]{%
    \left(\begin{array}{@{}*{#1}{c}|*{#2}{c}@{}}
}{%
    \end{array}\right)
}

\begin{document}

\begin{center}
\small
\noindent\makebox[\textwidth]{Линейная алгебра и геометрия \hfill ФКН НИУ ВШЭ, 2022/2023 учебный год, 1-й курс ОП ПМИ, группа 2212}
\end{center}

\large

\begin{center}
\textbf{Семинар 19 (7.02.2023)}
\end{center}

\textbf{Краткое содержание}

Из ДЗ разобрали три последних номера.

Новая тема --- билинейные формы.
Разобрали понятие матрицы билинейной формы по отношению к заданному базису и формулу для вычисления значений билинейной формы в координатах.
Обсудили, как по виду билинейной формы в координатах восстановить её матрицу по отношению к рассматриваемому базису.
Посчитали матрицу билинейной формы $\beta(f,g) = f(2)\cdot g'(2)$ на пространстве $\R[x]_{\leqslant 2}$ в базисе $(1,x, x^2)$.

Формула изменения матрицы билинейной формы при замене базиса: пусть $\mathbb e, \mathbb e'$ --- два базиса пространства $V$, $C$ --- матрица перехода от $\mathbb e$ к $\mathbb e'$ (то есть $\mathbb e' = \mathbb e \cdot C$) и $B$ (соответственно $B'$) --- матрица билинейной формы $\beta$ в базисе $\mathbb e$ (соответственно $\mathbb e'$); тогда $B' = C^TBC$.

Дальше обсудили следующий вопрос: существует ли для билинейной формы $x_1y_2 + x_2y_1 - 3x_2y_2 +3x_2y_3 - 3x_3y_2$ базис, в котором её матрица диагональна?
Ответ отрицательный: так как данная форма не симметрична, то в любом базисе её матрица должна быть несимметрична (свойство симметричности матрицы билинейной формы сохраняется при замене базиса, что можно легко увидеть из явной формулы).

Следующий сюжет --- симметричный алгоритм Гаусса диагонализации симметричной билинейной формы, который базируется на следующем соображении. 
Пусть к некоторой матрице $X \in M_n$ применили одно элементарное преобразование строк и получили матрицу $Y$. Тогда $Y = UX$ для некоторой
элементарной матрицы $V$. Транспонировав последнее равенство, получаем $Y^T = X^TU^T$. Но $Y^T$ получается из $X^T$ ровно таким же элементарным 
преобразованием столбцов. Следовательно, это элементарное преобразование столбцов реализуется при помощи умножения справа на матрицу $U^T$. 
Теперь предположим, что $X$ — матрица некоторой симметричной билинейной формы в каком-то базисе. Тогда $UXU^T$ — это матрица той же формы в 
другом базисе, и она получается из $U$ «симметричным» элементарным преобразованием матрицы $X$ (то есть мы делаем какое-то элементарное 
преобразование строк и затем такое же элементарное преобразование столбцов). Заметим, что при таком преобразовании матрицей перехода от 
старого базиса к новому будет $U^T$. Ввиду симметричности матрицы X, выполняя цепочку симметричных элементарных преобразований матрицы формы, 
её можно привести к диагональному виду; обсудили общий алгоритм для этого («симметричный алгоритм Гаусса»).

Данный алгоритм можно модифицировать таким образом, чтобы наряду с диагональным видом матрицы он выдавал ещё и матрицу перехода к новому базису. 
Для этого нужно к матрице $X$ приписать справа $E$ и во время работы алгоритма все элементарные преобразования строк применять ко всей большой матрице 
$(X\mid E)$, а элементарные преобразования столбцов — только к $X$. Алгоритм заканчивается, когда пара $(X \mid E)$ преобразована к виду $(D\mid P)$, 
где матрица $D$ диагональна. Из конструкции получается, что $D = PXP^T$; значит, матрицей перехода к новому базису будет $P^T$.

Применили симметричный метод Гаусса к матрице 
$\begin{pmatrix}
    1 & -3 & 2 \\ -3 & 7 & -5 \\ 2 & -5 & 8 
\end{pmatrix}$,
нашли диагональный вид и новый базис, в котором матрица диагональна.

\heart

\textbf{Домашнее задание к семинару 20. Дедлайн 14.02.2023}

Номера с пометкой П даны по задачнику Проскурякова, с пометкой К -- Кострикина.

В обоих задачниках координаты векторов из $\R^n$ всегда записываются в строчку через запятую, однако нужно помнить, что мы всегда записываем эти координаты в столбец.

В первых двух задачах предполагается, что все матрицы квадратны порядка $n$.
Линейность функции $\beta(x,y)$ по первому аргументу эквивалентна условию
\[
\beta(\alpha_1x_1 +\alpha_2x_2,y) = \alpha_1\beta(x_1,y) + \alpha_2\beta(x_2,y)
\]
(где $\alpha_1,\alpha_2$ --- скаляры, а $x_1,x_2,y$ --- векторы), часто бывает удобнее проверять именно его.
Аналогично с линейностью по второму аргументу.

\begin{enumerate}

    \item
К37.1(б,в,е,ж) (<<функция>> =  <<форма>>)

\item
К37.1(г,д,з) (<<функция>> =  <<форма>>)

\item
Докажите, что функция $\beta(f,g) = \int \limits_{-1}^1 f(t)g'(t)\,dt$ является билинейной формой на пространстве $\R[x]_{\leqslant 3}$.
Найдите матрицу этой билинейной формы в базисе $(1,x,x^2,x^3)$.

\item
Для каждой из билинейных форм $\beta_1(A,B) = \tr(AB)$ и $\beta_2(A,B) = \tr(A^TB)$ на пространстве $\mathrm M_2(\R)$ найдите её матрицу в базисе из матричных единиц.

\item
К37.6(а), К37.8(а)

\item Билинейная форма $\beta$ на трехмерном векторном пространстве $V$ над $\R$ в базисе $(e_1, e_2, e_3)$ имеет матрицу $B =
\begin{pmatrix}
    1 & 2 & 1 \\ 2 & 1 & 1 \\ 1 & 1 & 3
\end{pmatrix}
$. Найдите базис пространства $V$, в котором форма $\beta$ имеет диагональную матрицу, и выпишите эту матрицу.

\item Тот же вопрос для матрицы  $B =
\begin{pmatrix}
    0 & 1 & 1 \\ 1 & 2 & 3\\ 1 & 3 & 4
\end{pmatrix}
$.

\item Тот же вопрос для матрицы  $B =
\begin{pmatrix}
    0 & 1 & -1 \\ 1 & 0 & 1 \\ -1 & 1 & 0
\end{pmatrix}
$.

\item Модифицируйте симметричный алгоритм Гаусса так, чтобы он мог приводить данную целочисленную матрицу к диагональному виду, не выходя из области целых чисел.

\item (бонус) Пусть $f_1, f_2$ -- две линейные функции на пространстве $V$. Тогда легко видеть, что функция $g(x, y) := f_1(x)\cdot f_2(y)$
является билинейной формой на  $V$. Докажите, что билинейная форма $\beta$ (не обязательно симметричная) на $V$ представляется в виде $\beta(x, y) = f_1(x)\cdot f_2(y)$
(где $f_1, f2$ -- ненулевые функции) тогда и только тогда, когда $\rk \beta =1$. (Напомним, что рангом билинейной формы называется ранг ее матрицы в каком-то базисе).

\end{enumerate}
\heart


\end{document}
