\documentclass[10pt, a4paper]{extarticle}

%% Язык
\usepackage{cmap} % Поиск в PDF
\usepackage{mathtext} % Кириллица в формулах
\usepackage[T2A]{fontenc} % Кодировка
\usepackage[utf8]{inputenc} % Кодировка
\usepackage[english,russian]{babel} % Локализация, переносы
\usepackage{bbold} % для ажурных буковок

\pagestyle{empty} \textwidth=19.0cm \oddsidemargin=-1.3cm
\textheight=26cm \topmargin=-3.0cm

%% Математика
\usepackage{amsmath, amsfonts, amssymb, amsthm, mathtools}
\usepackage{icomma}


% Операторы
\DeclareMathOperator{\tr}{tr}
\renewcommand{\le}{\leqslant}
\renewcommand{\ge}{\geqslant}
\renewcommand{\leq}{\leqslant}
\renewcommand{\geq}{\geqslant}

% Множества
\def \R{\mathbb{R}}
\def \N{\mathbb{N}}
\def \Z{\mathbb{Z}}
\newcommand{\rk}{\operatorname{\mathrm{rk}}}
\newcommand{\Ker}{\mathop{\mathrm{Ker}}}
\renewcommand{\Im}{\mathop{\mathrm{Im}}}

\def \a{\alpha}
\def \be{\beta}

% Другое

\newcommand{\const}{\mathrm{const}}
\theoremstyle{definition}
\newtheorem*{proposal}{Предложение}
\newtheorem{Task}{Задача}
%\newtheorem*{Taskn}{Задача {#1}}
\newtheorem*{Sol}{Решение}
\usepackage[dvipsnames]{xcolor}

\newcommand{\heart}{\begin{center}
\textcolor{RoyalPurple}{\ensuremath\heartsuit} 
\end{center}}
\usepackage{mathtools}
\usepackage{nicefrac}

%% Гиперссылки
\usepackage{xcolor}
\usepackage{hyperref}
\definecolor{linkcolor}{HTML}{8b00ff}
\hypersetup{colorlinks = true,
			linkcolor = linkcolor,
			urlcolor = linkcolor,
			citecolor = linkcolor}

%% Выравнивание
% \setlength{\parskip}{0.5em} % Расстояние между абзацами
\usepackage{geometry} % Поля
\geometry{
	a4paper,
	left=12mm,
	top=10mm,
    bottom=20mm,
	right=12mm}
% \setlength{\parindent}{0cm} % Отступ (красная строка)
% \linespread{1.1} % Интерлиньяж
\usepackage[many]{tcolorbox}  
\usepackage{enumitem}

%% Оформление

% Код
\newcommand{\code}[1]{{\tt #1}}

\newenvironment{amatrix}[2]{%
    \left(\begin{array}{@{}*{#1}{c}|*{#2}{c}@{}}
}{%
    \end{array}\right)
}

\begin{document}

\begin{center}
\small
\noindent\makebox[\textwidth]{Линейная алгебра и геометрия \hfill ФКН НИУ ВШЭ, 2022/2023 учебный год, 1-й курс ОП ПМИ, группа 2212}
\end{center}

\large

\begin{center}
\textbf{Семинар 20 (14.02.2023)}
\end{center}

\textbf{Краткое содержание}

Новая тема --- квадратичные формы.
Если на векторном пространстве $V$ задана билинейная форма $\beta$, то функция $Q_\beta(x) = \beta(x,x)$ называется квадратичной формой, ассоциированной с билинейной формой $\beta$.
Если $B = (b_{ij})$ --- матрица билинейной формы $\beta$ в каком-либо базисе $\mathbb e = (e_1,\ dots, e_n)$, то в координатах в этом же базисе получаем
\begin{equation} \label{qform}
Q_\beta(x) = \sum \limits_{i=1}^nb_{ii}x_i^2 + \sum \limits_{1 \le i<j \le n}(b_{ij} + b_{ji})x_ix_j.
\end{equation}
Отображение $\beta \mapsto Q_\beta$ из множества билинейных форм на $V$ в множество квадратичных форм на $V$ является сюръективным, однако оно не инъективно: одно и то же значение суммы $b_{ij} + b_{ji}$ может получаться при различных значениях слагаемых.
Как было показано на лекциях, если ограничить данное отображение на множество симметричных билинейных форм, то тогда оно станет биективным (при условии, что $1+1 \ne 0$ в поле $F$).
Таким образом, имеется естественное взаимно однозначное соответствие между симметричными билинейными формами и квадратичными формами.

В обозначениях выше, если билинейная форма симметрична (то есть $b_{ij} = b_{ji}$ при всех $i,j$), то формула (\ref{qform}) принимает вид
\begin{equation*} 
Q_\beta(x) = \sum \limits_{i=1}^nb_{ii}x_i^2 + \sum \limits_{1 \le i<j \le n}2b_{ij}x_ix_j.
\end{equation*}
Отсюда видно, что по квадратичной форме легко выписывается матрица соответствующей ей симметричной билинейной формы (только не забывайте про коэффициент 2 в слагаемых вида $2b_{ij}x_ix_j$ при $i \ne j$!).
Эта матрица и называется матрицей квадратичной формы в рассматриваемом базисе.
Пример: матрица квадратичной формы $x_1^2 + 4x_1x_2- 3x_2^2$ есть $\begin{pmatrix} 1 & 2 \\ 2 & -3 \end{pmatrix}$.

Далее поговорили про канонический вид квадратичной формы и про способы его нахождения. Ранее упомянутый симметричный метод Гаусса можно использовать для приведения к каноническому виду, вдобавок 
этот метод дает еще и матрицу перехода от старого базиса к новому. 

Еще один метод, приводящий к каноническому виду  --- метод Якоби: если $\delta_1,\dots,\delta_n$ --- угловые миноры матрицы квадратичной формы и все они отличны от нуля, то квадратичную форму можно привести к каноническому виду $\delta_1 x_1^2 + \frac{\delta_2}{\delta_1}x_2^2 + \ dots + \frac{\delta_n}{\delta_{n-1}}x_n^2$.
Упомянули, что на самом деле метод Якоби работает и в случае $\delta_n = 0$ (но все остальные угловые миноры по-прежнему должны быть отличны от нуля!).
Применили метод Якоби к нахождению канонического вида квадратичной формы из П1175.
Дальше обсудили, что если из этой квадратичной формы выкинуть слагаемое $x_1^2$, то к новой квадратичной форме метод Якоби неприменим, однако он станет применимым, если занумеровать переменные в обратном порядке.
В общем случае можно попробовать придумать какую-то промежуточную замену координат, и это возможно поможет вам избежать нулей в минорах и применить метод.

В конце поговорили про нормальный вид квадратичной формы над полем $\R$ и обсудили, как перейти от канонического вида к нормальному. В случае метода Гаусса приведение к нормальному виду заключается в применениях
симметричных элементарных преобразований третьего вида.
\heart


\textbf{Домашнее задание к семинару 21. Дедлайн 21.02.2023}

Номера с пометкой П даны по задачнику Проскурякова, с пометкой К -- Кострикина.

В обоих задачниках координаты векторов из $\R^n$ всегда записываются в строчку через запятую, однако нужно помнить, что мы всегда записываем эти координаты в столбец.


\begin{enumerate}

    \item
К38.15(а,б) (<<функция>> =  <<форма>>)

\item
К38.16(а,б) (<<функция>> =  <<форма>>)

\item
Определите нормальный вид квадратичной формы из номера П1176.

\item Тот же вопрос для номера П1177.

\item Приведите квадратичную форму из номера П1181 к нормальному виду, выпишите полученный нормальный вид и базис, в котором этот вид принимается, а также соответствующую замену 
координат (выражение старых координат через новые).
\item Выясните, к каким из квадратичных форм номеров П1176, П1177, П1181 применим метод Якоби, и в каждом случае применимости найдите с его помощью нормальный вид соответствующей квадратичной формы.

\item Приведите квадратичную форму $x_1x_2 + x_1x_3 + 2x_2x_3$ к нормальному виду, выпишите полученный нормальный вид и базис, в котором этот вид принимается, а также соответствующую замену 
координат (выражение старых координат через новые).

\item Тот же вопрос для номера П1185.

\end{enumerate}
\heart


\end{document}
