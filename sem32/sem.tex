\documentclass[10pt, a4paper]{extarticle}

%% Язык
\usepackage{cmap} % Поиск в PDF
\usepackage{mathtext} % Кириллица в формулах
\usepackage[T2A]{fontenc} % Кодировка
\usepackage[utf8]{inputenc} % Кодировка
\usepackage[english,russian]{babel} % Локализация, переносы
\usepackage{bbold} % для ажурных буковок

\pagestyle{empty} \textwidth=19.0cm \oddsidemargin=-1.3cm
\textheight=26cm \topmargin=-3.0cm

%% Математика
\usepackage{amsmath, amsfonts, amssymb, amsthm, mathtools}
\usepackage{icomma}
\usepackage{algpseudocode}


% Операторы
\DeclareMathOperator{\tr}{tr}
\renewcommand{\le}{\leqslant}
\renewcommand{\ge}{\geqslant}
\renewcommand{\leq}{\leqslant}
\renewcommand{\geq}{\geqslant}

% Множества
\def \R{\mathbb{R}}
\def \N{\mathbb{N}}
\def \Z{\mathbb{Z}}
\def \CC{\mathbb{C}}
\newcommand{\rk}{\operatorname{\mathrm{rk}}}
\newcommand{\diag}{\operatorname{diag}}
\newcommand{\pr}{\mathop{\mathrm{pr}}}
\newcommand{\ort}{\mathop{\mathrm{ort}}}
\newcommand{\Ker}{\mathop{\mathrm{Ker}}}
\renewcommand{\Im}{\mathop{\mathrm{Im}}}
\newcommand{\Vol}{\mathop{\mathrm{Vol}}}
\newcommand{\Id}{\operatorname{Id}}
\newcommand{\Spec}{\operatorname{Spec}}

\def \a{\alpha}
\def \be{\beta}

% Другое

\newcommand{\const}{\mathrm{const}}
\theoremstyle{definition}
\newtheorem*{proposal}{Предложение}
\newtheorem{Task}{Задача}
%\newtheorem*{Taskn}{Задача {#1}}
\newtheorem*{Sol}{Решение}
\usepackage[dvipsnames]{xcolor}

\newcommand{\heart}{\begin{center}
\textcolor{RoyalPurple}{\ensuremath\heartsuit} 
\end{center}}
\usepackage{mathtools}
\usepackage{nicefrac}

%% Гиперссылки
\usepackage{xcolor}
\usepackage{hyperref}
\definecolor{linkcolor}{HTML}{8b00ff}
\hypersetup{colorlinks = true,
			linkcolor = linkcolor,
			urlcolor = linkcolor,
			citecolor = linkcolor}

%% Выравнивание
% \setlength{\parskip}{0.5em} % Расстояние между абзацами
\usepackage{geometry} % Поля
\geometry{
	a4paper,
	left=12mm,
	top=10mm,
    bottom=20mm,
	right=12mm}
% \setlength{\parindent}{0cm} % Отступ (красная строка)
% \linespread{1.1} % Интерлиньяж
\usepackage[many]{tcolorbox}  
\usepackage{enumitem}

%% Оформление

% Код
\newcommand{\code}[1]{{\tt #1}}

\newenvironment{amatrix}[2]{%
    \left(\begin{array}{@{}*{#1}{c}|*{#2}{c}@{}}
}{%
    \end{array}\right)
}

\begin{document}

\begin{center}
\small
\noindent\makebox[\textwidth]{Линейная алгебра и геометрия \hfill ФКН НИУ ВШЭ, 2022/2023 учебный год, 1-й курс ОП ПМИ, группа 2212}
\end{center}

\large

\begin{center}
\textbf{Семинар 32 (31.05.2023)}
\end{center}

\textbf{Краткое содержание}

Новая тема~--- жорданова нормальная форма (ЖНФ) линейного оператора.
Обсудили понятия корневого вектора, отвечающего данному собственному значению~$\lambda$ линейного оператора~$\varphi$, и соответствующего корневого подпространства $V^\lambda(\varphi) = \{ v\in V \mid \exists \ m \geqslant 0 : (\varphi - \lambda \cdot \Id)^mv=0 \}$, то есть подпространства, состоящего из всех корневых векторов, отвечающих собственному значению~$\lambda$.
Из определения следует включение $V_\lambda(\varphi) \subseteq V^{\lambda}(\varphi)$ для всякого $\lambda \in \Spec \varphi$, то есть всякий собственный вектор автоматически является корневым.

Если характеристический многочлен линейного оператора $\varphi$ разлагается на линейные множители и $\Spec \varphi = \{ \lambda_1, \ldots, \lambda_s \}$, то $V = V^{\lambda_1}(\varphi) \oplus \ldots \oplus V^{\lambda_s}(\varphi)$.
Кроме того, для всякого собственного значения $\lambda \in \Spec \varphi$ имеет место равенство $\dim V^{\lambda}(\varphi) = a_\lambda$ и характеристический многочлен ограничения оператора $\varphi$ на подпространство $V^{\lambda}(\varphi)$ равен $(t - \lambda)^{a_\lambda}$.

Число и размеры жордановых клеток с собственным значением~$\lambda$ в жордановой форме линейного оператора $\varphi$ однозначно определяются действием оператора $\varphi$ в соответствующем корневом подпространстве $V^{\lambda}(\varphi)$.
А~именно, рассмотрим неубывающую цепочку подпространств
\begin{equation} \label{chain}
\{ 0 \} = \Ker (\varphi - \lambda \cdot \Id)^0 \subseteq \Ker (\varphi - \lambda \cdot \Id)^1 \subseteq \Ker (\varphi - \lambda \cdot \Id)^2 \subseteq \ldots.
\end{equation}
Так как объемлющее векторное пространство $V$ конечномерно, то в этой цепочке рано или поздно встретится знак равенства.
Пусть $m \geqslant 0$~--- наименьшее число, для которого $\Ker (\varphi - \lambda \cdot \Id)^m = \Ker (\varphi - \lambda \cdot \Id)^{m+1}$. Тогда начиная с этого места в цепочке~(\ref{chain}) все знаки \guillemotleft$\subseteq$\guillemotright{} на самом деле являются равенствами.
Легко видеть, что тогда $\Ker (\varphi - \lambda \cdot \Id)^m =\nobreak V^{\lambda}(\varphi)$.

Для каждого $i \geqslant 0$ положим $d_i = \dim \Ker (\varphi - \lambda \cdot \Id)^i$.
Тогда имеем цепочку
\[
0 = d_0 < d_1 < \ldots < d_m = a_\lambda,
\]
где $a_\lambda = \dim V^{\lambda}(\varphi)$~--- алгебраическая кратность собственного значения~$\lambda$.
Важно отметить, что цепочка чисел $d_1,d_2,d_3,\ldots$ строго возрастает до тех пор, пока не достигнет значения~$a_\lambda$.
На практике числа $d_i$ вычисляются очень просто: если $A \in \mathrm M_n$~--- матрица линейного оператора~$\varphi$ в каком-либо базисе, то $d_i = n - r_i$, где $r_i= \rk (A - \lambda E)^i$.

На семинаре обсудили, что число $d_1$, оно же размерность собственного подпространства $V_\lambda(\varphi)$, оно же геометрическая кратность собственного значения~$\lambda$, равно количеству жордановых клеток с собственным значением~$\lambda$ в ЖНФ оператора~$\varphi$.
Поэтому  если вдруг обнаружилось, что $d_1 = 1$, то жорданова клетка с собственным значением~$\lambda$ будет одна (размера~$a_\lambda$).
Дальше вывели, что в ЖНФ число жордановых клеток (с собственным значением~$\lambda$) размера $k$ равно $2d_k - d_{k+1} - d_{k-1}$, или же $r_{k-1} + r_{k+1} - 2r_k$.

Получили следующий \textbf{алгоритм поиска жордановой формы}:
\begin{enumerate}
	\item Вычисляем характеристический многочлен оператора и находим его спектр $\Spec \varphi = \{ \lambda_1, \dots, \lambda_s \}$ 
	с алгебраическими кратностями для каждого значения  $\{ a_{\lambda_1}, \dots, a_{\lambda_s} \}$
	\item Для каждого $\lambda_i$ суммарный размер всех клеток в ЖНФ, отвечающих этому собственному значению, равен $a_{\lambda_i}$. 
	Осталось определить, сколько всего таких клеток и какого они размера. 
	\begin{enumerate}
		\item Общее количество клеток равно числу $d_1 = n - \rk (A - \lambda_i E)$
		\item Количество клеток размера ровно $k$ вычисляется через $2d_k - d_{k+1} - d_{k-1}$ или же \;\;\;\;\;\;\;\;\;\;\; $\rk (A - \lambda_i E)^{k-1} + \rk (A - \lambda_i E)^{k+1} - 2 \rk (A - \lambda_i E)^k$
	\end{enumerate}
	С этой информацией можно однозначно определить набор искомых клеток.
\end{enumerate}
Нашли жорданову форму у линейных операторов, имеющих в некотором базисе матрицы
\[
\begin{pmatrix} 1 & 0 & 2 \\ 2 & 0 & 0 \\ 0 & 0 & 1 \end{pmatrix}, \
\begin{pmatrix} 2 & -1 & 2 \\ 1 & 0 & 0 \\ 0 & 0 & 1 \end{pmatrix}, \
\begin{pmatrix} 2 & -1 & 0 \\ 1 & 0 & 0 \\ 0 & 0 & 1 \end{pmatrix}, \
\begin{pmatrix} 2 & -1 & 0 & 0 \\ 1 & 0 & 0 & 1 \\ 0 & 0 & 2 & -1 \\ 0 & 0 & 1 & 0 \end{pmatrix}, \
\begin{pmatrix} 2 & -1 & 0 & 0 \\ 1 & 0 & -1 & 1 \\ 0 & 0 & 2 & -1 \\ 0 & 0 & 1 & 0 \end{pmatrix}.
\]



\heart

\textbf{Домашнее задание к семинару 33. Дедлайн 7.06.2023}

Номера с пометкой П даны по задачнику Проскурякова, с пометкой К -- Кострикина, с пометкой КК -- Ким-Крицкова.

Во всех номерах требуется найти жорданову нормальную форму линейного оператора с заданной матрицей.

\begin{enumerate}
	\item
П1533

\item
П1534

\item
К41.1(е)

\item
$
\begin{pmatrix}
2 & 2 & 1 & 1 & -1 \\
0 & 3 & 1 & -1 & -1 \\
0 & -1 & 1 & 1 & 2 \\
0 & 0 & 0 & 2 & 1 \\
0 & 0 & 0 & 0 & 2
\end{pmatrix}
$

\item
$\begin{pmatrix}
3 & 1 & 1 & -3 & -2 \\
0 & 4 & 1 & -1 & -1 \\
0 & -1 & 2 & 2 & 3 \\
0 & 0 & 0 & 3 & 1 \\
0 & 0 & 0 & 0 & 3
\end{pmatrix}$

\item
П1536

\item
К41.1(и)

\item
К41.1(л)
\end{enumerate}
\heart


\end{document}
