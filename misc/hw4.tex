\documentclass[10pt, a4paper]{extarticle}

%% Язык
\usepackage{cmap} % Поиск в PDF
\usepackage{mathtext} % Кириллица в формулах
\usepackage[T2A]{fontenc} % Кодировка
\usepackage[utf8]{inputenc} % Кодировка
\usepackage[english,russian]{babel} % Локализация, переносы

\pagestyle{empty} \textwidth=19.0cm \oddsidemargin=-1.3cm
\textheight=26cm \topmargin=-3.0cm

%% Математика
\usepackage{amsmath, amsfonts, amssymb, amsthm, mathtools}
\usepackage{icomma}

% Операторы
\DeclareMathOperator{\tr}{tr}
\renewcommand{\le}{\leqslant}
\renewcommand{\ge}{\geqslant}
\renewcommand{\leq}{\leqslant}
\renewcommand{\geq}{\geqslant}

% Множества
\def \R{\mathbb{R}}
\def \N{\mathbb{N}}
\def \Z{\mathbb{Z}}

\def \a{\alpha}
\def \be{\beta}

% Другое

\newcommand{\const}{\mathrm{const}}
\theoremstyle{definition}
\newtheorem{Task}{Задача}
%\newtheorem*{Taskn}{Задача {#1}}
\newtheorem*{Sol}{Решение}
\usepackage[dvipsnames]{xcolor}

\newcommand{\heart}{\begin{center}
\textcolor{RoyalPurple}{\ensuremath\heartsuit} 
\end{center}}
\usepackage{mathtools}
\usepackage{nicefrac}

%% Гиперссылки
\usepackage{xcolor}
\usepackage{hyperref}
\definecolor{linkcolor}{HTML}{8b00ff}
\hypersetup{colorlinks = true,
			linkcolor = linkcolor,
			urlcolor = linkcolor,
			citecolor = linkcolor}

%% Выравнивание
% \setlength{\parskip}{0.5em} % Расстояние между абзацами
\usepackage{geometry} % Поля
\geometry{
	a4paper,
	left=12mm,
	top=10mm,
	right=12mm, 
	bottom=10mm}
% \setlength{\parindent}{0cm} % Отступ (красная строка)
% \linespread{1.1} % Интерлиньяж
\usepackage[many]{tcolorbox}  

%% Оформление

% Код
\newcommand{\code}[1]{{\tt #1}}
\setcounter{MaxMatrixCols}{20}

\newenvironment{amatrix}[2]{%
    \left(\begin{array}{@{}*{#1}{c}|*{#2}{c}@{}}
}{%
    \end{array}\right)
}

\begin{document}

\begin{center}
\small
\noindent\makebox[\textwidth]{Линейная алгебра и геометрия \hfill ПМИ ФКН НИУ ВШЭ, 2022/2023 учебный год, группа 2212}
\end{center}

\large

\begin{center}
\textbf{Домашнее задание 4}
\end{center}

\begin{Task} (Номер 7.)
Для данной матрицы $A = 
    \begin{pmatrix}
     1 & 1 & 0 & 2 \\
     2 & -1 & -1 & 1 \\
     -3 & 3 & 2 & 0\\
    \end{pmatrix}$ найдите такую квадратную матрицу $U$, что матрица $UA$ имеет улучшенный ступенчатый вид, а также выпишите этот вид.
\end{Task}

\begin{Sol}
Как мы знаем, приведение матрицы к ул.ступ. виду происходит за счет элементарных преобразований, что по сути происходит через умножение нашей матрицы $A$ на соответствующую элементарную матрицу слева. Поэтому, приводя матрицу $A$ к ул.ступ. виду через цепочку 

\noindent $A \implies A_1 \implies A_2 \implies \dots \implies A_k$, где $A_k$ - ул.ступ. вид матрицы, на самом деле мы имеем выражения $A \implies U_1 \cdot A \implies U_2 U_1 \cdot A \implies \dots \implies U_k \dots U_2 U_1 \cdot A$. В задаче матрица $U$ -- это и есть $ U_k \dots U_2 U_1$. 

Есть способ найти эту матрицу $U$, который лежит на поверхности и которым воспользовалось большинство из вас. Можно привести матрицу А к у.с. виду, запомнить все те преобразования, которые вы над ней совершили, и перемножить соответствующие матрицы этих преобразований. Но умножать несколько матриц друг на друга это муторно и отстойно, есть большой шанс намесить ошибок.

Есть способ полегче. Вспоминаем, что <<умножить элементарную матрицу на нашу матрицу А>> и <<применить к матрице А элементарное преобразование>> -- это одно и то же действие. Только применять построчные преобразования куда легче. Тогда давайте я буду приводить матрицу к у.с. виду через элементарные \textbf{преобразования}, но рядом с ней запишу еще единичную матрицу $E$ -- $(A \mid E)$. И буду применять преобразования вот к такой расширенной матрице. 

К чему это я клоню? Давайте посмотрим, к чему нас эта конструкция приведет. Я хочу применить к матрице А эл. преобразование $\text{Э}_1$ (неважно, какое оно именно). Применяя это преобразование, я делаю что-то со строчками матрицы $A$ и $E$ и получаю расширенную матрицу $(A_1 \mid W)$, $W$ пока нам неизвестна. Но еще раз как мы помним, применить преобразование по строчкам, это все равно что умножить слева на $U_1$ -- $A_1 = U_1 \cdot A$. Но значит, $W$ это не какая-то там матрица непонятная, а это $U_1 \cdot E$. Значит $W=U_1$.

Применяя второе преобразование  $\text{Э}_2$,  я из  $(A_1 \mid U_1)$ получаю  $(A_2 \mid U_2U_1)$. И тд и тп, пока не дойду до  $(A_k \mid U_k \dots U_1)$, где $A_k$ -- у.с. вид. По завершении этого алгоритма я получу справа от черты мою искомую матрицу $U= U_k \dots U_1$, и при этом я вообще не вспотела даже и ни одной матрицы и не умножила, а только делала операции со строчками матриц. Давайте теперь решим в чиселках, чтобы стало понятнее:

\[
\begin{amatrix}{4}{3}
1 & 1 & 0 & 2 & 1 & 0 & 0 \\
2 & -1 & -1 & 1 & 0 & 1 & 0 \\
-3 & 3 & 2 & 0 & 0 & 0 & 1
\end{amatrix} \implies_{(2) - 2\cdot (1)}
\begin{amatrix}{4}{3}
1 & 1 & 0 & 2 & 1 & 0 & 0 \\
0 & -3 & -1 & -3 & -2 & 1 & 0 \\
-3 & 3 & 2 & 0 & 0 & 0 & 1
\end{amatrix} \implies_{(3)+3\cdot(1)}
\]
\[
\begin{amatrix}{4}{3}
1 & 1 & 0 & 2 & 1 & 0 & 0 \\
0 & -3 & -1 & -3 & -2 & 1 & 0 \\
0 & 6 & 2 & 6 & 3 & 0 & 1
\end{amatrix} \implies_{(3) + 2 \cdot (2)}
\begin{amatrix}{4}{3}
1 & 1 & 0 & 2 & 1 & 0 & 0 \\
0 & -3 & -1 & -3 & -2 & 1 & 0 \\
0 & 0 & 0 & 0 & 1 & 2 & 1
\end{amatrix} \implies_{(2) \cdot \nicefrac{-1}{3}}
\]
\[
\begin{amatrix}{4}{3}
1 & 1 & 0 & 2 & 1 & 0 & 0 \\
0 & 1 & \nicefrac{1}{3} & 1 & \nicefrac{2}{3} & -\nicefrac{1}{3} & 0 \\
0 & 0 & 0 & 0 & 1 & 2 & 1
\end{amatrix} \implies_{(1) - (2)}
\begin{amatrix}{4}{3}
1 & 0 & -\nicefrac{1}{3} & 1 & \nicefrac{1}{3} & \nicefrac{1}{3}& 0 \\
0 & 1 & \nicefrac{1}{3} & 1 & \nicefrac{2}{3} & -\nicefrac{1}{3} & 0 \\
0 & 0 & 0 & 0 & 1 & 2 & 1
\end{amatrix}
\]

Все, я привела матрицу $A$ к у.с. виду, и справа от черты у меня записана результирующая матрица $U$, такая что $UA==$ у.с. вид.

Важно отметить, что 1) элементарная матрица имеет размер 3 на 3, и матрица $U$ также имеет размер 3 на 3; 2) алгоритм гаусса для приведения матрицы к у.с. виду я применяла именно к матрице до черты; то, что справа от черты стоит, я грубо говоря не трогаю; <<не трогаю>> значит, что я не обращаю на ступеньки после черты, меня интересуют ступеньки только до черты. 
\end{Sol}


\end{document}