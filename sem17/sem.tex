\documentclass[10pt, a4paper]{extarticle}

%% Язык
\usepackage{cmap} % Поиск в PDF
\usepackage{mathtext} % Кириллица в формулах
\usepackage[T2A]{fontenc} % Кодировка
\usepackage[utf8]{inputenc} % Кодировка
\usepackage[english,russian]{babel} % Локализация, переносы
\usepackage{bbold} % для ажурных буковок

\pagestyle{empty} \textwidth=19.0cm \oddsidemargin=-1.3cm
\textheight=26cm \topmargin=-3.0cm

%% Математика
\usepackage{amsmath, amsfonts, amssymb, amsthm, mathtools}
\usepackage{icomma}


% Операторы
\DeclareMathOperator{\tr}{tr}
\renewcommand{\le}{\leqslant}
\renewcommand{\ge}{\geqslant}
\renewcommand{\leq}{\leqslant}
\renewcommand{\geq}{\geqslant}

% Множества
\def \R{\mathbb{R}}
\def \N{\mathbb{N}}
\def \Z{\mathbb{Z}}
\newcommand{\rk}{\operatorname{\mathrm{rk}}}
\newcommand{\Ker}{\mathop{\mathrm{Ker}}}
\renewcommand{\Im}{\mathop{\mathrm{Im}}}

\def \a{\alpha}
\def \be{\beta}

% Другое

\newcommand{\const}{\mathrm{const}}
\theoremstyle{definition}
\newtheorem*{proposal}{Предложение}
\newtheorem{Task}{Задача}
%\newtheorem*{Taskn}{Задача {#1}}
\newtheorem*{Sol}{Решение}
\usepackage[dvipsnames]{xcolor}

\newcommand{\heart}{\begin{center}
\textcolor{RoyalPurple}{\ensuremath\heartsuit} 
\end{center}}
\usepackage{mathtools}
\usepackage{nicefrac}

%% Гиперссылки
\usepackage{xcolor}
\usepackage{hyperref}
\definecolor{linkcolor}{HTML}{8b00ff}
\hypersetup{colorlinks = true,
			linkcolor = linkcolor,
			urlcolor = linkcolor,
			citecolor = linkcolor}

%% Выравнивание
% \setlength{\parskip}{0.5em} % Расстояние между абзацами
\usepackage{geometry} % Поля
\geometry{
	a4paper,
	left=12mm,
	top=10mm,
	right=12mm}
% \setlength{\parindent}{0cm} % Отступ (красная строка)
% \linespread{1.1} % Интерлиньяж
\usepackage[many]{tcolorbox}  
\usepackage{enumitem}

%% Оформление

% Код
\newcommand{\code}[1]{{\tt #1}}

\newenvironment{amatrix}[2]{%
    \left(\begin{array}{@{}*{#1}{c}|*{#2}{c}@{}}
}{%
    \end{array}\right)
}

\begin{document}

\begin{center}
\small
\noindent\makebox[\textwidth]{Линейная алгебра и геометрия \hfill ФКН НИУ ВШЭ, 2022/2023 учебный год, 1-й курс ОП ПМИ, группа 2212}
\end{center}

\large

\begin{center}
\textbf{Семинар 17 (24.01.2023)}
\end{center}

\textbf{Краткое содержание}

Из ДЗ разобрали номер 3.

Начали семинар с типовой задачей про прямую сумму: для подпространства $U \subseteq V$ найти дополнительное к нему (т.е. такое 
$W \subseteq V$, что $V = U \oplus W$). Для его нахождения достаточно выбрать базис $(e_1,\dots, e_k)$ в $U$, дополнить его до базиса 
$(e_1,\dots, e_k, e_{k+1}, \dots, e_n)$ всего $V$, и тогда в качестве $W$ можно взять $W = \langle e_{k+1},\dots, e_n \rangle$.
Следует обратить внимание, что это самое $W$ можно выбрать многими способами.

Обсудили описание всех способов, которыми можно линейно независимую систему векторов дополнить до базиса всего пространства.

Новая тема --- линейные отображения векторных пространств.
Поговорили про изоморфизмы, отождествление любого векторного пространства $V$ (над полем $F$) размерности $n$ с пространством $F^n$ 
посредством выбора базиса: если $(e_1,\dots,e_n)$ --- выбранный базис, то соответствие выглядит как
\[
    x_1e_1+ \dots + x_ne_n \leftrightarrow \begin{pmatrix} x_1 \\ \vdots \\ x_n \end{pmatrix}
\]

Проговорили, что всякое линейное отображение $\varphi \colon V \to W$ однозначно определяется образами векторов фиксированного базиса в $V$.
В соответствии с этим линейному отображению $\varphi$ при фиксированных базисах $\mathbb e = (e_1,\dots, e_n)$ в $V$ и $\mathbb f = (f_1,\dots, f_m)$ 
в $W$ сопоставляется матрица $A = A(\varphi, \mathbb e, \mathbb f)$ отображения $\varphi$ в паре базисов $(\mathbb e, \mathbb f)$.
Она определяется соотношением

\[
(\varphi(e_1),\dots, \varphi(e_n)) = (f_1,\dots,f_m) \cdot A,
\]
то есть в её $j$-м столбце стоят координаты вектора $\varphi(e_j)$ в базисе $\mathbb f$.

В качестве примера показали, что если $\varphi \colon \R^2 \to \R^2$ --- поворот на угол $\alpha$ и $\mathbb{e} = \mathbb f = (e_1,e_2)$ --- стандартный 
ортонормированный базис, то матрица $A(\varphi, \mathbb e, \mathbb f)$ равна 
$\begin{pmatrix} \cos \alpha & - \sin \alpha \\ \sin \alpha & \cos \alpha \end{pmatrix}$.

Дальше упомянули, что если $v = x_1e_1+\dots+x_ne_n$ и $\varphi(v) = y_1f_1+ \dots + y_m f_m$, то координаты вектора $v$ и его образа $\varphi(v)$ 
связаны соотношением
\[
\begin{pmatrix} y_1 \\ \vdots \\ y_m \end{pmatrix} = A \cdot \begin{pmatrix} x_1 \\ \vdots \\ x_n \end{pmatrix},
\]
где $A = A(\varphi, \mathbb e, \mathbb f)$.
Таким образом, всякое линейное отображение в координатах предсталяет собой просто умножение на матрицу.

Следующий сюжет: пусть $\mathbb e'$ --- другой базис в $V$ и $\mathbb f'$ --- другой базис в $W$, причём $\mathbb e' = \mathbb e \cdot C$ и 
$\mathbb f' = \mathbb f \cdot D$, где $C$ и $D$ --- соответствующие матрицы перехода; пусть $A = A(\varphi, \mathbb e, \mathbb f)$ и 
$A' = A(\varphi, \mathbb e', \mathbb f')$, тогда справедливо соотношение $A' = D^{-1}AC$.
На эту тему разобрали следующую задачу:

Пусть $V = \R[x]_{\leqslant 2}$ --- пространство многочленов с действительными коэффициентами от переменной $x$ степени не выше $2$. 
Линейное отображение $\varphi \colon V \to \R^2$ в базисе $(x-x^2,x^2,1+2x^2)$ пространства $V$ и базисе $((1,2),(1,3))$ пространства $\R^2$ имеет матрицу
\(
\begin{pmatrix}
3 & -1 & -1 \\
-5 & 2 & 5
\end{pmatrix}.
\)
Найти $\varphi(1+2x+3x^2)$.

Обсудили, что решать задачу можно двумя способами. 

Первый -- разложить многочлен $1+2x+3x^2$ по базису из условия, решив СЛУ и тем самым найти координаты 
в этом базисе; умножить матрицу л.о. на этот вектор координат, тем самым получив вектор координат в базисе $\R^2$ из условия; после этого посчитать вектор.

Второй способ решения -- обозначить базисы из условия как новые $\mathbb e, \mathbb f$, а за старые взять удобные стандартные базисы в обоих пространствах; 
найти матрицы перехода от старых к новым; найти старую матрицу через формулу выше ($A' = D^{-1}AC \implies A = DA'C^{-1}$); умножить полученную старую матрицу л.о. 
на вектор координат в старом базисе (так как он стандартный, то вектор будет (1, 2, 3)), тем самым получив вектор координат в стандартном базисе $\R^2$, 
и так как он стандартный, по сути он и будет являться ответом.

Для всякого линейного отображения $\varphi \colon V \to W$ определяются его ядро $\Ker \varphi = \lbrace v \in V \mid \varphi(v) =\nobreak 0 \rbrace$ и образ 
$\Im \varphi = \varphi(V)$. Из лекций знаем, что $\Ker \varphi$ является подпространством в $V$, а $\Im \varphi$ --- подпространством в $W$.
Обсудили, как находить базис ядра и базис образа, если известна матрица линейного отображения (в какой-либо паре базисов).
Так как в координатах $\varphi$ записывается как $x \mapsto Ax$ (где $A$ --- матрица этого отображения в данной паре базисов), то элементы ядра --- это все решения ОСЛУ $Ax = 0$.
Таким образом, базис ядра --- это просто ФСР для этой ОСЛУ (в координатах!).
Чтобы найти базис образа линейного отображения, можно действовать двумя путями.

1.
Если $(e_1,\dots,e_k)$ --- базис ядра и векторы $e_{k+1}, \dots, e_n$ дополняют его до базиса всего пространства, то тогда векторы $\varphi(e_{k+1}), \dots, \varphi(e_n)$ образуют базис в образе
(этот факт будет доказан на следующей лекции, если вам интересно сейчас, можете пролистать эту пдфку вниз и почитать); важно отметить, что построенная по стандартному алгоритму ФСР очень легко 
дополняется до базиса всего пространства. А именно, пусть $i_1,\dots, i_r$ --- номера главных неизвестных ОСЛУ $Ax=0$, тогда в качестве дополнения нужно взять векторы стандартного базиса в 
$\R^n$ с теми же номерами $i_1,\dots, i_r$. В итоге получается, что базис образа (в координатах!) состоит из столбцов матрицы $A$ с номерами $i_1,\dots, i_r$.

2.
Если в задаче ядро находить не требуется, то можно найти базис образа по-другому: в столбцах матрицы $A$ стоят образы векторов базиса пространства $V$, они всегда порождают образ; 
поэтому базис образа (в координатах!) есть просто базис линейной оболочки столбцов матрицы $A$.
Для нахождения последнего мы знаем два алгоритма, один из которых даёт в точности тот же результат, что и в случае 1.

Пример решения задачи на эту тему:

\noindent Пусть $\dim V = 4$, $\dim W = 3$ и линейное отображение $\varphi \colon V \to W$ в базисе $\mathbb e = (e_1,e_2,e_3,e_4)$ пространства $V$ и базисе $\mathbb f = (f_1,f_2,f_3)$ 
пространства $W$ имеет матрицу
$A = \begin{pmatrix} 1 & 2 & 0 & 1 \\ 2 & 1 & 3 & -1 \\ 1 & 1 & 1 & 0\end{pmatrix}$.
Найти базис ядра и базис образа этого линейного отображения.

(Обратите внимание, что пространства $V$ и $W$ могут не иметь никакого отношения к $F^4$ и $F^3$, а $\mathbb e$ и $\mathbb f$ --- к стандартным базисам!!)

\textbf{Решение}. На всякий случай расшифруем, что по условию нам дано следующее:\\
$\varphi(e_1) = f_1 + 2f_2 + f_3$, $\varphi(e_2) = 2f_1 + f_2 + f_3$, $\varphi(e_3) = 3f_2 + f_3$, $\varphi(e_4) = f_1-f_2$.

Далее, вектор $x_1e_1+x_2e_2+x_3e_3+x_4e_4 \in V$ лежит в ядре отображения $\varphi$ тогда и только тогда, когда набор координат $(x_1,x_2,x_3,x_4)$ является решением ОСЛУ $Ax=0$.
Улучшенный ступенчатый вид матрицы $A$ равен $\begin{pmatrix} 1 & 0 & 2 & -1 \\ 0 & 1 & -1 & 1 \\ 0 & 0 & 0 & 0 \end{pmatrix}$, откуда получаем следующую ФСР для ОСЛУ: $(-2,1,1,0), (1,-1,0,1)$.
Таким образом, базис ядра есть $(-2e_1+e_2+e_3, e_1-e_2+e_4)$.

Чтобы найти базис образа $\varphi$, мы дополняем базис ядра до базиса всего $V$ векторами с координатами $(1,0,0,0)$, $(0,1,0,0)$ (в базисе $\mathbb e$), то есть это просто $e_1$ и $e_2$ 
(соответствуют <<главным неизвестным>> матрицы $A$). В качестве базиса в образе $\varphi$ можно взять образы этих векторов при отображении $\varphi$, то есть векторы с координатами 
$(1,2,1)$, $(2,1,1)$ (в базисе $\mathbb f$), то есть это $f_1+2f_2+f_3$ и $2f_1+f_2+f_3$.

\heart
\newpage
\textbf{Домашнее задание к семинару 18. Дедлайн 31.01.2023}

Номера с пометкой П даны по задачнику Проскурякова, с пометкой К -- Кострикина.

\begin{enumerate}

    \item Пусть $U$ --- подпространство в $\R^4$, натянутое на векторы $(1,1,1,-1), (2,1,1,-2), (0,1,1,0)$.

    (а) Укажите (предъявив базис) какое-нибудь дополнительное к $U$ подпространство $W \subseteq \R^4$ (то есть такое, что $\R^4 = U \oplus W$).

    (б) Укажите (предъявив базис) какое-нибудь другое дополнительное к $U$ подпространство $W' \subseteq\nobreak \R^4$ (обратите внимание, что предъявление разных базисов ещё не означает, что подпространства разные!).


    \item
    В пространстве $\R^4$ даны вектор $v = (1,1,1,1)$ и подпространство $U$, являющееся множеством решений системы
    \[
    \begin{cases} x_1 + x_3 = 0, \\ x_1 + x_2 - 2x_4 = 0. \end{cases}
    \]
    Найдите какое-нибудь подпространство $W \subseteq \R^4$, такое что $\R^4 = U \oplus W$ и проекция вектора $v$ на $U$ вдоль $W$ равна $(1,-1,-1,0)$.


    \item К36.4

    \item
    Рассмотрим отображение $\varphi \colon \mathrm{M}_2(F) \to \mathrm{M}_2(F)$, $X \mapsto X^T$. Докажите, что это отображение линейно, и найдите его матрицу в базисах $\mathbb e$ и $\mathbb f$, где $\mathbb e = \mathbb f$ --- это базис из матричных единиц.

    \item
    Рассмотрим отображение $\varphi \colon \R[x]_{\leqslant 3} \to \R^2$, действующее по правилу $f \mapsto (f(-1), f'(1))$. Докажите, что это отображение линейно, и найдите его матрицу в базисах $\mathbb e$ и $\mathbb f$, где $\mathbb e = (1,x,x^2,x^3)$, а $\mathbb f$ --- стандартный базис в $\R^2$.

    \item
    Пусть векторное пространство $V$ представлено в виде $V = U \oplus W$ для двух подпространств $U,W \subseteq V$.
    Докажите , что отображение $\varphi \colon V \to\nobreak U$, сопоставляющее каждому вектору $v$ его проекцию на $U$ вдоль $W$, является линейным.
    Найдите матрицу этого линейного отображения в паре базисов $(\mathbb e \cup \mathbb f, \mathbb e)$, где $\mathbb e$ --- какой-то базис подпространства $U$, а $\mathbb f$ --- какой-то базис подпространства $W$.

    \item
    К36.3

    \item
    Пусть $V = \R[x]_{\le 2}$ --- пространство многочленов с действительными коэффициентами от переменной $x$ степени не выше $2$. Линейное отображение $\varphi \colon V \to \R^2$ в базисе $(2x+x^2,\ x, \ 1- x)$ пространства $V$ и базисе $((3,2),(1,1))$ пространства $\R^2$ имеет матрицу
    \[
    \begin{pmatrix}
    1 & 1 & -3 \\
    -3 & -1 & 6
    \end{pmatrix}.
    \]
    Найдите $\varphi(3+2x+x^2)$.

    \item
    Линейное отображение $\varphi \colon \R^4 \to \R^3$ в паре стандартных базисов имеет матрицу $\begin{pmatrix} 1 & 1 & 0 & 2 \\ 3 & -3 & 2 & 0 \\ 2 & -1 & 1 & 1 \end{pmatrix}$. Найдите базис ядра и базис образа этого линейного отображения.

    \item
    Найдите базис ядра и базис образа линейного отображения $\varphi \colon \mathrm{M}_2(\R) \to \mathrm{M}_2(\R)$, $\varphi(X) = AX$, где $A = \begin{pmatrix} 1 & 2 \\ 2 & 4 \end{pmatrix}$.


\end{enumerate}
\heart

\begin{proposal}
    Пусть $(e_1, \dots, e_k)$ --- базис $\Ker \varphi$ и векторы $(e_{k + 1}, \dots, e_n)$ дополняют его до базиса всего пространства $V$. Тогда, 
    $(\varphi(e_{k + 1}), \dots, \varphi(e_n))$ образуют базис в $\Im \varphi$.
\end{proposal}

\begin{proof}
    $\Im \varphi = \langle \varphi(e_1), \dots, \varphi(e_k), \varphi(e_{k + 1}), \dots, \varphi(e_n) \rangle = 
    \langle \varphi(e_{k + 1}), \dots, \varphi(e_n) \rangle$ (так как $\varphi(e_1) = \dots = \varphi(e_k) = 0$).
    Осталось показать, что $\varphi(e_{k + 1}), \dots, \varphi(e_n)$ линейно независимы.

    \noindent Пусть $\alpha_{k + 1} \varphi(e_{k + 1}) + \dots + \alpha_n \varphi(e_n) = 0$, где $\alpha_i \in F$.
    Тогда по линейности $\varphi(\alpha_{k + 1} e_{k + 1} + \dots \alpha_n e_n) = 0 \implies \alpha_{k + 1} e_{k + 1} + \dots + \alpha_n e_n \in \ker \varphi$.
    Значит вектор $\alpha_{k + 1} e_{k + 1} + \dots + \alpha_n e_n$ разлагается по базису ядра:

    \[
        \alpha_{k + 1} e_{k + 1} + \dots \alpha_n e_n = \beta_1 e_1 + \dots + \beta_k e_k
    \]
    где $\beta_j \in F$. 

    \noindent Если перенести все в одну сторону, то получится линейная комбинация векторов из базиса $V$. Она равна нулю тогда и только тогда, когда
    все коэффициенты равны нулю, то есть, $\alpha_i = \beta_j = 0 \ \forall i, j$.
\end{proof}

\end{document}
