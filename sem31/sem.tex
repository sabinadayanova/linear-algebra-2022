\documentclass[10pt, a4paper]{extarticle}

%% Язык
\usepackage{cmap} % Поиск в PDF
\usepackage{mathtext} % Кириллица в формулах
\usepackage[T2A]{fontenc} % Кодировка
\usepackage[utf8]{inputenc} % Кодировка
\usepackage[english,russian]{babel} % Локализация, переносы
\usepackage{bbold} % для ажурных буковок

\pagestyle{empty} \textwidth=19.0cm \oddsidemargin=-1.3cm
\textheight=26cm \topmargin=-3.0cm

%% Математика
\usepackage{amsmath, amsfonts, amssymb, amsthm, mathtools}
\usepackage{icomma}
\usepackage{algpseudocode}


% Операторы
\DeclareMathOperator{\tr}{tr}
\renewcommand{\le}{\leqslant}
\renewcommand{\ge}{\geqslant}
\renewcommand{\leq}{\leqslant}
\renewcommand{\geq}{\geqslant}

% Множества
\def \R{\mathbb{R}}
\def \N{\mathbb{N}}
\def \Z{\mathbb{Z}}
\def \CC{\mathbb{C}}
\newcommand{\rk}{\operatorname{\mathrm{rk}}}
\newcommand{\diag}{\operatorname{diag}}
\newcommand{\pr}{\mathop{\mathrm{pr}}}
\newcommand{\ort}{\mathop{\mathrm{ort}}}
\newcommand{\Ker}{\mathop{\mathrm{Ker}}}
\renewcommand{\Im}{\mathop{\mathrm{Im}}}
\newcommand{\Vol}{\mathop{\mathrm{Vol}}}

\def \a{\alpha}
\def \be{\beta}

% Другое

\newcommand{\const}{\mathrm{const}}
\theoremstyle{definition}
\newtheorem*{proposal}{Предложение}
\newtheorem{Task}{Задача}
%\newtheorem*{Taskn}{Задача {#1}}
\newtheorem*{Sol}{Решение}
\usepackage[dvipsnames]{xcolor}

\newcommand{\heart}{\begin{center}
\textcolor{RoyalPurple}{\ensuremath\heartsuit} 
\end{center}}
\usepackage{mathtools}
\usepackage{nicefrac}

%% Гиперссылки
\usepackage{xcolor}
\usepackage{hyperref}
\definecolor{linkcolor}{HTML}{8b00ff}
\hypersetup{colorlinks = true,
			linkcolor = linkcolor,
			urlcolor = linkcolor,
			citecolor = linkcolor}

%% Выравнивание
% \setlength{\parskip}{0.5em} % Расстояние между абзацами
\usepackage{geometry} % Поля
\geometry{
	a4paper,
	left=12mm,
	top=10mm,
    bottom=20mm,
	right=12mm}
% \setlength{\parindent}{0cm} % Отступ (красная строка)
% \linespread{1.1} % Интерлиньяж
\usepackage[many]{tcolorbox}  
\usepackage{enumitem}

%% Оформление

% Код
\newcommand{\code}[1]{{\tt #1}}

\newenvironment{amatrix}[2]{%
    \left(\begin{array}{@{}*{#1}{c}|*{#2}{c}@{}}
}{%
    \end{array}\right)
}

\begin{document}

\begin{center}
\small
\noindent\makebox[\textwidth]{Линейная алгебра и геометрия \hfill ФКН НИУ ВШЭ, 2022/2023 учебный год, 1-й курс ОП ПМИ, группа 2212}
\end{center}

\large

\begin{center}
\textbf{Семинар 31 (24.05.2023)}
\end{center}

\textbf{Краткое содержание}

Тема занятия~--- прямоугольные декартовы системы координат, кривые второго порядка в $\R^2$ и поверхности второго порядка в~$\R^3$, 
теорема о приведении их уравнений к каноническому виду, основные канонические виды и названия соответствующих кривых и поверхностей.

Разобрали номера КК35.24(2,7) и КК35.27(2) и привели кривые к каноничекому виду, используя алгоритм с лекции.


\heart

\textbf{Домашнее задание к семинару 32. Дедлайн 31.05.2023}

Номера с пометкой П даны по задачнику Проскурякова, с пометкой К -- Кострикина, с пометкой КК -- Ким-Крицкова.

В заданиях 1)--8) требуется привести уравнение кривой/поверхности к каноничекому виду в прямоугольной декартовой системе координат и определить её тип.

\begin{enumerate}
	\item
	КК35.24(1,3,5,10)
	
	\item
	КК35.27(6)
	
	\item
	КК35.27(11)
	
	\item
	КК35.27(12)
	
	\item
	КК38.10(1,2,3)
	
	\item
	КК38.12(6)
	
	\item
	КК38.12(8)
	
	\item
	$8x^2-28xy-13y^2 -z^2-44x+2y-37=0$
	
	\item (бонус)
	Докажите, что через каждую точку однополостного гиперболоида проходят ровно две прямые, целиком лежащие на нём.

\end{enumerate}
\heart


\end{document}
