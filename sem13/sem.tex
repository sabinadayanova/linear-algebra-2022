\documentclass[10pt, a4paper]{extarticle}

%% Язык
\usepackage{cmap} % Поиск в PDF
\usepackage{mathtext} % Кириллица в формулах
\usepackage[T2A]{fontenc} % Кодировка
\usepackage[utf8]{inputenc} % Кодировка
\usepackage[english,russian]{babel} % Локализация, переносы

\pagestyle{empty} \textwidth=19.0cm \oddsidemargin=-1.3cm
\textheight=26cm \topmargin=-3.0cm

%% Математика
\usepackage{amsmath, amsfonts, amssymb, amsthm, mathtools}
\usepackage{icomma}

% Операторы
\DeclareMathOperator{\tr}{tr}
\renewcommand{\le}{\leqslant}
\renewcommand{\ge}{\geqslant}
\renewcommand{\leq}{\leqslant}
\renewcommand{\geq}{\geqslant}

% Множества
\def \R{\mathbb{R}}
\def \N{\mathbb{N}}
\def \Z{\mathbb{Z}}
\newcommand{\rk}{\operatorname{\mathrm{rk}}}

\def \a{\alpha}
\def \be{\beta}

% Другое

\newcommand{\const}{\mathrm{const}}
\theoremstyle{definition}
\newtheorem{Task}{Задача}
%\newtheorem*{Taskn}{Задача {#1}}
\newtheorem*{Sol}{Решение}
\usepackage[dvipsnames]{xcolor}

\newcommand{\heart}{\begin{center}
\textcolor{RoyalPurple}{\ensuremath\heartsuit} 
\end{center}}
\usepackage{mathtools}
\usepackage{nicefrac}

%% Гиперссылки
\usepackage{xcolor}
\usepackage{hyperref}
\definecolor{linkcolor}{HTML}{8b00ff}
\hypersetup{colorlinks = true,
			linkcolor = linkcolor,
			urlcolor = linkcolor,
			citecolor = linkcolor}

%% Выравнивание
% \setlength{\parskip}{0.5em} % Расстояние между абзацами
\usepackage{geometry} % Поля
\geometry{
	a4paper,
	left=12mm,
	top=10mm,
	right=12mm}
% \setlength{\parindent}{0cm} % Отступ (красная строка)
% \linespread{1.1} % Интерлиньяж
\usepackage[many]{tcolorbox}  
\usepackage{enumitem}

%% Оформление

% Код
\newcommand{\code}[1]{{\tt #1}}

\newenvironment{amatrix}[2]{%
    \left(\begin{array}{@{}*{#1}{c}|*{#2}{c}@{}}
}{%
    \end{array}\right)
}

\begin{document}

\begin{center}
\small
\noindent\makebox[\textwidth]{Линейная алгебра и геометрия \hfill ФКН НИУ ВШЭ, 2022/2023 учебный год, 1-й курс ОП ПМИ, группа 2212}
\end{center}

\large

\begin{center}
\textbf{Семинар 13 (5.12.2022)}
\end{center}

\textbf{Краткое содержание}

Сначала разобрали следующую типовую задачу: дана линейно независимая система векторов $v_1,\dots, v_m \in F^n$, требуется дополнить её 
до базиса всего пространства $F^n$.

\textbf{ВНИМАНИЕ, 2212 ГРУППА, Я ЭТОТ МЕТОД НА СЕМИНАРЕ ОБЬЯСНИЛА НЕПРАВИЛЬНО, ЧИТАЕМ КОНСПЕКТ И ПОНИМАЕМ!!!}

\textbf{Первый способ (не оптимальный):} 
\begin{enumerate}
    \item приписать к векторам-столбцам $v_1, \dots, v_m$ векторы стандартного базиса $e_1, \dots, e_n$
    \item привести матрицу к ступенчатому виду элементарными преобразованиями \textbf{строк}
    \item Если теперь рассмотреть столбцы, содержащие ведущие элементы строк, то соответствующие им векторы исходной системы 
    $v_1, \dots, v_m, e_1, \dots, e_n$ и будут образовывать искомый базис
\end{enumerate} 
Последнее утверждение справедливо по мотивам предыдущего семинара, где мы с вами повторили, что при элементарных преобразованиях
строк линейные зависимости между столбцами сохраняются.

\textbf{ИТОГО. ГДЕ Я ВАС ОБМАНУЛА}: над получившейся матрицей, где по столбцам записаны $v_1, \dots, v_m, e_1, \dots, e_n$,
надо делать преобразования \textbf{НЕ ПО СТОЛБЦАМ}, а по \textbf{СТРОЧКАМ}.

В качестве примера рассмотрим, как данный алгоритм работает на векторах $v_1 = (1,2,3,4)$ и $v_2 = (2,4,5,6)$:

\noindent 1. приписываю векторы $e_1, \dots, e_4$:
\[
\begin{pmatrix}
    1 & 2 & 1 & 0 & 0 & 0 \\
    2 & 4 & 0 & 1 & 0 & 0 \\
    3 & 5 & 0 & 0 & 1 & 0 \\
    4 & 6 & 0 & 0 & 0 & 1 
\end{pmatrix}    
\]
2. элементарными преобразованиями строк привожу к ступенчатому виду:
\[
\begin{pmatrix}
    1 & 2 & 1 & 0 & 0 & 0 \\
    2 & 4 & 0 & 1 & 0 & 0 \\
    3 & 5 & 0 & 0 & 1 & 0 \\
    4 & 6 & 0 & 0 & 0 & 1 
\end{pmatrix}
\sim
\begin{pmatrix}
    1 & 2 & 1 & 0 & 0 & 0 \\
    0 & -1 & -3 & 0 & 1 & 0 \\
    0 & 0 & 2 & 0 & -2 & 1 \\
    0 & 0 & 0 & 1 & -2 & 1 \\ 
\end{pmatrix}
\]
3. столбцы, содержащие ведущие элементы строк, --- это 1, 2, 3, 4. на 1 и 2 столбцах и так стояли мои линейно независимые $v_1, v_2$.
а вот их дополнением стали векторы на столбцах 3 и 4, то есть $e_1 = (1,0,0,0)$ и $e_2 = (0,1,0,0)$.

\textbf{Второй способ (быстрый):}
Более быстрый алгоритм использует следующее ключевое соображение, уже обсуждавшееся нами ранее: при элементарных преобразованиях 
системы векторов сохраняется её линейная оболочка. Алгоритм будет следующим:
\begin{enumerate}
    \item записать векторы в столбцы матрицы
    \item элементарными преобразованиями \textbf{столбцов} привести матрицу к транспонированному ступенчатому виду
    \item дополнить полученную систему до базиса $F^n$ векторами стандартного базиса, номера которых не 
    являются номерами ведущих элементов столбцов полученного транспонированного ступенчатого вида. другими словами, дополнить теми
    векторами из стандартного базиса, которые <<заполняют недостающие ступеньки вашей матрицы>>
\end{enumerate}

В результате применения данного алгоритма для тех же векторов $v_1 = (1,2,3,4)$ и $v_2 = (2,4,5,6)$ дополняющими до базиса всего $F^4$ 
получаются векторы $e_2 = (0,1,0,0)$ и $e_4 = (0,0,0,1)$. (Обратите внимание, что дополнение до базиса не единственно!)

Дальше перешли к понятию ранга системы векторов и ранга матрицы.
Разобрали базовые свойства ранга матрицы и алгоритм его вычисления для заданной матрицы: 
\begin{enumerate}
    \item привести матрицу к ступенчатому виду элементарными преобразованиями строк
    \item ранг равен числу ненулевых строк в ступенчатом виде
\end{enumerate}

Доказали соотношение $\rk (A+B) \leq \rk A + \rk B$ и вывели из него соотношение $\rk (A-B) \geq \rk A - \rk B$.
Обсудили, что нулевая матрица является единственной матрицей ранга $0$ и что матрицы ранга $1$ --- это в точности матрицы, 
представимые в виде произведения ненулевого столбца на ненулевую строку.

Обсудили, что всякая матрица ранга $r$ представима в виде суммы $r$ матриц ранга $1$ и не представима в виде суммы меньшего числа таких матриц.
Алгоритм нахождения такого представления для конкретной матрицы $A$: 

\begin{enumerate}
    \item найти максимальную линейно независимую систему столбцов $A^{(i_1)}, \dots, A^{(i_r)}$
    \item выразить через неё все остальные столбцы из $A$
    \item записать в $j$-й столбец $k$-го слагаемого ту компоненту разложения столбца $A^{(j)}$, которая пропорциональна столбцу $A^{(i_k)}$.
\end{enumerate}
Применили этот алгоритм для матрицы 
$\begin{pmatrix} 
    1 & 1 & 0 & 2 \\ 
    3 & -3 & 2 & 0 \\ 
    2 & -1 & 1 & 1 
\end{pmatrix}$, 
нашли ее ранг и разложили её в сумму двух матриц ранга 1.


\heart
\textbf{Домашнее задание к семинару 14. Дедлайн 12.12.2022}

Номера с пометкой П даны по задачнику Проскурякова, с пометкой К -- Кострикина.

\begin{enumerate}
    \item К34.14(а,б)
    \item К7.2(б,в).
    \item П624
    \item П628
    \item Докажите, что $\rk(AB) \leq \min(\rk(A), \rk(B))$ для любых матриц $A \in \mathrm{Mat}_{m \times n}(F), B \in \mathrm{Mat}_{n \times p}(F)$.
    \item Найдите ранг $r$ матрицы из номера П608 и представьте её в виде суммы $r$ матриц ранга~$1$.

    \item
    Дана матрица
    $
    A = \begin{pmatrix}
    1 & 0 & 1 & -1 \\
     0 & 1 & 1 & 2 \\
      -1 & 1 & 0 & 3 \\
       1 & 1 & 2 & 1
    \end{pmatrix}
    $.
    Найдите все возможные значения величины $\rk (A+B)$, где $B$ --- матрица того же размера и $\rk B = 1$.
    Ответ обоснуйте.

    \item
    Про матрицу $A$ размера $5 \times 5$ известно, что её ранг равен $3$.
    Какие значения может принимать ранг матрицы $A + E$?


\end{enumerate}
\heart
    
\end{document}
