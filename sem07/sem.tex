\documentclass[10pt, a4paper]{extarticle}

%% Язык
\usepackage{cmap} % Поиск в PDF
\usepackage{mathtext} % Кириллица в формулах
\usepackage[T2A]{fontenc} % Кодировка
\usepackage[utf8]{inputenc} % Кодировка
\usepackage[english,russian]{babel} % Локализация, переносы

\pagestyle{empty} \textwidth=19.0cm \oddsidemargin=-1.3cm
\textheight=26cm \topmargin=-3.0cm

%% Математика
\usepackage{amsmath, amsfonts, amssymb, amsthm, mathtools}
\usepackage{icomma}

% Операторы
\DeclareMathOperator{\tr}{tr}
\renewcommand{\le}{\leqslant}
\renewcommand{\ge}{\geqslant}
\renewcommand{\leq}{\leqslant}
\renewcommand{\geq}{\geqslant}

% Множества
\def \R{\mathbb{R}}
\def \N{\mathbb{N}}
\def \Z{\mathbb{Z}}

\def \a{\alpha}
\def \be{\beta}

% Другое

\newcommand{\const}{\mathrm{const}}
\theoremstyle{definition}
\newtheorem{Task}{Задача}
%\newtheorem*{Taskn}{Задача {#1}}
\newtheorem*{Sol}{Решение}
\usepackage[dvipsnames]{xcolor}

\newcommand{\heart}{\begin{center}
\textcolor{RoyalPurple}{\ensuremath\heartsuit} 
\end{center}}
\usepackage{mathtools}
\usepackage{nicefrac}

%% Гиперссылки
\usepackage{xcolor}
\usepackage{hyperref}
\definecolor{linkcolor}{HTML}{8b00ff}
\hypersetup{colorlinks = true,
			linkcolor = linkcolor,
			urlcolor = linkcolor,
			citecolor = linkcolor}

%% Выравнивание
% \setlength{\parskip}{0.5em} % Расстояние между абзацами
\usepackage{geometry} % Поля
\geometry{
	a4paper,
	left=12mm,
	top=10mm,
	right=12mm}
% \setlength{\parindent}{0cm} % Отступ (красная строка)
% \linespread{1.1} % Интерлиньяж
\usepackage[many]{tcolorbox}  

%% Оформление

% Код
\newcommand{\code}[1]{{\tt #1}}

\newenvironment{amatrix}[2]{%
    \left(\begin{array}{@{}*{#1}{c}|*{#2}{c}@{}}
}{%
    \end{array}\right)
}

\begin{document}

\begin{center}
\small
\noindent\makebox[\textwidth]{Линейная алгебра и геометрия \hfill ФКН НИУ ВШЭ, 2022/2023 учебный год, 1-й курс ОП ПМИ, группа 2212}
\end{center}

\large

\begin{center}
\textbf{Семинар 7 (18.10.2022)}
\end{center}

\textbf{Краткое содержание}

Сначала выяснили, какое наибольшее значение может принимать определитель 3-го порядка при условии, что все его элементы равны $\pm 1$.

Затем проговорили поведение определителя при элементарных преобразованиях строк или столбцов, разобрали определители с углом нулей и 
разложение определителя по строке/столбцу. Обсудили оптимальный алгоритм вычисления определителя с конкретными числовыми элементами: 
при помощи элементарных преобразований строк/столбцов добиться того, чтобы в какой-либо строке (или каком-либо столбце) остался только 
один ненулевой элемент, после чего разложить по этой строке (соответственно столбцу) и свести задачу к определителю меньшего порядка.

Разобрали номера П238, П239, П257.

Следующий сюжет -- определитель Вандермонда и явная формула для него. Проговорили, что в задаче нахождения интерполяционного многочлена
матрица коэффициентов возникающей СЛУ является как раз матрицей Вандермонда, а из неравенства нулю её определителя получается доказательство 
единственности интерполяционного многочлена: определитель не равен нулю $\implies$ можно умножить обе части СЛУ слева на обратную матрицу.

В конце обсудили вопрос о том, как обратить блочную матрицу вида
$\begin{pmatrix} E & U \\ 0 & E \end{pmatrix}$,
где оба блока $E$ квадратны (но могут быть разных размеров). Первый способ нахождения обратной матрицы -- используя элементарные преобразования. 
Второй способ -- используя блочное умножение матриц.
\heart

\textbf{Домашнее задание к семинару 8. Дедлайн 31.10.2022}

Номера с пометкой П даны по задачнику Проскурякова, с пометкой К -- Кострикина.

\begin{enumerate}
    \item П212, П221, П225
    \item Как изменится определитель матрицы, если её <<транспонировать>> (то есть отразить) относительно побочной диагонали?
    \item Как изменится определитель матрицы $A$, если при всех $i,j$ элемент $a_{ij}$ умножить на $c^{i-j}$, где $c \ne 0$ -- некоторое фиксированное число?
    \item П227, П228
    \item П229
    \item П236, П240
    \item П260, П263

    \item Даны матрицы $A, B \in M_4(\R)$. Известно, что $\det A = 1$ и
    \[
    B^{(1)} = 3A^{(3)} - 2A^{(4)}, \;\; B^{(2)} = 2A^{(1)} + 3A^{(2)}, \;\; B^{(3)} = -2A^{(1)} + 2A^{(3)} + A^{(4)}, \;\; B^{(4)} = 2A^{(2)} + 3A^{(4)},
    \]
    где $A^{(i)}$ и $B^{(j)}$ обозначают $i$-й столбец матрицы $A$ и $j$-й столбец матрицы $B$ соответственно. Найдите $\det B$.

    \item Используя блочное умножение матриц, найдите матрицу, обратную к матрице с углом нулей $\begin{pmatrix} A & B \\ 0 & C \end{pmatrix}$,
    где матрицы $A$ и $C$ квадратны и невырожденны (не обязательно одного размера).
\end{enumerate}
\heart
    
\end{document}
