\documentclass[10pt, a4paper]{extarticle}

%% Язык
\usepackage{cmap} % Поиск в PDF
\usepackage{mathtext} % Кириллица в формулах
\usepackage[T2A]{fontenc} % Кодировка
\usepackage[utf8]{inputenc} % Кодировка
\usepackage[english,russian]{babel} % Локализация, переносы

\pagestyle{empty} \textwidth=19.0cm \oddsidemargin=-1.3cm
\textheight=26cm \topmargin=-3.0cm

%% Математика
\usepackage{amsmath, amsfonts, amssymb, amsthm, mathtools}
\usepackage{icomma}

% Операторы
\DeclareMathOperator{\tr}{tr}
\renewcommand{\le}{\leqslant}
\renewcommand{\ge}{\geqslant}
\renewcommand{\leq}{\leqslant}
\renewcommand{\geq}{\geqslant}

% Множества
\def \R{\mathbb{R}}
\def \N{\mathbb{N}}
\def \Z{\mathbb{Z}}

\def \a{\alpha}
\def \be{\beta}

% Другое

\newcommand{\const}{\mathrm{const}}
\theoremstyle{definition}
\newtheorem{Task}{Задача}
%\newtheorem*{Taskn}{Задача {#1}}
\newtheorem*{Sol}{Решение}
\usepackage[dvipsnames]{xcolor}

\newcommand{\heart}{\begin{center}
\textcolor{RoyalPurple}{\ensuremath\heartsuit} 
\end{center}}
\usepackage{mathtools}
\usepackage{nicefrac}

%% Гиперссылки
\usepackage{xcolor}
\usepackage{hyperref}
\definecolor{linkcolor}{HTML}{8b00ff}
\hypersetup{colorlinks = true,
			linkcolor = linkcolor,
			urlcolor = linkcolor,
			citecolor = linkcolor}

%% Выравнивание
% \setlength{\parskip}{0.5em} % Расстояние между абзацами
\usepackage{geometry} % Поля
\geometry{
	a4paper,
	left=12mm,
	top=10mm,
	right=12mm}
% \setlength{\parindent}{0cm} % Отступ (красная строка)
% \linespread{1.1} % Интерлиньяж
\usepackage[many]{tcolorbox}  

%% Оформление

% Код
\newcommand{\code}[1]{{\tt #1}}

\newenvironment{amatrix}[2]{%
    \left(\begin{array}{@{}*{#1}{c}|*{#2}{c}@{}}
}{%
    \end{array}\right)
}

\begin{document}

\begin{center}
\small
\noindent\makebox[\textwidth]{Линейная алгебра и геометрия \hfill ФКН НИУ ВШЭ, 2022/2023 учебный год, 1-й курс ОП ПМИ, группа 2212}
\end{center}

\large

\begin{center}
\textbf{Семинар 10 (14.11.2022)}
\end{center}

\textbf{Краткое содержание}

Продолжили тему векторных пространств и подпространств. Обсудили два способа, как можно задать подпространства в $F^n$:
\begin{enumerate}
    \item $U = \{\vec{x} \in F^n \mid Ax = \vec{0}\}$ -- множество решений заданной ОСЛУ.
    \item $\langle v_1, v_2, \dots, v_m \rangle = \{ \alpha_1 v_1 + \alpha_2 v_2 + \dots + \alpha_m v_m \mid \alpha_i \in F^n \}$ 
    -- \textit{линейная оболочка} векторов $v_1, v_2, \dots, v_m$. Линейная \textit{оболочка} -- это по сути множество ВСЕХ линейных \textit{комбинаций} над 
    векторами $v_1, v_2, \dots, v_m$.
\end{enumerate}

Рассмотрели общую задачу принадлежности вектора подпространству в $F^n$.
Проговорили, что если подпространство $U \subseteq F^n$ задано как множество решений ОСЛУ, то решение данной задачи сводится к подстановке вектора в систему 
и проверке, является ли он её решением. Если же подпространство $U \subseteq F^n$ задано как линейная оболочка векторов $v_1,\ldots, v_m$, 
то задача принадлежности вектора $v_0$ этому подпространству сводится к составлению СЛУ и исследованию вопроса о её совместности.

Разобрали номер П667: выяснить, при каких значениях $\lambda$ вектор $b = (9,12,\lambda)$ принадлежит линейной оболочке векторов $a_1 = (3,4,2)$ и $a_2 = (6,8,7)$.

Разобрали контрольную работу. Посмотрели, что из себя представляет лабораторная работа.

\heart
\textbf{Домашнее задание к семинару 11. Дедлайн 21.11.2022}

Номера с пометкой П даны по задачнику Проскурякова, с пометкой К -- Кострикина.

\begin{enumerate}
    \item К34.1
    \item П665
    \item П668
    \item Пусть $U \subset \R^4$ -- множество решений ОСЛУ
    \[
    \begin{cases}
    x_1 + x_2 + x_3 + x_4 = 0,\\
    x_1 - x_2 + x_3 - x_4 = 0.
    \end{cases}
    \]
    Из общей теории мы знаем, что $U$ является подпространством в $\R^4$.
    Исходя из общего решения данной ОСЛУ, попробуйте найти конечный набор векторов $v_1, \dots, v_m \in \R^4$,
    линейная оболочка которого совпадает с $U$.
\end{enumerate}
\heart
    
\end{document}
