\documentclass[10pt, a4paper]{extarticle}

%% Язык
\usepackage{cmap} % Поиск в PDF
\usepackage{mathtext} % Кириллица в формулах
\usepackage[T2A]{fontenc} % Кодировка
\usepackage[utf8]{inputenc} % Кодировка
\usepackage[english,russian]{babel} % Локализация, переносы

\pagestyle{empty} \textwidth=19.0cm \oddsidemargin=-1.3cm
\textheight=26cm \topmargin=-3.0cm

%% Математика
\usepackage{amsmath, amsfonts, amssymb, amsthm, mathtools}
\usepackage{icomma}

% Операторы
\DeclareMathOperator{\tr}{tr}
\renewcommand{\le}{\leqslant}
\renewcommand{\ge}{\geqslant}
\renewcommand{\leq}{\leqslant}
\renewcommand{\geq}{\geqslant}

% Множества
\def \R{\mathbb{R}}
\def \N{\mathbb{N}}
\def \Z{\mathbb{Z}}

\def \a{\alpha}
\def \be{\beta}

% Другое

\newcommand{\const}{\mathrm{const}}
\theoremstyle{definition}
\newtheorem{Task}{Задача}
%\newtheorem*{Taskn}{Задача {#1}}
\newtheorem*{Sol}{Решение}
\usepackage[dvipsnames]{xcolor}

\newcommand{\heart}{\begin{center}
\textcolor{RoyalPurple}{\ensuremath\heartsuit} 
\end{center}}
\usepackage{mathtools}
\usepackage{nicefrac}

%% Гиперссылки
\usepackage{xcolor}
\usepackage{hyperref}
\definecolor{linkcolor}{HTML}{8b00ff}
\hypersetup{colorlinks = true,
			linkcolor = linkcolor,
			urlcolor = linkcolor,
			citecolor = linkcolor}

%% Выравнивание
% \setlength{\parskip}{0.5em} % Расстояние между абзацами
\usepackage{geometry} % Поля
\geometry{
	a4paper,
	left=12mm,
	top=10mm,
	right=12mm}
% \setlength{\parindent}{0cm} % Отступ (красная строка)
% \linespread{1.1} % Интерлиньяж
\usepackage[many]{tcolorbox}  

%% Оформление

% Код
\newcommand{\code}[1]{{\tt #1}}

\newenvironment{amatrix}[2]{%
    \left(\begin{array}{@{}*{#1}{c}|*{#2}{c}@{}}
}{%
    \end{array}\right)
}

\begin{document}

\begin{center}
\small
\noindent\makebox[\textwidth]{Линейная алгебра и геометрия \hfill ФКН НИУ ВШЭ, 2022/2023 учебный год, 1-й курс ОП ПМИ, группа 2212}
\end{center}

\large

\begin{center}
\textbf{Семинар 4 (27.09.2022)}
\end{center}

\textbf{Краткое содержание}

Обсудили задачу интерполяции: даны $n+1$ попарно различных точек $x_1, x_2, \dots, x_n \in \R$
и набор чисел $y_1, y_2, \dots, y_n \in \R$. Требуется найти многочлен $f(x) = a_0 + a_1x + a_2x^2 + \dots + a_{n-1}x^{n-1}$ такой,
чтобы $f(x_i) = y_i \;\; \forall i = 1\dots n$. Обратите внимание, что количество точек на единицу больше, чем степень многочлена
и совпадает с количеством коэффициентов у многочлена. Условия $y_i = f(x_i)$ дают СЛУ вида

\[
   \begin{pmatrix}
    1 & x_1 & x_1^2 & \dots & x_1^{n-1}\\
    1 & x_2 & x_2^2 & \dots & x_2^{n-1} \\
    \vdots & \vdots & \vdots & \ddots & \vdots \\
    1 & x_n & x_n^2 & \dots & x_n^{n-1} \\
   \end{pmatrix} 
   \begin{pmatrix}
    a_0 \\ a_1 \\ \vdots \\ a_{n-1}
   \end{pmatrix} =
   \begin{pmatrix}
    y_1 \\ y_2 \\ \vdots \\ y_{n}
   \end{pmatrix}
\]
Такая система в случае разных $x_i$ имеет единственное решение. Также была показана явная конструкция построения такого многочлена
через интерполяционные многочлены Лагранжа:
\[
f(x) = \sum_{i=1}^n y_i l_i(x), \;\; 
l_i(x) = \frac{(x-x_1)(x-x_2)\dots (x-x_{i-1}) (x - x_{i+1}) \dots (x- x_n)}{(x_i-x_1)(x_i-x_2)\dots (x_i-x_{i-1}) (x_i - x_{i+1}) \dots (x_i- x_n)}
\]
Дальше поговорили, как оптимизировать решение большого количества СЛУ $Ax=b_1, Ax = b_2, \dots Ax = b_s$ с одной и той же матрицей коэффициентов.
Их не нужно решать по отдельности, часть вычислений можно провести одновременно, выполняя элементарные преобразования строк в матрице $(A \mid B)$, 
где $B$ состалена из столбцов $b_1, b_2, \dots, b_s$. После приведения матрицы $A$ к улучшенному ступенчатому виду уже можно выполнять дальнейший анализ
(сводящийся к выписыванию ответа) каждой из систем по отдельности.

Затем рассмотрели очень похожую, но другую ситуацию: дана матрица коэффициентов $A$, и нам нужно решать большое количество систем вида $Ax=b$,
где столбцы $b$ к нам поступают в режиме реального времени (их полный список неизвестен). В этом случае удобно заранее <<запомнить>>, каким образом
матрица $A$ приводится к улучшенному ступенчатому виду. Для этого вспомним, что каждое элементарное преобразование строк матрицы реализуется при помощи
умножения слева на соответственную <<элементарную>> матрицу. Тогда если $A$ приводится к улучшенному ступенчатому виду $A'$ при помощи элементарных преобразований, 
соответствующих умножению слева на матрицы $U_1, \dots, U_k$, то $A' = U_k \dots U_1A$.
Если обозначить $U = U_k \dots U_1$, то любая система $Ax = b$ после умножения слева на $U$ переходит в эквивалентную систему $A'x = Ub$.
Таким образом, при поступлении очередного столбца $b$ для решения системы $Ax = b$ достаточно вычислить~$Ub$, и уже можно выписывать ответ.
Для оптимального вычисления обеих матриц $A'$ и $U$ нужно записать матрицу $(A \mid E)$ и, выполняя в ней элементарные преобразования строк, привести $A$ к улучшенному ступенчатому виду.
Тогда справа от черты будет стоять~$U$, то есть результирующая матрица как раз и будет $(A' \mid U)$. (Обратите внимание, что должно выполняться соотношение $UA = A'$.)

Дальше обсудили, что решение матричного уравнения вида $AX = B$ (где $A, X, B$ имеют размеры $m \times n, n \times k, m \times k$ соответственно) эквивалентно решению $k$ систем
\[
AX^{(1)} = B^{(1)}, \, AX^{(2)} = B^{(2)}, \, \dots, \, AX^{(k)} = B^{(k)}.
\]
С учётом сказанного выше в этом случае для решения нужно составить матрицу $(A \mid B)$ и, выполняя в ней элементарные преобразования строк, 
привести $A$ к улучшенному ступенчатому виду, из которого уже выписывается общее решение.
Единственный нюанс: исходное уравнение разрешимо тогда и только тогда, когда разрешима каждая из систем $AX^{(i)} = B^{(i)}$.
Проговорили, что матричное уравнение вида $YA = B$ при помощи транспонирования приводится к виду $A^TY^T = B^T$, рассмотренному выше.
Также упомянули, что уравнение вида $AXB = C$ надо решать в два этапа, сначала решая уравнение $YB = C$, а затем $AX = Y$.

Матрицей, обратной к данной матрице $A \in M_n$, называется такая матрица $B \in M_n$, что $AB = BA = E$. Обозначение: $A^{-1}$.
Позже на лекциях будет доказано, что если обратная матрица существует, то она определена однозначно, и что из условия $AB = E$ автоматически следует $BA=E$.
Последнее даёт практический способ нахождения обратной матрицы: она является решением матричного уравнения $AX = E$ (если решение существует).
Для решения этого уравнения нужно взять матрицу $(A \mid E)$ и элементарными преобразованиями строк привести её к виду $(E \mid B)$, тогда $B = A^{-1}$.
В связи с данным методом нахождения обратной матрицы обсудили, что если квадратная матрица $A \in M_n$ приведена к улучшенному ступенчатому виду $A'$, 
то либо $A' = E$, либо последняя строка в $A'$ нулевая. В первом случае матрица $A$ обратима (как следует из алгоритма выше), а вот во втором -- нет.
 

\heart

\textbf{Домашнее задание к семинару 5. Дедлайн 7.10.2022}

Номера с пометкой П даны по задачнику Проскурякова, с пометкой К -- Кострикина.

\begin{enumerate}
    \item П76
   
\end{enumerate}
\heart
    
\end{document}
