\documentclass[10pt, a4paper]{extarticle}

%% Язык
\usepackage{cmap} % Поиск в PDF
\usepackage{mathtext} % Кириллица в формулах
\usepackage[T2A]{fontenc} % Кодировка
\usepackage[utf8]{inputenc} % Кодировка
\usepackage[english,russian]{babel} % Локализация, переносы
\usepackage{bbold} % для ажурных буковок

\pagestyle{empty} \textwidth=19.0cm \oddsidemargin=-1.3cm
\textheight=26cm \topmargin=-3.0cm

%% Математика
\usepackage{amsmath, amsfonts, amssymb, amsthm, mathtools}
\usepackage{icomma}
\usepackage{algpseudocode}


% Операторы
\DeclareMathOperator{\tr}{tr}
\renewcommand{\le}{\leqslant}
\renewcommand{\ge}{\geqslant}
\renewcommand{\leq}{\leqslant}
\renewcommand{\geq}{\geqslant}

% Множества
\def \R{\mathbb{R}}
\def \N{\mathbb{N}}
\def \Z{\mathbb{Z}}
\def \CC{\mathbb{C}}
\newcommand{\rk}{\operatorname{\mathrm{rk}}}
\newcommand{\diag}{\operatorname{diag}}
\newcommand{\pr}{\mathop{\mathrm{pr}}}
\newcommand{\ort}{\mathop{\mathrm{ort}}}
\newcommand{\Ker}{\mathop{\mathrm{Ker}}}
\renewcommand{\Im}{\mathop{\mathrm{Im}}}
\newcommand{\Vol}{\mathop{\mathrm{Vol}}}

\def \a{\alpha}
\def \be{\beta}

% Другое

\newcommand{\const}{\mathrm{const}}
\theoremstyle{definition}
\newtheorem*{proposal}{Предложение}
\newtheorem{Task}{Задача}
%\newtheorem*{Taskn}{Задача {#1}}
\newtheorem*{Sol}{Решение}
\usepackage[dvipsnames]{xcolor}

\newcommand{\heart}{\begin{center}
\textcolor{RoyalPurple}{\ensuremath\heartsuit} 
\end{center}}
\usepackage{mathtools}
\usepackage{nicefrac}

%% Гиперссылки
\usepackage{xcolor}
\usepackage{hyperref}
\definecolor{linkcolor}{HTML}{8b00ff}
\hypersetup{colorlinks = true,
			linkcolor = linkcolor,
			urlcolor = linkcolor,
			citecolor = linkcolor}

%% Выравнивание
% \setlength{\parskip}{0.5em} % Расстояние между абзацами
\usepackage{geometry} % Поля
\geometry{
	a4paper,
	left=12mm,
	top=10mm,
    bottom=20mm,
	right=12mm}
% \setlength{\parindent}{0cm} % Отступ (красная строка)
% \linespread{1.1} % Интерлиньяж
\usepackage[many]{tcolorbox}  
\usepackage{enumitem}

%% Оформление

% Код
\newcommand{\code}[1]{{\tt #1}}

\newenvironment{amatrix}[2]{%
    \left(\begin{array}{@{}*{#1}{c}|*{#2}{c}@{}}
}{%
    \end{array}\right)
}

\begin{document}

\begin{center}
\small
\noindent\makebox[\textwidth]{Линейная алгебра и геометрия \hfill ФКН НИУ ВШЭ, 2022/2023 учебный год, 1-й курс ОП ПМИ, группа 2212}
\end{center}

\large

\begin{center}
\textbf{Семинар 27 (19.04.2023)}
\end{center}

\textbf{Краткое содержание}

Продолжили тему линейных операторов.

Обсудили соответствие между линейными операторами и матрицами, формулу действия линейного оператора в координатах, формулу изменения матрицы линейного оператора при замене базиса.
Обсудили, как изменится матрица линейного оператора при перестановке двух векторов базиса, а также при умножении одного из векторов базиса на ненулевой скаляр.

Определили собственные векторы и собственные значения линейного оператора, спектр, собственные подпространства, характеристический многочлен, а также критерий диагонализуемости.

Для линейного оператора с прошлого семинара посчитали его характеристический многочлен; его единственный корень --- $\lambda = 0$, это единственное собственное значение.
Для этого собственного значения нашли базис соответствующего собственного подпространства.

Определили алгебраическую и геометрическую кратности собственного значения линейного оператора, обсудили связь между ними.

Сформулировали критерий диагонализуемости: линейный оператор $\varphi$ диагонализуем тогда и только тогда, когда выполнены два условия:

1) характеристический многочлен разлагается на линейные множители;

2) для всякого собственного значения геометрическая кратность равна алгебраической.

Решили номера П1472 и П1473, в каждом случае исследовали диагонализуемость, в П1473 нашли также базис, в котором матрица оператора диагональна, соответствующую матрицу перехода и саму диагональную матрицу.

Проговорили алгоритм проверки оператора на диагонализуемость и нахождения диагонального вида и соответствующей матрицы перехода: пусть $A$ --- матрица линейного оператора $\varphi$ в некотором базисе.

1) вычисляем многочлен $\det(A - tE)$ от переменной $t$ (с точностью до знака это характеристический многочлен оператора $\varphi$) и находим все его корни; это будут в точности все собственные значения оператора $\varphi$;
убеждаемся, что многочлен раскладывается на линейные множители со своими корнями: $ \chi_{\varphi}(t) = (x - \lambda_1)^{a_1} \dots (x - \lambda_k)^{a_k}$ 

2) для каждого найденного собственного значения $\lambda_i$ находим базис соответствующего ему собственного подпространства $V_{\lambda_i}(\varphi) = \{ v \in V \mid \varphi(v) = \lambda_i \cdot v \}$ --- это ФСР для ОСЛУ $(A-\lambda_i E)x = 0$;
убеждаемся, что $g_i := dim V_{\lambda_i}(\varphi)$ равно  $a_i$

3) если первые два пункта выполнились, то оператор диагонализуем. В этом случае матрица перехода $C$ будет состоять из координат собственных векторов, которые вы нашли при поиске ФСР. 
Если собственные векторы (а точнее, их координаты) для $\lambda_i$ есть $\{v_{i1}, \dots , v_{ia_i}\}$, то $C = \left( v_{11} \mid \dots \mid v_{1a_1} \mid  v_{21} \mid \dots \mid v_{2a_2} \mid \dots \mid v_{k1} \mid \dots \mid v_{ka_k} \right)$.
В новом базисе новая будет диагональная матрица $C^{-1}AC = diag(\lambda_1, \dots, \lambda_1, \lambda_2, \dots, \lambda_2, \dots, \lambda_k, \dots, \lambda_k)$, где каждое $\lambda_i$ встречается по $a_i$ раз.


\heart

\textbf{Домашнее задание к семинару 28. Дедлайн 26.04.2023}

Номера с пометкой П даны по задачнику Проскурякова, с пометкой К -- Кострикина, с пометкой КК -- Ким-Крицкова.

\begin{enumerate}

	
	\item
П1448 (здесь и далее  <<линейное преобразование>> $=$ <<линейный оператор>>)

\item
П1449

\item
П1452(б)

\item
П1458

Номера 5)--8) решите с такой формулировкой: найти все собственные значения линейного оператора и базисы всех его собственных подпространств. Проверить, что линейный оператор не диагонализуем.


\item
П1465

\item
П1466

\item
П1469

\item \label{matrix}
матрица $\begin{pmatrix} 3 & -1 & -1 \\ 2 & 0 & -2 \\ -1 & 1 & 1\end{pmatrix}$ (линейный оператор рассматривается над $\R$)

В номерах 9)--11) докажите, что линейный оператор с заданной матрицей диагонализуем; найдите базис, в котором его матрица диагональна, выпишите эту матрицу и матрицу перехода к новому базису.

\item
П1481 (над $\R$)

\item
П1483 (над $\R$)

\item 
матрица из номера \ref{matrix} (над $\CC$)

	

\end{enumerate}
\heart


\end{document}
