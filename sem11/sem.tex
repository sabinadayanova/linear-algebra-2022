\documentclass[10pt, a4paper]{extarticle}

%% Язык
\usepackage{cmap} % Поиск в PDF
\usepackage{mathtext} % Кириллица в формулах
\usepackage[T2A]{fontenc} % Кодировка
\usepackage[utf8]{inputenc} % Кодировка
\usepackage[english,russian]{babel} % Локализация, переносы

\pagestyle{empty} \textwidth=19.0cm \oddsidemargin=-1.3cm
\textheight=26cm \topmargin=-3.0cm

%% Математика
\usepackage{amsmath, amsfonts, amssymb, amsthm, mathtools}
\usepackage{icomma}

% Операторы
\DeclareMathOperator{\tr}{tr}
\renewcommand{\le}{\leqslant}
\renewcommand{\ge}{\geqslant}
\renewcommand{\leq}{\leqslant}
\renewcommand{\geq}{\geqslant}

% Множества
\def \R{\mathbb{R}}
\def \N{\mathbb{N}}
\def \Z{\mathbb{Z}}

\def \a{\alpha}
\def \be{\beta}

% Другое

\newcommand{\const}{\mathrm{const}}
\theoremstyle{definition}
\newtheorem{Task}{Задача}
%\newtheorem*{Taskn}{Задача {#1}}
\newtheorem*{Sol}{Решение}
\usepackage[dvipsnames]{xcolor}

\newcommand{\heart}{\begin{center}
\textcolor{RoyalPurple}{\ensuremath\heartsuit} 
\end{center}}
\usepackage{mathtools}
\usepackage{nicefrac}

%% Гиперссылки
\usepackage{xcolor}
\usepackage{hyperref}
\definecolor{linkcolor}{HTML}{8b00ff}
\hypersetup{colorlinks = true,
			linkcolor = linkcolor,
			urlcolor = linkcolor,
			citecolor = linkcolor}

%% Выравнивание
% \setlength{\parskip}{0.5em} % Расстояние между абзацами
\usepackage{geometry} % Поля
\geometry{
	a4paper,
	left=12mm,
	top=10mm,
	right=12mm}
% \setlength{\parindent}{0cm} % Отступ (красная строка)
% \linespread{1.1} % Интерлиньяж
\usepackage[many]{tcolorbox}  

%% Оформление

% Код
\newcommand{\code}[1]{{\tt #1}}

\newenvironment{amatrix}[2]{%
    \left(\begin{array}{@{}*{#1}{c}|*{#2}{c}@{}}
}{%
    \end{array}\right)
}

\begin{document}

\begin{center}
\small
\noindent\makebox[\textwidth]{Линейная алгебра и геометрия \hfill ФКН НИУ ВШЭ, 2022/2023 учебный год, 1-й курс ОП ПМИ, группа 2212}
\end{center}

\large

\begin{center}
\textbf{Семинар 10 (14.11.2022)}
\end{center}

\textbf{Краткое содержание}

Обсудили понятия линейной зависимости и независимости набора векторов.
Выяснили, что для пространства $F^n$ вопрос о линейной зависимости/независимости конечного набора векторов сводится к составлению ОСЛУ
и вопросу о наличии у неё ненулевого решения.

Дальше обсудили, почему  в пространстве всех функций $f: \R \to \R$ система функций 

\noindent $\sin x,  \sin^2 x,  \dots,  \sin^n x$ линейно независима при любом $n$.
Это можно свести к определителю Вандермонда, выбрав $n$ точек, в которых принимаются попарно различные ненулевые значения.
Затем поговорили про систему функций $e^{a_1x}, \dots, e^{a_nx}$ при попарно различных $a_1, \dots, a_n$, разобрали два
разных доказательства линейной независимости (оба сводятся к определителю Вандермонда): один способ -- взять значения в точках 
$0, 1, \dots, n-1$; второй способ -- последовательно дифференцировать приравненную к нулю линейную комбинацию 
(достаточно дифференцировать $n-1$ раз) и каждый раз брать значения в точке $0$ (для удобства).

Следующая тема -- базис и размерность векторного пространства.
Нашли базис и размерность подпространств из номера К35.2(а,б).
Нашли базис и размерность для пространства симметричных квадратных матриц произвольного порядка.

\heart
\textbf{Домашнее задание к семинару 11. Дедлайн 21.11.2022}

Номера с пометкой П даны по задачнику Проскурякова, с пометкой К -- Кострикина.

\begin{enumerate}
    \item
П641, 642

\item
К34.2(а)

\item
П649, П650

\item
П652

\item
П1825 (считать областью определения функций множество $x >0$)

\item
П1826(а,б)

\item
П1828

\item
Найдите базис и размерность для подпространств из номера К35.2(в,г). Ответы обоснуйте.

\item
Пусть $\R[x]_{\leqslant n}$~--- векторное пространство всех многочленов степени не выше~$n$ с действительными коэффициентами.
Пусть $U \subseteq \R[x]_{\leqslant n}$~--- подмножество, состоящее из всех многочленов, имеющих корень $c \in \R$.
Докажите, что $U$ является подпространством, а также найдите его базис и размерность (тоже с доказательством!).
\end{enumerate}
\heart
    
\end{document}
