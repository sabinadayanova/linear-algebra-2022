\documentclass[10pt, a4paper]{extarticle}

%% Язык
\usepackage{cmap} % Поиск в PDF
\usepackage{mathtext} % Кириллица в формулах
\usepackage[T2A]{fontenc} % Кодировка
\usepackage[utf8]{inputenc} % Кодировка
\usepackage[english,russian]{babel} % Локализация, переносы

\pagestyle{empty} \textwidth=19.0cm \oddsidemargin=-1.3cm
\textheight=26cm \topmargin=-3.0cm

%% Математика
\usepackage{amsmath, amsfonts, amssymb, amsthm, mathtools}
\usepackage{icomma}

% Операторы
\DeclareMathOperator{\tr}{tr}
\renewcommand{\le}{\leqslant}
\renewcommand{\ge}{\geqslant}
\renewcommand{\leq}{\leqslant}
\renewcommand{\geq}{\geqslant}

% Множества
\def \R{\mathbb{R}}
\def \N{\mathbb{N}}
\def \Z{\mathbb{Z}}

\def \a{\alpha}
\def \be{\beta}

% Другое

\newcommand{\const}{\mathrm{const}}
\theoremstyle{definition}
\newtheorem{Task}{Задача}
%\newtheorem*{Taskn}{Задача {#1}}
\newtheorem*{Sol}{Решение}
\usepackage[dvipsnames]{xcolor}

\newcommand{\heart}{\begin{center}
\textcolor{RoyalPurple}{\ensuremath\heartsuit} 
\end{center}}
\usepackage{mathtools}
\usepackage{nicefrac}

%% Гиперссылки
\usepackage{xcolor}
\usepackage{hyperref}
\definecolor{linkcolor}{HTML}{8b00ff}
\hypersetup{colorlinks = true,
			linkcolor = linkcolor,
			urlcolor = linkcolor,
			citecolor = linkcolor}

%% Выравнивание
% \setlength{\parskip}{0.5em} % Расстояние между абзацами
\usepackage{geometry} % Поля
\geometry{
	a4paper,
	left=12mm,
	top=10mm,
	right=12mm}
% \setlength{\parindent}{0cm} % Отступ (красная строка)
% \linespread{1.1} % Интерлиньяж
\usepackage[many]{tcolorbox}  

%% Оформление

% Код
\newcommand{\code}[1]{{\tt #1}}
\setcounter{MaxMatrixCols}{20}

\newenvironment{amatrix}[2]{%
    \left(\begin{array}{@{}*{#1}{c}|*{#2}{c}@{}}
}{%
    \end{array}\right)
}

\begin{document}

\begin{center}
\small
\noindent\makebox[\textwidth]{Линейная алгебра и геометрия \hfill ФКН НИУ ВШЭ, 2022/2023 учебный год, 1-й курс ОП ПМИ, группа 2212}
\end{center}

\large

\begin{center}
\textbf{Семинар 6 (11.10.2022)}
\end{center}

\textbf{Краткое содержание}

Сначала обсудили разложение цикла в произведение транспозиций и сделали вывод о его чётности: если цикл имеет длину $k$, то его знак равен $(-1)^{k-1}$.
Применили это к вычислению знака перестановки, разложенной в произведение независимых циклов.
В результате получили правило определения знака перестановки по декременту: $sgn(\sigma) = (-1)^{dec(\sigma)}$.

Затем обсудили разложение произвольной перестановки в произведение транспозиций, а также произведение элементарных транспозиций.


Новая тема -- определители.
Выписали общую формулу определителя квадратной матрицы порядка $n$. Расписали явную формулу определителя для матриц порядка 2 и 3.
Разобрали номера П188 и П189.

Упомянули, с каким знаком входит в определитель произведение элементов главной диагонали, после чего разобрали аналогичный вопрос для побочной диагонали. 
Запутались в побочной диагонали, оставили номер для домашнего задания.

Разобрали номер П200 (не до конца).


\heart

\textbf{Домашнее задание к семинару 7. Дедлайн 18.10.2022}

Номера с пометкой П даны по задачнику Проскурякова, с пометкой К -- Кострикина.

\begin{enumerate}
    \item П157, П158, П159
    \item П1, П2, П4, П6, П9
    \item П44, П47, П58
    \item П190, П191
    \item П197, П198
    \item Вычислите, с каким знаком входит в определитель произведение элементов побочной диагонали 
    матрицы порядка $n$.
    \item Найдите коэффициент при $x^5$ в выражении определителя
    \[
    \begin{vmatrix}
    2 & -3 & x & 4 & -5\\
    3 & x & -1 & -2 & 4\\
    1 & 3 & 1 & x & 1\\
    -3 & x^2 & -1 & 1 & x\\
    x & -2 & 4 & 5 & -2
    \end{vmatrix}.
    \]

    \item Найдите коэффициент при $x^4$ в выражении определителя
    \[
    \begin{vmatrix}
    1 & 3 & x & 2 & 2\\
    x & 2 & 1 & 4 & 5\\
    x & 1 & x & 5 & x\\
    3 & x & 1 & 2 & 3\\
    1 & 2 & 4 & x & 2
    \end{vmatrix}.
    \]

    \item Найдите наибольшее значение определителя матрицы $3 \times 3$,
     у которой все элементы равны 0 или 1.

\end{enumerate}
\heart
    
\end{document}
