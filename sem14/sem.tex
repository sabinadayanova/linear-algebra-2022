\documentclass[10pt, a4paper]{extarticle}

%% Язык
\usepackage{cmap} % Поиск в PDF
\usepackage{mathtext} % Кириллица в формулах
\usepackage[T2A]{fontenc} % Кодировка
\usepackage[utf8]{inputenc} % Кодировка
\usepackage[english,russian]{babel} % Локализация, переносы

\pagestyle{empty} \textwidth=19.0cm \oddsidemargin=-1.3cm
\textheight=26cm \topmargin=-3.0cm

%% Математика
\usepackage{amsmath, amsfonts, amssymb, amsthm, mathtools}
\usepackage{icomma}

% Операторы
\DeclareMathOperator{\tr}{tr}
\renewcommand{\le}{\leqslant}
\renewcommand{\ge}{\geqslant}
\renewcommand{\leq}{\leqslant}
\renewcommand{\geq}{\geqslant}

% Множества
\def \R{\mathbb{R}}
\def \N{\mathbb{N}}
\def \Z{\mathbb{Z}}
\newcommand{\rk}{\operatorname{\mathrm{rk}}}

\def \a{\alpha}
\def \be{\beta}

% Другое

\newcommand{\const}{\mathrm{const}}
\theoremstyle{definition}
\newtheorem{Task}{Задача}
%\newtheorem*{Taskn}{Задача {#1}}
\newtheorem*{Sol}{Решение}
\usepackage[dvipsnames]{xcolor}

\newcommand{\heart}{\begin{center}
\textcolor{RoyalPurple}{\ensuremath\heartsuit} 
\end{center}}
\usepackage{mathtools}
\usepackage{nicefrac}

%% Гиперссылки
\usepackage{xcolor}
\usepackage{hyperref}
\definecolor{linkcolor}{HTML}{8b00ff}
\hypersetup{colorlinks = true,
			linkcolor = linkcolor,
			urlcolor = linkcolor,
			citecolor = linkcolor}

%% Выравнивание
% \setlength{\parskip}{0.5em} % Расстояние между абзацами
\usepackage{geometry} % Поля
\geometry{
	a4paper,
	left=12mm,
	top=10mm,
	right=12mm}
% \setlength{\parindent}{0cm} % Отступ (красная строка)
% \linespread{1.1} % Интерлиньяж
\usepackage[many]{tcolorbox}  
\usepackage{enumitem}

%% Оформление

% Код
\newcommand{\code}[1]{{\tt #1}}

\newenvironment{amatrix}[2]{%
    \left(\begin{array}{@{}*{#1}{c}|*{#2}{c}@{}}
}{%
    \end{array}\right)
}

\begin{document}

\begin{center}
\small
\noindent\makebox[\textwidth]{Линейная алгебра и геометрия \hfill ФКН НИУ ВШЭ, 2022/2023 учебный год, 1-й курс ОП ПМИ, группа 2212}
\end{center}

\large

\begin{center}
\textbf{Семинар 14 (12.12.2022)}
\end{center}

\textbf{Краткое содержание}

Обсудили эквивалентное определение ранга матрицы как наибольшего порядка ненулевого минора, решили номер П613.

Следующий сюжет --- алгоритм нахождения ОСЛУ, множеством решений которой является линейная оболочка заданного набора векторов. Алгоритм следующий:
\begin{enumerate}
    \item берем эти заданные нам векторы $v_1, \dots, v_m$ и укладываем их по столбцам в матрицу B. Транспонируем матрицу B, получаем $B^T$.
    \item находим ФСР у ОСЛУ $B^Tx=0$. Обозначим полученные векторы как $a_1, \dots, a_q$.
    \item кладем полученные векторы по столбцам в матрицу A. Транспонируем ее, получаем $A^T$. Ответ: ОСЛУ $A^Tx=0$.
\end{enumerate}
Применили этот алгоритм к векторам $(1, 1, 0, 2)$, $(3, -3, 2, 0), (2, -1, 1, 1)$.

Дальше разобрали описание всех базисов конечномерного векторного пространства в терминах одного базиса.
Обсудили матрицы перехода и формулу для преобразования координат вектора при замене базиса.
Если $(e_1,\dots,e_n)$ и $(e'_1,\dots,e'_n)$ --- два базиса одного векторного пространства, то матрица перехода $C$ между ними определяется из соотношения
\begin{equation} \label{eqn1}
(e'_1,\dots,e'_n) = (e_1,\dots,e_n)C,
\end{equation}
то есть в $j$-м столбце матрицы $C$ стоят координаты вектора $e'_j$ в базисе $(e_1,\dots,e_n)$.
Если один и тот же вектор $v$ имеет координаты $(x_1,\dots, x_n)$ в базисе $(e_1,\dots, e_n)$ и координаты $(x'_1,\dots, x'_n)$ в 
базисе $(e'_1,\dots, e'_n)$, то эти наборы координат связаны друг с другом соотношением
\begin{equation} \label{eqn2}
\begin{pmatrix} x_1 \\ \vdots \\ x_n \end{pmatrix} =
C
\begin{pmatrix} x'_1 \\ \vdots \\ x_n \end{pmatrix}.
\end{equation}
(Обратите внимание, что в формулах (\ref{eqn1}) и (\ref{eqn2}) штрихи стоят с разных сторон от знака равенства!)
Таким образом, чтобы пересчитать координаты вектора из одного базиса в другой, нужно решить СЛУ.

\heart
\textbf{Домашнее задание к семинару 15. Дедлайн 9.01.2023}

Номера с пометкой П даны по задачнику Проскурякова, с пометкой К -- Кострикина.

\begin{enumerate}
    \item К35.16

    \item К34.10

    \item К34.11

    \item К34.12

    \item К34.13

\end{enumerate}
\heart
    
\end{document}
