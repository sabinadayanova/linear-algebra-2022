\documentclass[12pt, a4paper]{extarticle}

%% Язык
\usepackage{cmap} % Поиск в PDF
\usepackage{mathtext} % Кириллица в формулах
\usepackage[T2A]{fontenc} % Кодировка
\usepackage[utf8]{inputenc} % Кодировка
\usepackage[english,russian]{babel} % Локализация, переносы

%% Шрифты

% Serif
%\usepackage{euscript} % Шрифт Евклид
%\usepackage{mathrsfs} %  рифт для математики
\usepackage{libertinus}

% Sans-serif
%\renewcommand{\rmdefault}{cmss}
%\renewcommand{\ttdefault}{cmss}
%\usepackage{sfmath}

% Настройки для xelatex
%\usepackage{polyglossia} % Для выбора языка в xelatex
%\setmainlanguage{russian}
%\setotherlanguages{english}
% Ligatures=TeX is on by default
% https://tex.stackexchange.com/questions/323542/
%\setmainfont[Ligatures=TeX]{Cantarell}
%\newfontfamily{\cyrillicfonttt}{Times New Roman}
%\newfontfamily\cyrillicfont{Cantarell}[Script=Cyrillic]
%\setsansfont[Ligatures=TeX]{Cantarell}
%\newfontfamily\cyrillicfontsf{Cantarell}[Script=Cyrillic]
%\setmonofont{Courier New}
%\newfontfamily\cyrillicfonttt{Courier New}[Script=Cyrillic]

%% Математика
\usepackage{amsmath, amsfonts, amssymb, amsthm, mathtools}
\usepackage{icomma}

\newtheorem{theor}{Теорема}
\newtheorem{defn}{Определение}

% Операторы
\DeclareMathOperator*\plim{plim}
\DeclareMathOperator{\sgn}{sign}
\DeclareMathOperator{\sign}{sign}
\DeclareMathOperator*{\argmin}{arg\,min}
\DeclareMathOperator*{\argmax}{arg\,max}
\DeclareMathOperator*{\amn}{arg\,min}
\DeclareMathOperator*{\amx}{arg\,max}
\DeclareMathOperator{\cov}{Cov}
\DeclareMathOperator{\Var}{Var}
\DeclareMathOperator{\Cov}{Cov}
\DeclareMathOperator{\Corr}{Corr}
\DeclareMathOperator{\pCorr}{pCorr}
\DeclareMathOperator{\E}{\mathbb{E}}
\let\P\relax
\DeclareMathOperator{\P}{\mathbb{P}}
\DeclareMathOperator{\tr}{tr}
\renewcommand{\le}{\leqslant}
\renewcommand{\ge}{\geqslant}
\renewcommand{\leq}{\leqslant}
\renewcommand{\geq}{\geqslant}

% Распределения
\newcommand{\cN}{\mathcal{N}}
\newcommand{\cU}{\mathcal{U}}
\newcommand{\cBinom}{\mathcal{Binom}}
\newcommand{\cPois}{\mathcal{Pois}}
\newcommand{\cBeta}{\mathcal{Beta}}
\newcommand{\cGamma}{\mathcal{Gamma}}

% Множества
\def \R{\mathbb{R}}
\def \N{\mathbb{N}}
\def \Z{\mathbb{Z}}

\def \a{\alpha}
\def \be{\beta}

% Другое
\newcommand{\dx}[1]{\,\mathrm{d}#1} % Для интеграла: маленький отступ и прямая d
\newcommand{\ind}[1]{\mathbbm{1}_{\{#1\}}} % Индикатор события
\newcommand{\iid}{\mathrel{\stackrel{\rm i.\,i.\,d.}\sim}}
\newcommand{\const}{\mathrm{const}}
\theoremstyle{definition}
\newtheorem{Task}{Задача}
%\newtheorem*{Taskn}{Задача {#1}}
\newtheorem*{Sol}{Решение}
\usepackage[dvipsnames]{xcolor}
\newcommand{\heart}{\begin{center}
\textcolor{RoyalPurple}{\ensuremath\heartsuit} 
\end{center}}
\usepackage{mathtools}


%% Изображения
\usepackage{graphicx}
\usepackage{caption}
\usepackage{subcaption}
\usepackage{physics}
\usepackage{wrapfig} % Обтекание рисунков и таблиц текстом
\usepackage{tikz}

%% Таблицы
\usepackage{array, tabularx, tabulary, booktabs}
\usepackage{longtable}  % Длинные таблицы
\usepackage{multirow} % Слияние строк в таблице

%% Cписки
\usepackage{multicol}
\usepackage{enumitem}

%% Гиперссылки
\usepackage{xcolor}
\usepackage{hyperref}
\definecolor{linkcolor}{HTML}{8b00ff}
\hypersetup{colorlinks = true,
			linkcolor = linkcolor,
			urlcolor = linkcolor,
			citecolor = linkcolor}

%% Выравнивание
\setlength{\parskip}{0.5em} % Расстояние между абзацами
\usepackage{geometry} % Поля
\geometry{
	a4paper,
	left=20mm,
	top=20mm,
	right=20mm}
\setlength{\parindent}{0cm} % Отступ (красная строка)
\linespread{1.1} % Интерлиньяж
\usepackage[many]{tcolorbox}  

%% Оформление

\newtcolorbox{rulesbox}[1]{%
	tikznode boxed title,
	enhanced,
	arc=0mm,
	interior style={white},
	attach boxed title to top center= {yshift=-\tcboxedtitleheight/2},
	fonttitle=\bfseries,
	colbacktitle=white,coltitle=black,
	boxed title style={size=normal,colframe=white,boxrule=0pt},
	title={#1}}

% Красивый серый фон
\usepackage{framed} 
\definecolor{shadecolor}{gray}{0.9}

% Код
\newcommand{\code}[1]{{\tt #1}}

% Колонтитулы
\usepackage{fancyhdr}
\pagestyle{fancy}
\fancyhf{}
\fancyhead[L]{}
\fancyhead[R]{\thepage}

% Разделы и подразделы
\usepackage[sf, sl, outermarks]{titlesec}
\titleformat{\section}{\Large\bfseries\sffamily}{\thesection}{0.5em}{}
\titleformat{\subsection}{\large\sffamily}{\thesubsection}{0.5em}{}

% Содержание
%\usepackage{tocloft}
%\renewcommand{\cftsecfont}{\hspace{4.5em}\normalfont}
%\renewcommand{\cftsubsecfont}{\hspace{5em}\normalfont}
%\renewcommand{\cftsecpagefont}{\normalfont\hfill}
%\renewcommand{\cfttoctitlefont}{\large\normalfont\hfill}
%\renewcommand{\cftaftertoctitle}{\hfill}
%\renewcommand{\cftsecleader}{\cftdotfill{\cftdotsep}}
%\renewcommand{\cftsecafterpnum}{\hspace*{5.5em}\hfill}
%\renewcommand{\cftsubsecafterpnum}{\hspace*{5.5em}\hfill}
%\renewcommand{\cftsecaftersnum}{.}
%\renewcommand{\cftsubsecaftersnum}{.}

%% Комментарии
\usepackage{comment}

%% To-do
\usepackage{todonotes}

%% Литература
\usepackage[backend = biber,
			bibencoding = utf8, 
			sorting = nty, 
			maxcitenames = 4,
			style = numeric-verb]{biblatex}
\addbibresource{lit.bib}
\usepackage{csquotes}

\newenvironment{amatrix}[2]{%
  \left(\begin{array}{@{}*{#1}{c}|{}*{#2}{c}}
}{%
  \end{array}\right)
}

%% Заголовок
\title{{\normalsize Линейная алгебра и геометрия 2022} \\ \vspace{0.5em}  Семинар 1}
\author{ \rule{15cm}{0.4pt}}

\date{}
\begin{document}
	\maketitle
	\subsection*{Краткое содержание}
    Обсудили сложение матриц. Доказали коммутативность этой операции. Упомянули транспонирование матриц.
    
    Обсудили умножение матриц. Решили простой пример на понимание операций -- вычислили значение $AB^T + 3C$, где
    \[
    A = \begin{pmatrix}
        2 & 1 & 0 \\
        0 & 2 & -1 \\
    \end{pmatrix}, \;
    B = \begin{pmatrix}
        1 & 0 & 2 \\
        -3 & 2 & 0 \\
    \end{pmatrix}, \;
    C = \begin{pmatrix}
        0 & 1 \\
        1 & 2 \\
    \end{pmatrix}
    \]
    Перемножили матрицы 
    $\begin{pmatrix}
        0 & 1 \\
        0 & 0 \\
    \end{pmatrix}$ 
    и
    $\begin{pmatrix}
        a & b \\
        c & d \\
    \end{pmatrix}$
    в разном порядке и убедились, что умножение матриц не коммутативно.

    Определили натуральную степень матрицы, а также нулевую степень: $A^0 = E$. Методом математической индукции доказали, что 
    $\begin{pmatrix}
        1 & 1 \\
        0 & 1 \\
    \end{pmatrix}^n = 
    \begin{pmatrix}
        1 & n \\
        0 & 1 \\
    \end{pmatrix}.$
    
    Обсудили быстрое возведение матрицы в степень. Поняли, что оно работает абсолютно также, как и с действительными числами.
    Игрушечный пример:

    $A^4 = A \cdot A \cdot A \cdot A \to$  3 операции (3 раза умножили).

    $A^4 = A \cdot A \cdot A \cdot A = A^2 \cdot A^2 \to$ 2 операции (один раз возвели в степень, во второй раз перемножили результат друг на друга).

    $A^9 = A^8 \cdot A = A^4 \cdot A^4 \cdot A \to$ 4 операции
    (2 операции на возведение $A$ в 4ую степень, 3-я на перемножение $A^4$ на $A^4$, 4-ая на умножение на последнюю $A$)

    Суть: переиспользуем уже подсчитанные результаты, перемножаем уже посчитанные степени матрицы.
    Обычно это делается по основанию двойки.

    Посмотрели на результат умножения следующих матриц и получили:
    \[
    \begin{pmatrix}
        \lambda_1 & 0 & \dots & 0 \\
        0 & \lambda_2 & \dots & 0 \\
        \dots & \dots & \dots & \dots \\
        0 & 0 & \dots & \lambda_n
    \end{pmatrix}
    \cdot 
    \begin{pmatrix}
        \mu_1 & 0 & \dots & 0 \\
        0 & \mu_2 & \dots & 0 \\
        \dots & \dots & \dots & \dots \\
        0 & 0 & \dots & \mu_n
    \end{pmatrix}
    =
    \begin{pmatrix}
        \lambda_1 \mu_1 & 0 & \dots & 0 \\
        0 & \lambda_2 \mu_2 & \dots & 0 \\
        \dots & \dots & \dots & \dots \\
        0 & 0 & \dots & \lambda_n \mu_n
    \end{pmatrix}
    \]
    Сделали вывод, что умножение диагональных матриц дает нам тоже диагональную матрицу.

    Посмотрели на результат умножения следующих матриц и получили:
    \[
    \begin{pmatrix}
        a_{11} & a_{12} & \dots & a_{1n} \\
        a_{21} & a_{22} & \dots & a_{2n} \\
        \dots & \dots & \dots & \dots \\
        a_{m1} & a_{m2} & \dots & a_{mn} \\
    \end{pmatrix}    
    \cdot
    \begin{pmatrix}
        \lambda_1 & 0 & \dots & 0 \\
        0 & \lambda_2 & \dots & 0 \\
        \dots & \dots & \dots & \dots \\
        0 & 0 & \dots & \lambda_n
    \end{pmatrix}
    =
    \begin{pmatrix}
        \lambda_1 a_{11} & \lambda_2 a_{12} & \dots & \lambda_n a_{1n} \\
        \lambda_1 a_{21} & \lambda_2 a_{22} & \dots & \lambda_n a_{2n} \\
        \dots & \dots & \dots & \dots \\
        \lambda_1 a_{m1} &\lambda_2  a_{m2} & \dots & \lambda_n a_{mn} \\
    \end{pmatrix} 
    \]
    То есть, используя нотацию $A^i$ -- i-ый столбец матрицы $A$, получаем:
    \[
    \begin{pmatrix}
        A^1 & A^2 & \dots & A^n\\
    \end{pmatrix}  
    \cdot
    \begin{pmatrix}
        \lambda_1 & 0 & \dots & 0 \\
        0 & \lambda_2 & \dots & 0 \\
        \dots & \dots & \dots & \dots \\
        0 & 0 & \dots & \lambda_n
    \end{pmatrix}
    =
    \begin{pmatrix}
        \lambda_1 A^1 & \lambda_2 A^2 & \dots & \lambda_n A^n\\
    \end{pmatrix} 
    \]
    \heart

    \subsection*{Домашнее задание к семинару 2 (13.09.2022)}
    Номера с пометкой П даны по задачнику Проскурякова, с пометкой К -- Кострикина.
    \begin{enumerate}
        \item Пусть $A, B$ -- произвольные матрицы одинакового размера и $\lambda, \mu \in \R$.
        Докажите (расписав подробно), что $\lambda (A+B) = \lambda A + \lambda B$ и $\lambda (\mu A) = (\lambda \mu) A$.
        \item K17.1(г), К17.2(б)
        \item Найдите все 2х2-матрицы $B$, коммутирующие с матрицей
        $A=\begin{pmatrix}
            2 & -1 \\
            1 & 0 \\
        \end{pmatrix}$, то есть удовлетворяющие условию $AB = BA$.
        \item Найти, чему равно выражение (расписать подробно):
        \[
        \begin{pmatrix}
            \lambda_1 & 0 & \dots & 0 \\
            0 & \lambda_2 & \dots & 0 \\
            \dots & \dots & \dots & \dots \\
            0 & 0 & \dots & \lambda_n
        \end{pmatrix}  
        \cdot 
        \begin{pmatrix}
            A_1 \\
            A_2 \\
            \dots \\
            A_n
        \end{pmatrix},
        \] 
        где $A_i$ -- i-ая строчка матрицы $A \in Mat_{n\times m}(\R)$.
        Другими словами, правая матрица в выражении -- это матрица $A$, записанная в терминах строчек. Найденный ответ надо записать тоже в таких же терминах.
        \item П802
        \item П803
        \item П805
        \item К17.7
         
        
    \end{enumerate}
    
\end{document}