\documentclass[10pt, a4paper]{extarticle}

%% Язык
\usepackage{cmap} % Поиск в PDF
\usepackage{mathtext} % Кириллица в формулах
\usepackage[T2A]{fontenc} % Кодировка
\usepackage[utf8]{inputenc} % Кодировка
\usepackage[english,russian]{babel} % Локализация, переносы

\pagestyle{empty} \textwidth=19.0cm \oddsidemargin=-1.3cm
\textheight=26cm \topmargin=-3.0cm

%% Математика
\usepackage{amsmath, amsfonts, amssymb, amsthm, mathtools}
\usepackage{icomma}

% Операторы
\DeclareMathOperator{\tr}{tr}
\renewcommand{\le}{\leqslant}
\renewcommand{\ge}{\geqslant}
\renewcommand{\leq}{\leqslant}
\renewcommand{\geq}{\geqslant}

% Множества
\def \R{\mathbb{R}}
\def \N{\mathbb{N}}
\def \Z{\mathbb{Z}}

\def \a{\alpha}
\def \be{\beta}

% Другое

\newcommand{\const}{\mathrm{const}}
\theoremstyle{definition}
\newtheorem{Task}{Задача}
%\newtheorem*{Taskn}{Задача {#1}}
\newtheorem*{Sol}{Решение}
\usepackage[dvipsnames]{xcolor}

\newcommand{\heart}{\begin{center}
\textcolor{RoyalPurple}{\ensuremath\heartsuit} 
\end{center}}
\usepackage{mathtools}
\usepackage{nicefrac}

%% Гиперссылки
\usepackage{xcolor}
\usepackage{hyperref}
\definecolor{linkcolor}{HTML}{8b00ff}
\hypersetup{colorlinks = true,
			linkcolor = linkcolor,
			urlcolor = linkcolor,
			citecolor = linkcolor}

%% Выравнивание
% \setlength{\parskip}{0.5em} % Расстояние между абзацами
\usepackage{geometry} % Поля
\geometry{
	a4paper,
	left=12mm,
	top=10mm,
	right=12mm}
% \setlength{\parindent}{0cm} % Отступ (красная строка)
% \linespread{1.1} % Интерлиньяж
\usepackage[many]{tcolorbox}  
\usepackage{enumitem}

%% Оформление

% Код
\newcommand{\code}[1]{{\tt #1}}

\newenvironment{amatrix}[2]{%
    \left(\begin{array}{@{}*{#1}{c}|*{#2}{c}@{}}
}{%
    \end{array}\right)
}

\begin{document}

\begin{center}
\small
\noindent\makebox[\textwidth]{Линейная алгебра и геометрия \hfill ФКН НИУ ВШЭ, 2022/2023 учебный год, 1-й курс ОП ПМИ, группа 2212}
\end{center}

\large

\begin{center}
\textbf{Семинар 12 (28.11.2022)}
\end{center}

\textbf{Краткое содержание}

\textbf{Первый сюжет} --- как находить (какой-то один) базис в подпространстве пространства $F^n$, заданном как линейная оболочка 
конечного набора векторов. \textbf{Первое соображение}: при элементарных преобразованиях данной системы векторов (прибавить к 
одному вектору другой, умноженный на скаляр; поменять местами два вектора; умножить один вектор на ненулевой скаляр) её
линейная оболочка сохраняется. \textbf{Второе соображение}: набор векторов вида
\[
\begin{pmatrix}
\diamond \\ * \\ * \\ \vdots \\ * \\ *
\end{pmatrix},
\begin{pmatrix}
0 \\ \diamond \\ * \\ \vdots \\ * \\ *
\end{pmatrix},
\begin{pmatrix}
0 \\ 0 \\ \diamond \\ \vdots \\ * \\ *
\end{pmatrix},
\dots,
\begin{pmatrix}
0 \\ 0 \\ 0 \\ \vdots \\ \diamond \\ *
\end{pmatrix},
\begin{pmatrix}
0 \\ 0 \\ 0 \\ \vdots \\ 0 \\ \diamond
\end{pmatrix},
\]
где $\diamond \ne 0$ и $*$ -- произвольные элементы, всегда линейно независим (и, в частности, образует базис в $F^n$), а значит, 
всякая подсистема набора такого вида тоже линейно независима. Эти два соображения дают алгоритм нахождения искомого базиса:
\begin{enumerate}[noitemsep]
    \item записать векторы в матрицу по столбцам
    \item элементарными преобразованиями столбцов привести её к транспонированно-ступенчатому виду
    \item записать в базис все ненулевые столбцы полученного вида
\end{enumerate} 
Можно заметить, что тот же самый алгоритм можно делать и по строчкам, и ничего концептуально не поменяется -- 
записать векторы в строки матрицы, элементарными преобразованиями строк привести её к ступенчатому виду,
после чего записать в базис все ненулевые строки полученного вида. 

Посмотрели, как данный алгоритм работает на наборе векторов $(2,-3,1), (3,-1,-2), (1,-4,3)$ в $\R^3$.

\textbf{Второй сюжет} --- даны векторы $v_1,\dots, v_m \in F^n$.
Требуется выбрать среди этих векторов базис их линейной оболочки $\langle v_1,\dots, v_m \rangle$ и выразить через этот базис 
все остальные векторы данной системы. \textbf{Ключевое соображение}: при элементарных преобразованиях \textbf{строк} матрицы сохраняются 
все линейные зависимости между её \textbf{столбцами}. Алгоритм: 
\begin{enumerate}[noitemsep]
    \item записываем векторы в столбцы матрицы
    \item элементарными преобразованиями строк приводим эту матрицу к ступенчатому виду
    \item пусть $i_1,\dots, i_s$ -- номера ведущих элементов строк этой получившейся матрицы; тогда векторы $v_{i_1},\ldots, v_{i_s}$ 
    (то есть векторы с теми же номерами в исходной матрице) образуют искомый базис в $\langle v_1,\ldots, v_m \rangle$.
\end{enumerate}
Чтобы выразить остальные векторы через найденный базис, нужно довести матрицу до улучшенного ступенчатого вида, из него вся нужная 
информация извлекается сразу. Например, если улучшенный ступенчатый вид есть 
$\begin{pmatrix} 
    1 & 2 & 0 & -1 \\ 
    0 & 0 & 1 & 2 \\ 
    0 & 0 & 0 & 0
\end{pmatrix}$,
то получаем, что базис будут образовывать векторы $v_1$ и $v_3$, 
а остальные будут выражаться через них так: $v_2 = 2v_1$, $v_4 = -v_1 + 2v_3$.

\textbf{Последний сюжет} --- понятие фундаментальной системы решений (ФСР) однородной системы линейных уравнений $Ax=0$ и метод построения
одной конкретной ФСР. Алгоритм:
\begin{enumerate}[noitemsep]
    \item приводим матрицу $A$ элементарными преобразованиями строк к улучшенному ступенчатому виду
    \item выражаем главные переменные через свободные (как делали раньше при решении ОСЛУ)
    \item число векторов в базисе (или ФСР) равно числу свободных переменных. i-ый вектор базиса получается так: в выражениях главных через 
    свободных в i-ую свободную переменную подставляем ненулевое значение (например, 1), а в остальные свободные -- 0; считаем получившиеся значения 
    главных переменных. записываем эти значения (и главных, и свободных) в вектор в соответстующие координаты и получаем i-ый вектор базиса.
\end{enumerate}

Разобрали, как этот метод работает для ОСЛУ с матрицами
$\begin{pmatrix} 
    1 & 2 & 0 & -1 \\ 
    0 & 0 & 1 & -2 
\end{pmatrix}$
и 
$\begin{pmatrix} 
    3 & 1 & 0 & -2 \\ 
    0 & 0 & 2 & -1 
\end{pmatrix}$.

При помощи ФСР нашли базис в подпространстве $\{ f \mid f(1) = 0, f'(1) = 0 \}$ пространства $\R[x]_{\leq 3}$ многочленов степени 
не выше 3 с действительными коэффициентами.

\heart
\textbf{Домашнее задание к семинару 13. Дедлайн 5.12.2022}

Номера с пометкой П даны по задачнику Проскурякова, с пометкой К -- Кострикина.

\begin{enumerate}
    \item П1310 (применить первый алгоритм с семинара, находящий какой-то базис)
    \item П1311
    \item Среди векторов $a_1, \dots, a_5$ из номера П1310 выберите базис их линейной оболочки и выразите через 
    него все остальные векторы данной системы.
    
    \item В пространстве $\R^5$ даны векторы $v_1 = (-2, 1, -3, 2, 3)$, $v_2 = (-2, 3, -5, 7, 4)$, 
    $v_3 = (2, 1, 1, 3, -2)$, $v_4 = (9, -2, 4, -3, -8)$.
    
    (а) Выделите среди этих векторов базис их линейной оболочки $\langle v_1, v_2, v_3, v_4 \rangle$.
    
    (б) Перечислите все подсистемы (подмножества) системы $\lbrace v_1, v_2, v_3, v_4 \rbrace$, являющиеся базисами в $\langle v_1, v_2, v_3, v_4 \rangle$.
    
    \item
    П725, П727
    
    \item
    П729, П730
    
    \item
    (1) Докажите, что для всякой матрицы $A \in \mathrm M_n(\R)$ множество векторов $x \in \R^n$ со свойством $Ax = 2x$ является подпространством в~$\R^n$.
    
    (2) Найдите базис и размерность этого подпространства, если $n = 4$ и
    \[
    A =
    \begin{pmatrix}
    2 & 1 & 0 & 2 \\
    0 & -1 & 0 & -6 \\
    0 & 3 & 2 & 6 \\
    0 & 1 & 0 & 4 \\
    \end{pmatrix}.
    \]
    
    \item
    (1) Пусть $A \in \mathrm M_n(\R)$.
    Докажите, что множество всех матриц $X \in \mathrm M_n(\R)$, коммутирующих с~$A$ (то есть удовлетворяющих условию $AX = XA$), является подпространством в $\mathrm M_n(\R)$.
    
    (2) Найдите базис и размерность этого подпространства, если $n = 2$ и $A = \begin{pmatrix} 1 & 2 \\ 3 & 4 \end{pmatrix}$.
    
    \item
    (1) Пусть $Y \in \mathrm M_n(\R)$. Докажите, что множество всех матриц $X \in \mathrm{M}_n(\R)$, удовлетворяющих условию $\tr (YX) = 0$, является подпространством в пространстве $\mathrm{M}_n(\R)$.
    
    (2) Найдите базис и размерность этого подпространства, если $n = 2$ и $Y = \begin{pmatrix} 2 & 5 \\ 3 & 4 \end{pmatrix}$.
    
    \end{enumerate}
\heart
    
\end{document}
