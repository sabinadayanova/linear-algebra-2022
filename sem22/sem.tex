\documentclass[10pt, a4paper]{extarticle}

%% Язык
\usepackage{cmap} % Поиск в PDF
\usepackage{mathtext} % Кириллица в формулах
\usepackage[T2A]{fontenc} % Кодировка
\usepackage[utf8]{inputenc} % Кодировка
\usepackage[english,russian]{babel} % Локализация, переносы
\usepackage{bbold} % для ажурных буковок

\pagestyle{empty} \textwidth=19.0cm \oddsidemargin=-1.3cm
\textheight=26cm \topmargin=-3.0cm

%% Математика
\usepackage{amsmath, amsfonts, amssymb, amsthm, mathtools}
\usepackage{icomma}


% Операторы
\DeclareMathOperator{\tr}{tr}
\renewcommand{\le}{\leqslant}
\renewcommand{\ge}{\geqslant}
\renewcommand{\leq}{\leqslant}
\renewcommand{\geq}{\geqslant}

% Множества
\def \R{\mathbb{R}}
\def \N{\mathbb{N}}
\def \Z{\mathbb{Z}}
\newcommand{\rk}{\operatorname{\mathrm{rk}}}
\newcommand{\diag}{\operatorname{diag}}
\newcommand{\Ker}{\mathop{\mathrm{Ker}}}
\renewcommand{\Im}{\mathop{\mathrm{Im}}}

\def \a{\alpha}
\def \be{\beta}

% Другое

\newcommand{\const}{\mathrm{const}}
\theoremstyle{definition}
\newtheorem*{proposal}{Предложение}
\newtheorem{Task}{Задача}
%\newtheorem*{Taskn}{Задача {#1}}
\newtheorem*{Sol}{Решение}
\usepackage[dvipsnames]{xcolor}

\newcommand{\heart}{\begin{center}
\textcolor{RoyalPurple}{\ensuremath\heartsuit} 
\end{center}}
\usepackage{mathtools}
\usepackage{nicefrac}

%% Гиперссылки
\usepackage{xcolor}
\usepackage{hyperref}
\definecolor{linkcolor}{HTML}{8b00ff}
\hypersetup{colorlinks = true,
			linkcolor = linkcolor,
			urlcolor = linkcolor,
			citecolor = linkcolor}

%% Выравнивание
% \setlength{\parskip}{0.5em} % Расстояние между абзацами
\usepackage{geometry} % Поля
\geometry{
	a4paper,
	left=12mm,
	top=10mm,
    bottom=20mm,
	right=12mm}
% \setlength{\parindent}{0cm} % Отступ (красная строка)
% \linespread{1.1} % Интерлиньяж
\usepackage[many]{tcolorbox}  
\usepackage{enumitem}

%% Оформление

% Код
\newcommand{\code}[1]{{\tt #1}}

\newenvironment{amatrix}[2]{%
    \left(\begin{array}{@{}*{#1}{c}|*{#2}{c}@{}}
}{%
    \end{array}\right)
}

\begin{document}

\begin{center}
\small
\noindent\makebox[\textwidth]{Линейная алгебра и геометрия \hfill ФКН НИУ ВШЭ, 2022/2023 учебный год, 1-й курс ОП ПМИ, группа 2212}
\end{center}

\large

\begin{center}
\textbf{Семинар 22 (28.02.2023)}
\end{center}

\textbf{Краткое содержание}

Начали с обсуждения понятия матрицы Грама системы векторов евклидова пространства и её свойств.

Следующий сюжет --- определение квадратных матриц произвольного порядка, являющихся матрицами Грама каких-то систем векторов евклидова пространства.

Пусть $v_1,\dots,v_k$ --- некоторая система векторов евклидова пространства $\mathbb E$, и пусть $G$ --- её матрица Грама.
Предположим, что векторы $v_1',\dots,v_m' \in \mathbb E$ выражаются через $v_1, \dots, v_k$; тогда можно записать $(v_1',\dots,v_m') = (v_1,\dots, v_k)C$ для некоторой матрицы $C \in \mathrm{Mat}_{k\times m}(\R)$.
Показали, что матрица Грама системы $v_1',\dots, v_m'$ равна $C^TGC$.
Применили полученный результат для доказательства следующей теоремы.

\textbf{Теорема.} Квадратная матрица $A$ является матрицей Грама некоторой системы векторов евклидова пространства тогда и только тогда, когда $A$ симметрична и неотрицательно определённа (то есть таковой является квадратичная форма с этой матрицей).

\textit{Доказательство} (содержит алгоритм построения требуемой системы векторов).
Если $A \in \mathrm{M}_k(\R)$ --- матрица Грама некоторой системы векторов в $\R^n$, то $A$ симметрична.
Мы уже знаем, что квадратичная форма с матрицей $A$ путём замены координат приводится к диагональному виду.
Это означает, что существует невырожденная матрица $C \in \mathrm{M}_k(\R)$, для которой $C^TAC = D$, где $D = \diag(d_1,\dots,d_k)$.
Как уже отмечалось выше, в данной ситуации $D$ тоже должна являться матрицей Грама некоторой системы векторов в $\R^n$, поэтому все её диагональные элементы должны быть неотрицательны, откуда и следует неотрицательная определённость соответствующей квадратичной формы.

Обратно, пусть теперь $d_i \geqslant 0$ для всех $i = 1,\dots,k$.
Тогда легко предъявить систему векторов $(f_1,\dots, f_k)$ с матрицей Грама $D$.
Например, если $e_1,\dots, e_n$ --- стандартный базис в $\R^n$, то можно взять $f_i = \sqrt{d_i}e_i$, $i=1,\dots,k$.
В соответствии с разобранным выше у системы векторов $(f_1,\dots,f_k)C^{-1}$ матрицей Грама будет $(C^{-1})^TDC^{-1} = A$, что и требовалось.

Применили данный алгоритм к матрице $\begin{pmatrix} 1 & 1 \\ 1 & 3 \end{pmatrix}$.

Дальше поговорили про ортогональное дополнение подмножества евклидова пространства, обсудили его основные свойства.
Ортогональное дополнение к системе векторов $v_1,\dots, v_k \in \R^n$ (а также к их линейной оболочке) совпадает с множеством решений ОСЛУ $\lbrace (v_i,x) = 0 \mid i = 1,\dots,k\rbrace$, поэтому базис ортогонального дополнения есть просто ФСР для указанной ОСЛУ.

Следующий сюжет --- ортогональные и ортонормированные системы векторов, ортогональные и ортонормированные базисы.
Обсудили формулу для координат вектора в ортогональном (ортонормированном) базисе.
Разобрали метод ортогонализации Грама--Шмидта, который позволяет построить ортогональный базис подпространства, стартуя с какого-то базиса.
А именно, для исходной линейно независимой системы векторов $e_1,\dots,e_k$ формулы $f_i = e_i - \sum \limits_{j=1}^{i-1}\frac{(e_i,f_j)}{(f_j,f_j)}f_j$ определяют ортогональный базис $f_1,\dots,f_k$ в подпространстве $\langle e_1,\dots, e_k \rangle$.

Обсудили, как с помощью ортогонализации Грама--Шмидта дополнять ортогональную систему векторов до ортогонального базиса.
Есть два способа:

1) сначала найти базис в ортогональном дополнении, а затем ортогонализовать его;

2) дополнить исходный базис до какого-нибудь базиса и затем ортогонализовать его.

\noindent
Применили оба этих способа для решения номера П1359.


\heart

\textbf{Домашнее задание к семинару 23. Дедлайн 7.03.2023}

Номера с пометкой П даны по задачнику Проскурякова, с пометкой К -- Кострикина.

В обоих задачниках координаты векторов из $\R^n$ всегда записываются в строчку через запятую, однако нужно помнить, что мы всегда записываем эти координаты в столбец.


\begin{enumerate}


	\item
Существует ли система векторов в $\R^3$ с матрицей Грама $\begin{pmatrix} 2 & 1 & -3 \\ 1 & 6 & 4 \\ -3 & 4 & 11 \end{pmatrix}$?
Если существует, то укажите её.

\item
Тот же вопрос для матрицы $\begin{pmatrix} 2 & 1 & 1 \\ 1 & 2 & -1 \\ 1 & -1 & 2 \end{pmatrix}$.

\item
Тот же вопрос для матрицы $\begin{pmatrix} 4 & 3 & 0 \\ 3 & 3 & 2 \\ 0 & 2 & 2 \end{pmatrix}$.

\item
П1366

\item
П1367

\item
Найдите базис ортогонального дополнения в $\R^3$ к множеству решений уравнения $x_1 + 2x_2 +\nobreak 3x_3 {=} \nobreak 0$.

\item
П1357, П1360

\item
П1361

\item
Рассмотрим евклидово пространство $\R[x]_{\leqslant 4}$ со скалярным произведением $(f,g) = \int\limits_{-1}^1 f(t)g(t)\,dt$.
При помощи метода ортогонализации постройте ортогональный базис в подпространстве $\langle 1,x,x^2,x^3 \rangle$.



\end{enumerate}
\heart


\end{document}
