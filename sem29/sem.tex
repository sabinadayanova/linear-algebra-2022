\documentclass[10pt, a4paper]{extarticle}

%% Язык
\usepackage{cmap} % Поиск в PDF
\usepackage{mathtext} % Кириллица в формулах
\usepackage[T2A]{fontenc} % Кодировка
\usepackage[utf8]{inputenc} % Кодировка
\usepackage[english,russian]{babel} % Локализация, переносы
\usepackage{bbold} % для ажурных буковок

\pagestyle{empty} \textwidth=19.0cm \oddsidemargin=-1.3cm
\textheight=26cm \topmargin=-3.0cm

%% Математика
\usepackage{amsmath, amsfonts, amssymb, amsthm, mathtools}
\usepackage{icomma}
\usepackage{algpseudocode}


% Операторы
\DeclareMathOperator{\tr}{tr}
\renewcommand{\le}{\leqslant}
\renewcommand{\ge}{\geqslant}
\renewcommand{\leq}{\leqslant}
\renewcommand{\geq}{\geqslant}

% Множества
\def \R{\mathbb{R}}
\def \N{\mathbb{N}}
\def \Z{\mathbb{Z}}
\def \CC{\mathbb{C}}
\newcommand{\rk}{\operatorname{\mathrm{rk}}}
\newcommand{\diag}{\operatorname{diag}}
\newcommand{\pr}{\mathop{\mathrm{pr}}}
\newcommand{\ort}{\mathop{\mathrm{ort}}}
\newcommand{\Ker}{\mathop{\mathrm{Ker}}}
\renewcommand{\Im}{\mathop{\mathrm{Im}}}
\newcommand{\Vol}{\mathop{\mathrm{Vol}}}

\def \a{\alpha}
\def \be{\beta}

% Другое

\newcommand{\const}{\mathrm{const}}
\theoremstyle{definition}
\newtheorem*{proposal}{Предложение}
\newtheorem{Task}{Задача}
%\newtheorem*{Taskn}{Задача {#1}}
\newtheorem*{Sol}{Решение}
\usepackage[dvipsnames]{xcolor}

\newcommand{\heart}{\begin{center}
\textcolor{RoyalPurple}{\ensuremath\heartsuit} 
\end{center}}
\usepackage{mathtools}
\usepackage{nicefrac}

%% Гиперссылки
\usepackage{xcolor}
\usepackage{hyperref}
\definecolor{linkcolor}{HTML}{8b00ff}
\hypersetup{colorlinks = true,
			linkcolor = linkcolor,
			urlcolor = linkcolor,
			citecolor = linkcolor}

%% Выравнивание
% \setlength{\parskip}{0.5em} % Расстояние между абзацами
\usepackage{geometry} % Поля
\geometry{
	a4paper,
	left=12mm,
	top=10mm,
    bottom=20mm,
	right=12mm}
% \setlength{\parindent}{0cm} % Отступ (красная строка)
% \linespread{1.1} % Интерлиньяж
\usepackage[many]{tcolorbox}  
\usepackage{enumitem}

%% Оформление

% Код
\newcommand{\code}[1]{{\tt #1}}

\newenvironment{amatrix}[2]{%
    \left(\begin{array}{@{}*{#1}{c}|*{#2}{c}@{}}
}{%
    \end{array}\right)
}

\begin{document}

\begin{center}
\small
\noindent\makebox[\textwidth]{Линейная алгебра и геометрия \hfill ФКН НИУ ВШЭ, 2022/2023 учебный год, 1-й курс ОП ПМИ, группа 2212}
\end{center}

\large

\begin{center}
\textbf{Семинар 29 (10.05.2023)}
\end{center}

\textbf{Краткое содержание}

Начали семинар с теоремы о приведении квадратичной формы к главным осям:

Для всякой квадратичной формы $Q$ в евклидовом пространстве существует ортонормированный базис, в котором $Q$ имеет канонический вид
 $Q(x) = \lambda_1x_1^2 + \lambda_2x_2^2 + \ldots + \lambda_n x_n^2$; причём числа $\lambda_1, \lambda_2, \ldots, \lambda_n$ определены 
 однозначно с точностью до перестановки (они являются собственными значениями самосопряжённого линейного оператора, который в ортонормированном 
 базисе имеет такую же матрицу, как и данная квадратичная форма).

Проговорили, что задача приведения квадратичной формы к главным осям эквивалентна задаче диагонализации (в ортонормированном базисе) самосопряжённого оператора и потому решается так же (а это мы разбирали на прошлом семинаре).
Подробно разобрали номер П1251.

Новая тема: ортогональные операторы.
Сформулировали определение и упомянули ещё 5 эквивалентных условий. Вывели, что среди действительных собственных значений ортогонального оператора могут быть только $1$ или $-1$.
Обсудили, какие бывают линейные операторы, которые одновременно являются ортогональными и самосопряжёнными.

Дальше проговорили теорему о каноническом виде ортогонального оператора.
Важный частный случай этой теоремы --- описание всех ортогональных операторов в трёхмерном евклидовом пространстве: всякий такой оператор --- это либо поворот вокруг некоторой прямой, либо  <<зеркальный поворот>> вокруг некоторой прямой.
Обсудили алгоритм нахождения ортонормированного базиса, в котором матрица ортогонального оператора $\varphi$ в $\R^3$ имеет канонический вид.
Если $A$ --- матрица данного линейного оператора в исходном ортонормированном базисе, то поступать можно так:

(I) Если $A = A^T$, то оператор самосопряжён, и к нему можно применить известный алгоритм для самосопряжённого оператора. (Единственный нюанс --- тут не надо считать характеристический многочлен, поскольку его корни мы и так знаем: это $1$ и $-1$.)

(II) Если $A \ne A^T$, то оператор не самосопряжён, поэтому в каноническом виде один из блоков обязательно будет матрицей $\begin{pmatrix} \cos \alpha & -\sin \alpha \\ \sin \alpha & \cos \alpha \end{pmatrix}$ поворота на угол $\alpha$, не кратный $\pi k$. Вторым из блоков будет $\pm 1$, поэтому одно из чисел $1$ или $-1$ будет собственным значением для $\varphi$. Дальше делаем следующее.

1) Рассматривая матрицу $A - \lambda E$ при $\lambda = 1, -1$, находим то значение $\lambda = \lambda_0 \in \lbrace \pm 1 \rbrace$, которое будет собственным для $\varphi$, а также соответствующий собственный вектор $e_3$ единичной длины.

2) Выбираем любой ортонормированный базис $(e_1,e_2)$ в $e_3^\perp$, тогда $(e_1,e_2,e_3)$ будет ортонормированным базисом, в котором $\varphi$ имеет канонический вид. Осталось только найти этот вид.

3) Если $\alpha$ --- угол поворота для $\varphi$, то $\varphi(e_1) = \cos \alpha \cdot e_1 + \sin \alpha \cdot e_2$ и тогда $\cos \alpha$ и $\sin \alpha$ находятся по формулам $\cos \alpha = (\varphi(e_1), e_1)$ и $\sin \alpha = (\varphi(e_1), e_2)$.


Полностью осуществили данный алгоритм для линейного оператора из номера П1574.

\heart

\textbf{Домашнее задание к семинару 30. Дедлайн 16.05.2023}

Номера с пометкой П даны по задачнику Проскурякова, с пометкой К -- Кострикина, с пометкой КК -- Ким-Крицкова.

\begin{enumerate}

	
	\item
	П1243
	
	В следующих трёх заданиях явно выписывайте искомую ортогональную замену координат (выражение старых координат через новые).
	
	\item
	П1248
	
	\item
	П1254
	
	\item
	П1259
	
	\item
	П1564
	
	\item
	П1571
	
	\item
	П1572
	
	\item
	К46.6(к)
	
	\item
	Ортогональный линейный оператор $\varphi \colon \R^3 \to \R^3$ имеет в стандартном базисе матрицу
	\[
	\frac13 \begin{pmatrix} -2 & -2 & 1 \\ -2 & 1 & -2 \\ -1 & 2 & 2 \end{pmatrix}.
	\]
	Найдите ортонормированный базис, в котором матрица оператора $\varphi$ имеет каноничесакий вид, и выпишите эту матрицу. Укажите ось и угол поворота, определяемого оператором $\varphi$.
	
	




\end{enumerate}
\heart


\end{document}
