\documentclass[10pt, a4paper]{extarticle}

%% Язык
\usepackage{cmap} % Поиск в PDF
\usepackage{mathtext} % Кириллица в формулах
\usepackage[T2A]{fontenc} % Кодировка
\usepackage[utf8]{inputenc} % Кодировка
\usepackage[english,russian]{babel} % Локализация, переносы

\pagestyle{empty} \textwidth=19.0cm \oddsidemargin=-1.3cm
\textheight=26cm \topmargin=-3.0cm

%% Математика
\usepackage{amsmath, amsfonts, amssymb, amsthm, mathtools}
\usepackage{icomma}

% Операторы
\DeclareMathOperator{\tr}{tr}
\renewcommand{\le}{\leqslant}
\renewcommand{\ge}{\geqslant}
\renewcommand{\leq}{\leqslant}
\renewcommand{\geq}{\geqslant}

% Множества
\def \R{\mathbb{R}}
\def \N{\mathbb{N}}
\def \Z{\mathbb{Z}}
\newcommand{\rk}{\operatorname{\mathrm{rk}}}

\def \a{\alpha}
\def \be{\beta}

% Другое

\newcommand{\const}{\mathrm{const}}
\theoremstyle{definition}
\newtheorem{Task}{Задача}
%\newtheorem*{Taskn}{Задача {#1}}
\newtheorem*{Sol}{Решение}
\usepackage[dvipsnames]{xcolor}

\newcommand{\heart}{\begin{center}
\textcolor{RoyalPurple}{\ensuremath\heartsuit} 
\end{center}}
\usepackage{mathtools}
\usepackage{nicefrac}

%% Гиперссылки
\usepackage{xcolor}
\usepackage{hyperref}
\definecolor{linkcolor}{HTML}{8b00ff}
\hypersetup{colorlinks = true,
			linkcolor = linkcolor,
			urlcolor = linkcolor,
			citecolor = linkcolor}

%% Выравнивание
% \setlength{\parskip}{0.5em} % Расстояние между абзацами
\usepackage{geometry} % Поля
\geometry{
	a4paper,
	left=12mm,
	top=10mm,
	right=12mm}
% \setlength{\parindent}{0cm} % Отступ (красная строка)
% \linespread{1.1} % Интерлиньяж
\usepackage[many]{tcolorbox}  
\usepackage{enumitem}

%% Оформление

% Код
\newcommand{\code}[1]{{\tt #1}}

\newenvironment{amatrix}[2]{%
    \left(\begin{array}{@{}*{#1}{c}|*{#2}{c}@{}}
}{%
    \end{array}\right)
}

\begin{document}

\begin{center}
\small
\noindent\makebox[\textwidth]{Линейная алгебра и геометрия \hfill ФКН НИУ ВШЭ, 2022/2023 учебный год, 1-й курс ОП ПМИ, группа 2212}
\end{center}

\large

\begin{center}
\textbf{Семинар 16 (17.01.2023)}
\end{center}

\textbf{Краткое содержание}

Поговорили про понятие линейной независимости подпространств и пять эквивалентных условий, определяющих эту линейную независимость.
Обсудили разложение векторного пространства в прямую сумму подпространств (по определению это означает, что подпространства линейно 
независимы и в сумме дают всё пространство).

Разобрали частный случай двух подпространств: подпространства $U,W$ векторного пространства $V$ линейно независимы тогда и только тогда, когда 
$U \cap W = \lbrace 0 \rbrace$ или, эквивалентно, $\dim U + \dim W = \dim (U+W)$ (см. пять эквивалентных условий).
Отсюда следует, что
\begin{equation} \label{eq}
V = U \oplus W \quad \Leftrightarrow \quad U \cap W = \lbrace 0 \rbrace \ \text{ и } \ U + W = V \quad \Leftrightarrow \quad 
U \cap W = \lbrace 0 \rbrace \ \text{ и } \ \dim U + \dim W = \dim V.
\end{equation}
(Последнее условие часто удобнее всего проверять в конкретных задачах.)
Если есть разложение в прямую сумму $V = U \oplus W$, то тогда всякий вектор $v \in V$ единственным образом представляется в 
виде суммы $v = u + w$, где $u \in U$ и $w \in W$. В этом случае $u$ называется проекцией вектора $v$ на $U$ вдоль $W$, а $w$ 
называется проекцией вектора $v$ на $W$ вдоль $U$.

В качестве простейшего примера разобрали разложение пространства $\R^2$ в прямую сумму двух одномерных подпространств, 
из которых первое --- это ось $Ox$, а второе --- прямая $y = x$. Разобрали графически, как в этом случае находятся обе 
проекции для произвольного вектора. Аналогично рассмотрели разложение $\R^2$ в прямую сумму оси $Ox$ и другой прямой и 
обсудили, как устроены обе проекции в этом случае. Подчеркнули, что для одного и того же вектора проекции на одну и ту же 
ось $Ox$ вдоль разных дополнительных прямых отличаются (вообще говоря).

Если $U = \langle u_1,\dots, u_k \rangle$ и $W = \langle w_1,\dots, w_m\rangle$ --- два подпространства в $F^n$, 
то разложение $F^n = U \oplus W$ удобно доказывать при помощи одного из условий в (\ref{eq}).
Чтобы найти проекции заданного вектора $v \in F^n$ на $U$ вдоль $W$ и наоборот, достаточно решить СЛУ 
$\lambda_1u_1 + \dots \lambda_k u_k + \mu_1 w_1+ \dots + \mu_m w_m = v$ относительно неизвестных $\lambda_i, \mu_j$, 
и тогда искомые проекции будут равны $\lambda_1u_1 + \dots \lambda_k u_k$ и $\mu_1 w_1+ \dots + \mu_m w_m$.

Разобрали номер 35.18.

Следующий вопрос: верно ли, что если три подпространства $U_1, U_2, U_3$ векторного пространства $V$ удовлетворяют 
условию $U_1 \cap U_2 = U_1 \cap U_3 = U_2 \cap U_3 = \lbrace 0 \rbrace$, то они линейно независимы? В общем случае неверно: 
в качестве примера можно взять три различных прямых в $\R^2$.

\heart
\textbf{Домашнее задание к семинару 16. Дедлайн 24.01.2023}

Номера с пометкой П даны по задачнику Проскурякова, с пометкой К -- Кострикина.

\begin{enumerate}

    \item К35.19

    \item К35.21

    \item
    Рассмотрим в пространстве $M_n(\R)$ подпространства $U$ и $W$, где $U$ состоит из всех симметричных матриц, а $W$ --- 
    из всех строго верхнетреугольных (то есть верхнетреугольных с нулями на диагонали) матриц. Докажите, что $M_n(\R) = U \oplus W$, 
    и найдите проекцию матрицы из предыдущей задачи на каждое из этих подпространств вдоль другого.

    \item
    Рассмотрим в пространстве $\R^5$ подпространства $U_1 = \langle (1,1,1,1,0) \rangle$, $U_2 = \langle (0,1,0,0,-1) \rangle$ и $U_3$, 
    являющееся множеством решений системы
    \[
    \begin{cases}
    x_1 + x_2 = 0,\\
    x_3 - x_5 = 0.
    \end{cases}
    \]
    Докажите, что $\R^5 = U_1 \oplus U_2 \oplus U_3$.

\end{enumerate}
\heart
    
\end{document}
