\documentclass[10pt, a4paper]{extarticle}

%% Язык
\usepackage{cmap} % Поиск в PDF
\usepackage{mathtext} % Кириллица в формулах
\usepackage[T2A]{fontenc} % Кодировка
\usepackage[utf8]{inputenc} % Кодировка
\usepackage[english,russian]{babel} % Локализация, переносы

\pagestyle{empty} \textwidth=19.0cm \oddsidemargin=-1.3cm
\textheight=26cm \topmargin=-3.0cm

%% Математика
\usepackage{amsmath, amsfonts, amssymb, amsthm, mathtools}
\usepackage{icomma}

% Операторы
\DeclareMathOperator{\tr}{tr}
\renewcommand{\le}{\leqslant}
\renewcommand{\ge}{\geqslant}
\renewcommand{\leq}{\leqslant}
\renewcommand{\geq}{\geqslant}

% Множества
\def \R{\mathbb{R}}
\def \N{\mathbb{N}}
\def \Z{\mathbb{Z}}

\def \a{\alpha}
\def \be{\beta}

% Другое

\newcommand{\const}{\mathrm{const}}
\theoremstyle{definition}
\newtheorem{Task}{Задача}
%\newtheorem*{Taskn}{Задача {#1}}
\newtheorem*{Sol}{Решение}
\usepackage[dvipsnames]{xcolor}

\newcommand{\heart}{\begin{center}
\textcolor{RoyalPurple}{\ensuremath\heartsuit} 
\end{center}}
\usepackage{mathtools}
\usepackage{nicefrac}

%% Гиперссылки
\usepackage{xcolor}
\usepackage{hyperref}
\definecolor{linkcolor}{HTML}{8b00ff}
\hypersetup{colorlinks = true,
			linkcolor = linkcolor,
			urlcolor = linkcolor,
			citecolor = linkcolor}

%% Выравнивание
% \setlength{\parskip}{0.5em} % Расстояние между абзацами
\usepackage{geometry} % Поля
\geometry{
	a4paper,
	left=12mm,
	top=10mm,
	right=12mm}
% \setlength{\parindent}{0cm} % Отступ (красная строка)
% \linespread{1.1} % Интерлиньяж
\usepackage[many]{tcolorbox}  

%% Оформление

% Код
\newcommand{\code}[1]{{\tt #1}}

\newenvironment{amatrix}[2]{%
    \left(\begin{array}{@{}*{#1}{c}|*{#2}{c}@{}}
}{%
    \end{array}\right)
}

\begin{document}

\begin{center}
\small
\noindent\makebox[\textwidth]{Линейная алгебра и геометрия \hfill ФКН НИУ ВШЭ, 2022/2023 учебный год, 1-й курс ОП ПМИ, группа 2212}
\end{center}

\large

\begin{center}
\textbf{Семинар 9 (7.11.2022)}
\end{center}

\textbf{Краткое содержание}

Продолжили тему комплексных чисел. Рассмотрели приложения формулы Муавра к нахождению выражений $\sin nx$ и $\cos nx$
через $\sin x$ и  $\cos x$, а также к вычислению сумм вида
\[
 \sin x + \sin 2x + \sin 3x + \dots + \sin nx \; \text{ и } \cos x + \cos 2x + \cos 3x + \dots + \cos nx 
\]
Разобрали номер К23.1(а, б).

Новая тема -- векторные пространства. Обсудили следствия 8 аксиом векторного пространства. Доказали, что 
$\alpha \cdot \vec 0 = \vec 0$ и $\alpha(-x) = -(\alpha x)$ для всех $\alpha \in F$ и $x \in V$.

Дальше обсудили понятие подпространства векторного пространства, посмотрели разные примеры.
Выяснили, какие бывают подпространства в $\R^2$.
\heart

\textbf{Домашнее задание к семинару 10. Дедлайн 14.11.2022}

Номера с пометкой П даны по задачнику Проскурякова, с пометкой К -- Кострикина.

\begin{enumerate}
    \item К21.4(а,в) (доказать алгебраически, не используя геометрическую интерпретацию)
    \item К21.11(в,г)
    \item К23.2(в,г)
    \item К22.1
    \item К22.7(б, з, л, р)
    
    \item Пусть $V$~--- векторное пространство.
    Докажите, что для всякого вектора $x \in V$ справедливы равенства $0 \cdot x = \vec 0$ и $(-1) \cdot x = -x$.

    \item К35.1(м,н)

    \item К35.2(а,б,г) (требуется только доказать, что это подпространства, вопрос про базисы и размерности игнорировать)

    \item К35.3(в,г,д,е) (требуется только выяснить, являются ли данные подмножества подпространствами, вопрос про базисы и размерности игнорировать)

\end{enumerate}
\heart
    
\end{document}
