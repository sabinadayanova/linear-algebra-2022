\documentclass[10pt, a4paper]{extarticle}

%% Язык
\usepackage{cmap} % Поиск в PDF
\usepackage{mathtext} % Кириллица в формулах
\usepackage[T2A]{fontenc} % Кодировка
\usepackage[utf8]{inputenc} % Кодировка
\usepackage[english,russian]{babel} % Локализация, переносы
\usepackage{bbold} % для ажурных буковок

\pagestyle{empty} \textwidth=19.0cm \oddsidemargin=-1.3cm
\textheight=26cm \topmargin=-3.0cm

%% Математика
\usepackage{amsmath, amsfonts, amssymb, amsthm, mathtools}
\usepackage{icomma}
\usepackage{algpseudocode}


% Операторы
\DeclareMathOperator{\tr}{tr}
\renewcommand{\le}{\leqslant}
\renewcommand{\ge}{\geqslant}
\renewcommand{\leq}{\leqslant}
\renewcommand{\geq}{\geqslant}

% Множества
\def \R{\mathbb{R}}
\def \N{\mathbb{N}}
\def \Z{\mathbb{Z}}
\def \CC{\mathbb{C}}
\newcommand{\rk}{\operatorname{\mathrm{rk}}}
\newcommand{\diag}{\operatorname{diag}}
\newcommand{\pr}{\mathop{\mathrm{pr}}}
\newcommand{\ort}{\mathop{\mathrm{ort}}}
\newcommand{\Ker}{\mathop{\mathrm{Ker}}}
\renewcommand{\Im}{\mathop{\mathrm{Im}}}
\newcommand{\Vol}{\mathop{\mathrm{Vol}}}

\def \a{\alpha}
\def \be{\beta}

% Другое

\newcommand{\const}{\mathrm{const}}
\theoremstyle{definition}
\newtheorem*{proposal}{Предложение}
\newtheorem{Task}{Задача}
%\newtheorem*{Taskn}{Задача {#1}}
\newtheorem*{Sol}{Решение}
\usepackage[dvipsnames]{xcolor}

\newcommand{\heart}{\begin{center}
\textcolor{RoyalPurple}{\ensuremath\heartsuit} 
\end{center}}
\usepackage{mathtools}
\usepackage{nicefrac}

%% Гиперссылки
\usepackage{xcolor}
\usepackage{hyperref}
\definecolor{linkcolor}{HTML}{8b00ff}
\hypersetup{colorlinks = true,
			linkcolor = linkcolor,
			urlcolor = linkcolor,
			citecolor = linkcolor}

%% Выравнивание
% \setlength{\parskip}{0.5em} % Расстояние между абзацами
\usepackage{geometry} % Поля
\geometry{
	a4paper,
	left=12mm,
	top=10mm,
    bottom=20mm,
	right=12mm}
% \setlength{\parindent}{0cm} % Отступ (красная строка)
% \linespread{1.1} % Интерлиньяж
\usepackage[many]{tcolorbox}  
\usepackage{enumitem}

%% Оформление

% Код
\newcommand{\code}[1]{{\tt #1}}

\newenvironment{amatrix}[2]{%
    \left(\begin{array}{@{}*{#1}{c}|*{#2}{c}@{}}
}{%
    \end{array}\right)
}

\begin{document}

\begin{center}
\small
\noindent\makebox[\textwidth]{Линейная алгебра и геометрия \hfill ФКН НИУ ВШЭ, 2022/2023 учебный год, 1-й курс ОП ПМИ, группа 2212}
\end{center}

\large

\begin{center}
\textbf{Семинар 28 (26.04.2023)}
\end{center}

\textbf{Краткое содержание}

Обсудили инвариантные подпространства.
На примере линейного оператора из домашнего задания показали, как для линейного оператора над $\R$ отыскивать двумерное инвариантное подпространство над $\R$, отвечающее комплексному собственному значению.

Дальше доказали, что если характеристический многочлен линейного оператора разлагается на линейные множители, то существует базис исходного пространства, в котором матрица линейного оператора имеет верхнетреугольный вид.

Дальше сформулировали теорему о жордановой нормальной форме.

Новая тема: сопряжённые линейные отображения, сопряжённые линейные операторы, самосопряжённые линейные операторы в евклидовых пространствах.
Обсудили, что если $A$ --- матрица линейного отображения в паре ортонормированных базисов $(\mathbb e, \mathbb f)$, то матрицей сопряжённого линейного отображения в паре базисов $(\mathbb f, \mathbb e)$ будет $A^T$.
Аналогично, если $A$ --- матрица линейного оператора в каком-то ортонормированном базисе, то матрицей сопряжённого линейного оператора в том же базисе будет $A^T$.
В частности, линейный оператор самосопряжён тогда и только тогда, когда его матрица в ортонормированном базисе симметрична.

Дальше сформулировали основную теорему о самосопряжённых линейных операторах:

Пусть $\varphi$ --- самосопряжённый линейный оператор в евклидовом пространстве $\mathbb E$.
Тогда:

1) существует ортонормированный базис в $\mathbb E$, состоящий из собственных векторов оператора $\varphi$ (в частности, $\varphi$ диагонализуем над $\R$);

2) собственные подпространства оператора $\varphi$, отвечающие различным собственным значениям, попарно ортогональны.

Разобрали пример с самосопряжённым оператором, заданным в некотором ортонормированном базисе матрицей $\begin{pmatrix} 1 & 1 & 1\\ 1 & 1 & 1 \\ 1 & 1 & 1 \end{pmatrix}$: нашли ортонормированный базис, в котором матрица этого оператора диагональна.

\heart

\textbf{Домашнее задание к семинару 29. Дедлайн 10.05.2023}

Номера с пометкой П даны по задачнику Проскурякова, с пометкой К -- Кострикина, с пометкой КК -- Ким-Крицкова.

\begin{enumerate}

	
	\item
	П1519
	
	
	
	\item
	К40.14 (характеристические числа матрицы --- это корни её характеристического многочлена, то есть собственные значения)

	\item
	Докажите, что операция перехода к сопряжённому линейному отображению в евклидовых пространствах обладает свойствами из номера К44.1(а,б,г,д) (в пункте (г) игнорировать черту над $\lambda$).

	\item
	Пусть $\varphi \colon \mathbb E \to \mathbb E'$ --- линейное отображение евклидовых пространств.
	Докажите, что \\
	\[
	\Ker \varphi^* = (\Im \varphi)^\perp \ \text{ и } \ \Im \varphi^* = (\Ker \varphi)^\perp.
	\]

	\item
	К44.4 (здесь речь именно о сопряжённом операторе, а не отображении!) + указать геометрический смысл сопряжённого оператора

	\item
	П1585

	\item
	П1586




\end{enumerate}
\heart


\end{document}
