\documentclass[10pt, a4paper]{extarticle}

%% Язык
\usepackage{cmap} % Поиск в PDF
\usepackage{mathtext} % Кириллица в формулах
\usepackage[T2A]{fontenc} % Кодировка
\usepackage[utf8]{inputenc} % Кодировка
\usepackage[english,russian]{babel} % Локализация, переносы
\usepackage{bbold} % для ажурных буковок

\pagestyle{empty} \textwidth=19.0cm \oddsidemargin=-1.3cm
\textheight=26cm \topmargin=-3.0cm

%% Математика
\usepackage{amsmath, amsfonts, amssymb, amsthm, mathtools}
\usepackage{icomma}


% Операторы
\DeclareMathOperator{\tr}{tr}
\renewcommand{\le}{\leqslant}
\renewcommand{\ge}{\geqslant}
\renewcommand{\leq}{\leqslant}
\renewcommand{\geq}{\geqslant}

% Множества
\def \R{\mathbb{R}}
\def \N{\mathbb{N}}
\def \Z{\mathbb{Z}}
\newcommand{\rk}{\operatorname{\mathrm{rk}}}
\newcommand{\Ker}{\mathop{\mathrm{Ker}}}
\renewcommand{\Im}{\mathop{\mathrm{Im}}}

\def \a{\alpha}
\def \be{\beta}

% Другое

\newcommand{\const}{\mathrm{const}}
\theoremstyle{definition}
\newtheorem*{proposal}{Предложение}
\newtheorem{Task}{Задача}
%\newtheorem*{Taskn}{Задача {#1}}
\newtheorem*{Sol}{Решение}
\usepackage[dvipsnames]{xcolor}

\newcommand{\heart}{\begin{center}
\textcolor{RoyalPurple}{\ensuremath\heartsuit} 
\end{center}}
\usepackage{mathtools}
\usepackage{nicefrac}

%% Гиперссылки
\usepackage{xcolor}
\usepackage{hyperref}
\definecolor{linkcolor}{HTML}{8b00ff}
\hypersetup{colorlinks = true,
			linkcolor = linkcolor,
			urlcolor = linkcolor,
			citecolor = linkcolor}

%% Выравнивание
% \setlength{\parskip}{0.5em} % Расстояние между абзацами
\usepackage{geometry} % Поля
\geometry{
	a4paper,
	left=12mm,
	top=10mm,
    bottom=20mm,
	right=12mm}
% \setlength{\parindent}{0cm} % Отступ (красная строка)
% \linespread{1.1} % Интерлиньяж
\usepackage[many]{tcolorbox}  
\usepackage{enumitem}

%% Оформление

% Код
\newcommand{\code}[1]{{\tt #1}}

\newenvironment{amatrix}[2]{%
    \left(\begin{array}{@{}*{#1}{c}|*{#2}{c}@{}}
}{%
    \end{array}\right)
}

\begin{document}

\begin{center}
\small
\noindent\makebox[\textwidth]{Линейная алгебра и геометрия \hfill ФКН НИУ ВШЭ, 2022/2023 учебный год, 1-й курс ОП ПМИ, группа 2212}
\end{center}

\large

\begin{center}
\textbf{Семинар 21 (21.02.2023)}
\end{center}

\textbf{Краткое содержание}

В начале обсудили индексы инерции квадратичной формы над $\R$ и закон инерции.

Дальше разобрали понятия положительной, отрицательной, неотрицательной, неположительной определённости и неопределённости квадратичной формы над $\R$.
Определили нормальный вид квадратичной формы из номера П1212 при каждом значении параметра.
Затем в той же квадратичной форме заменили $5x_1^2$ на $4x_1^2$ и обсудили для новой формы тот же вопрос.
Важный момент: как правило, метод Якоби даёт нормальный вид сразу для всех значений параметра кроме конечного числа \guillemotleft критических\guillemotright{} значений.
Для критических значений можно снова попытаться применить метод Якоби, изменив нумерацию переменных, или же воспользоваться симметричным Гауссом.

Сформулировали критерий Сильвестра положительной определённости и критерий отрицательной определённости квадратичной формы в терминах угловых миноров её матрицы.
Посмотрели, как работает критерий Сильвестра для обеих уже разобранных выше квадратичных форм.
Важный момент: критерий Сильвестра работает всегда, в то время как для метода Якоби требуется, чтобы все угловые миноры (ну, кроме последнего) были отличны от нуля.
Исследовали на отрицательную определённость квадратичную форму из номера К38.14(а), а также обсудили, как искать её нормальный вид в зависимости от значений параметра.

Следующий сюжет --- понятие эквивалентности квадратичных форм.
Обсудили, что две квадратичные формы над $\R$ эквивалентны тогда и только тогда, когда у них один и тот же нормальный вид, то есть совпадают положительный и отрицательный индексы инерции.
Вывели отсюда явную формулу для количества классов эквивалентности квадратичных форм на $n$-мерном векторном пространстве над $\R$: этих классов ровно $\frac{(n+1)(n+2)}2$.

Новая тема --- евклидовы пространства.
Обсудили следующие примеры конечномерных евклидовых пространств:

$\R^n$ со стандартным скалярным произведением $(x,y) = x_1y_1 + \dots x_ny_n$;

$\R[x]_{\leqslant n}$ со скалярным произведением $(f,g) = \int\limits_a^bf(x)g(x)\,dx$ (где $[a,b]$ --- некоторый отрезок);

$\R[x]_{\leqslant n}$ со скалярным произведением $(f,g) = f(0)g(0) + f(1)g(1) + \dots + f(n)g(n)$.

Ввели понятие длины вектора в евклидовом пространстве, обсудили неравенство 
\newline Коши--Буняковского и затем определили угол между двумя векторами.

\heart

\textbf{Домашнее задание к семинару 22. Дедлайн 28.02.2023}

Номера с пометкой П даны по задачнику Проскурякова, с пометкой К -- Кострикина.

В обоих задачниках координаты векторов из $\R^n$ всегда записываются в строчку через запятую, однако нужно помнить, что мы всегда записываем эти координаты в столбец.


\begin{enumerate}


	\item
	К38.11(б,в)
	
	\item
	Определите нормальный вид квадратичной формы из номера К38.11(г) для каждого значения параметра.
	
	\item
	К38.14(б) + определить нормальный вид этой квадратичной формы в зависимости от значений параметра
	
	\item
	Определите все значения параметров $a$ и $b$, при которых билинейная форма
	\[
	\beta(x,y)= x_1y_1+x_1y_2+x_2y_1+ax_2y_2+bx_2y_3-x_3y_2+2x_3y_3
	\]
	задаёт скалярное произведение в $\R^3$.
	
	\item
	П1385
	
	В следующем номере для простоты давайте считать, что $n$-мерный куб с ребром $a$ в $\R^n$ состоит из всех точек, у которых каждая координата принадлежит отрезку $[0,a]$. Вершины куба --- это точки, у которых каждая координата равна $0$ или $a$, рёбра --- отрезки, соединяющие две соседние вершины (отличающиеся одной координатой), диагонали --- отрезки, соединяющие противоположные вершины (то есть различающиеся в каждой координате).
	
	\item
	П1394,
	П1395
	
	\item
	Рассмотрим евклидово пространство $\R[x]_{\leqslant 3}$ со скалярным произведением $(f,g) = \int \limits_0^1 f(t)g(t)\,dt$.
	Найдите в этом пространстве угол между векторами $x^3$ и $x^2 + x + 1$.

\end{enumerate}
\heart


\end{document}
