\documentclass[10pt, a4paper]{extarticle}

%% Язык
\usepackage{cmap} % Поиск в PDF
\usepackage{mathtext} % Кириллица в формулах
\usepackage[T2A]{fontenc} % Кодировка
\usepackage[utf8]{inputenc} % Кодировка
\usepackage[english,russian]{babel} % Локализация, переносы

\pagestyle{empty} \textwidth=19.0cm \oddsidemargin=-1.3cm
\textheight=26cm \topmargin=-3.0cm

%% Математика
\usepackage{amsmath, amsfonts, amssymb, amsthm, mathtools}
\usepackage{icomma}

% Операторы
\DeclareMathOperator{\tr}{tr}
\renewcommand{\le}{\leqslant}
\renewcommand{\ge}{\geqslant}
\renewcommand{\leq}{\leqslant}
\renewcommand{\geq}{\geqslant}

% Множества
\def \R{\mathbb{R}}
\def \N{\mathbb{N}}
\def \Z{\mathbb{Z}}

\def \a{\alpha}
\def \be{\beta}

% Другое

\newcommand{\const}{\mathrm{const}}
\theoremstyle{definition}
\newtheorem{Task}{Задача}
%\newtheorem*{Taskn}{Задача {#1}}
\newtheorem*{Sol}{Решение}
\usepackage[dvipsnames]{xcolor}

\newcommand{\heart}{\begin{center}
\textcolor{RoyalPurple}{\ensuremath\heartsuit} 
\end{center}}
\usepackage{mathtools}
\usepackage{nicefrac}

%% Гиперссылки
\usepackage{xcolor}
\usepackage{hyperref}
\definecolor{linkcolor}{HTML}{8b00ff}
\hypersetup{colorlinks = true,
			linkcolor = linkcolor,
			urlcolor = linkcolor,
			citecolor = linkcolor}

%% Выравнивание
% \setlength{\parskip}{0.5em} % Расстояние между абзацами
\usepackage{geometry} % Поля
\geometry{
	a4paper,
	left=12mm,
	top=10mm,
	right=12mm}
\setlength{\parindent}{0cm} % Отступ (красная строка)
% \linespread{1.1} % Интерлиньяж
\usepackage[many]{tcolorbox}  

%% Оформление

% Код
\newcommand{\code}[1]{{\tt #1}}

\newenvironment{amatrix}[2]{%
    \left(\begin{array}{@{}*{#1}{c}|*{#2}{c}@{}}
}{%
    \end{array}\right)
}

\begin{document}

\begin{center}
\small
\noindent\makebox[\textwidth]{Линейная алгебра и геометрия \hfill ФКН НИУ ВШЭ, 2022/2023 учебный год, 1-й курс ОП ПМИ, группа 2212}
\end{center}

\large

\begin{center}
\textbf{Семинар 3 (23.09.2022)}
\end{center}

\textbf{Краткое содержание}
Повторили определение системы линейных уравнений (СЛУ) и разобрали число решений СЛУ в самом простом случае. Так,

$0 \cdot x = 1 \implies $ нет решений. \\
$1 \cdot x = 1 \implies $ 1 решение. \\
$0 \cdot x = 0 \implies \infty $ решений.


Разобрали элементарные преобразования строк матрицы и приведение матрицы к ступенчатому и улучшенному ступенчатому виду. Решили СЛУ
$\begin{amatrix}{4}{1}
   1 & 1 & 2 & 3 & 4\\
   1 & 1 & -1 & 2 & 1
\end{amatrix}$. 
Привели ее к ул. ступ. виду 
$\begin{amatrix}{4}{1}
    1 & 1 & 0 & \nicefrac{7}{3} & 4\\
    0 & 0 & 1 & \nicefrac{1}{3} & 1
 \end{amatrix}$. Столбцы со ступеньками соответствуют главным переменным. Если переписывать это на лад уравнения, получаем

 \[
    \begin{cases}
        x_1 + x_2 + \frac{7}{3} x_4 = 4, \\
        x_3 + \frac{1}{3}x_4 = 1
    \end{cases}  
    \iff
    \begin{cases}
        x_1 = 4 - x_2 - \frac{7}{3} x_4, \\
        x_3 = 1 - \frac{1}{3}x_4
    \end{cases} 
 \]
 Можно заметить, что в каждом уравнении главная переменная только одна -- $x_1$ в первом, $x_3$ во втором. И каждая главная переменная выражается
 через свободные переменные и свободные члены. Решение СЛУ можно записать в трех видах
 \begin{itemize}
    \item $x_2 = a \in \R, x_4 = b\in \R, \quad
       \left(x_1, x_2, x_3, x_4\right) = \left(4 - a - \tfrac{7}{3} b, a, 1 - \tfrac{1}{3}b, b\right)$
    \item 
    $
        \begin{pmatrix}
        x_1 \\ x_3
        \end{pmatrix} = 
        \begin{pmatrix}
        4 \\ 1
        \end{pmatrix} - 
        \begin{pmatrix}
        1 \\ 0
        \end{pmatrix}x_2 - 
        \begin{pmatrix}
        \nicefrac{7}{3} \\ \nicefrac{1}{3}
        \end{pmatrix}x_4
    $
    \item 
    $
    \begin{pmatrix}
        x_1 \\ x_2 \\ x_3 \\ x_4
    \end{pmatrix} = 
    \begin{pmatrix}
        4 \\ 0 \\ 1 \\ 0
    \end{pmatrix} - 
    \begin{pmatrix}
        1 \\ 1 \\ 0 \\ 0
    \end{pmatrix}x_2 - 
    \begin{pmatrix}
        \nicefrac{7}{3} \\ 0 \\ \nicefrac{1}{3} \\ 1
    \end{pmatrix}x_4
    $
 \end{itemize}
 
Поговорили про целочисленные матрицы и как упрощать себе жизнь и оставаться в целых числах, приводя их к ул. ступ. виду. 

Поговорили про то, как в общем случае понять, сколько решений имеет СЛУ. После этого решили номер П88, чтобы закрепить усвоенное. Не успели дорешать, полное решение номера здесь:
$\begin{cases}
    ax + 4y + z = 0,\\
    2y + 3z - 1 = 0,\\
    3x - bz + 2 = 0.
\end{cases}
\sim 
\begin{amatrix}{3}{1}
a & 4 & 1 & 0\\
0 & 2 & 3 & 1 \\
3 & 0 & -b & -2 
\end{amatrix}
$

Хочу привести матрицу с ступенчатому виду. Для этого надо избавиться от $a$ или от тройки в первом столбце. В любом случае придется делить на $a$. Рассмотрим случай, когда $a=0$:

1. $a=0$
\[
\begin{amatrix}{3}{1}
0 & 4 & 1 & 0\\
0 & 2 & 3 & 1 \\
3 & 0 & -b & -2 
\end{amatrix} \sim
\begin{amatrix}{3}{1}
0 & 0 & -5 & -2\\
0 & 2 & 3 & 1 \\
3 & 0 & -b & -2 
\end{amatrix} \sim
\begin{amatrix}{3}{1}
3 & 0 & -b & -2 \\
0 & 2 & 3 & 1 \\
0 & 0 & -5 & -2\\
\end{amatrix}
\]
Привели к  ступенчатому виду, где ступеньки есть на каждом столбце. Значит, все переменные, главные, свободных переменных нет. в конечном счете, когда мы приведем матрицу к ул. ступ. виду, все переменные будут однозначно выражаться через свободные члены. Значит при $a=0, b\in R$ решение системы одно.

2. $a \neq 0$
\[
\begin{amatrix}{3}{1}
a & 4 & 1 & 0\\
0 & 2 & 3 & 1 \\
3 & 0 & -b & -2 
\end{amatrix} \sim
\begin{amatrix}{3}{1}
1 & \nicefrac{4}{a} & \nicefrac{1}{a} & 0\\
0 & 2 & 3 & 1 \\
3 & 0 & -b & -2 
\end{amatrix} \sim
\begin{amatrix}{3}{1}
1 & \nicefrac{4}{a} & \nicefrac{1}{a} & 0\\
0 & 2 & 3 & 1 \\
0 & -\nicefrac{12}{a} & -b - \nicefrac{3}{a} & -2 
\end{amatrix} \sim
\]
\[
\sim
\begin{amatrix}{3}{1}
1 & \nicefrac{4}{a} & \nicefrac{1}{a} & 0\\
0 & 1 & \nicefrac{3}{2} & \nicefrac{1}{2} \\
0 & -\nicefrac{12}{a} & -b - \nicefrac{3}{a} & -2 
\end{amatrix} \sim
\begin{amatrix}{3}{1}
1 & \nicefrac{4}{a} & \nicefrac{1}{a} & 0\\
0 & 1 & \nicefrac{3}{2} & \nicefrac{1}{2} \\
0 & 0 & -b - \nicefrac{3}{a} + \nicefrac{3}{2}\cdot \nicefrac{12}{a}& -2 + \nicefrac{1}{2}\cdot \nicefrac{12}{a} 
\end{amatrix} \sim
\]
\[
\sim \begin{amatrix}{3}{1}
1 & \nicefrac{4}{a} & \nicefrac{1}{a} & 0\\
0 & 1 & \nicefrac{3}{2} & \nicefrac{1}{2} \\
0 & 0 & -b - \nicefrac{15}{a} & -2 + \nicefrac{6}{a} 
\end{amatrix} 
\]
Число решений СЛУ зависит напрямую от двух чисел внизу в последней строчке в расширенной матрице.
Обозначим $-b - \nicefrac{15}{a}$ за $M$, $-2 + \nicefrac{6}{a} $ за $N$. Надо понять, что
\begin{enumerate}
    \item Если $M=0, N=0$, то СЛУ будет иметь бесконечное число решений.
    \item Если $M=0, N \neq 0$, то СЛУ решений иметь не будет.
    \item Если $M \neq 0, N = 0$, то СЛУ будет иметь одно решение.
\end{enumerate}

Можем решить простую систему уравнений на переменные $a, b$ и понять, что
\begin{enumerate}
    \item $M=0, N=0 \iff a = 3, b = 5$
    \item $M=0, N \neq 0 \iff a \in (-\infty, 3) \cup (3, +\infty), b = \tfrac{15}{a}$
    \item $M \neq 0, N = 0 \iff$ все остальные пары $a, b$.
\end{enumerate}

\heart

\textbf{Домашнее задание к семинару 4. Дедлайн 3.10.2022}

Номера с пометкой П даны по задачнику Проскурякова, с пометкой К -- Кострикина.

\textbf{В заданиях 1. -- 6. требуется решить СЛУ методом Гаусса.}
\begin{enumerate}
    \item П76
    \item П83
    \item П85
    \item П567
    \item П578
    \item П580
    \item Найдите число решений СЛУ из номера П89 в зависимости от значений параметров.
    \item П715
    \item П718
    \item Докажите, что элементарное преобразование второго типа можно выразить через 
    преобразования первого и третьего типов.
    \item По мотивам обсуждения на семинаре про целочисленные вычисления в методе Гаусса.
    Докажите, что всякую целочисленную матрицу можно привести к ступенчатому виду целочисленными
    элементарными преобразованиями строк только первого типа.
    \end{enumerate}
\heart
    
\end{document}
