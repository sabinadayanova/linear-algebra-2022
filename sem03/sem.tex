\documentclass[10pt, a4paper]{extarticle}

%% Язык
\usepackage{cmap} % Поиск в PDF
\usepackage{mathtext} % Кириллица в формулах
\usepackage[T2A]{fontenc} % Кодировка
\usepackage[utf8]{inputenc} % Кодировка
\usepackage[english,russian]{babel} % Локализация, переносы

\pagestyle{empty} \textwidth=19.0cm \oddsidemargin=-1.3cm
\textheight=26cm \topmargin=-3.0cm

%% Математика
\usepackage{amsmath, amsfonts, amssymb, amsthm, mathtools}
\usepackage{icomma}

% Операторы
\DeclareMathOperator{\tr}{tr}
\renewcommand{\le}{\leqslant}
\renewcommand{\ge}{\geqslant}
\renewcommand{\leq}{\leqslant}
\renewcommand{\geq}{\geqslant}

% Множества
\def \R{\mathbb{R}}
\def \N{\mathbb{N}}
\def \Z{\mathbb{Z}}

\def \a{\alpha}
\def \be{\beta}

% Другое

\newcommand{\const}{\mathrm{const}}
\theoremstyle{definition}
\newtheorem{Task}{Задача}
%\newtheorem*{Taskn}{Задача {#1}}
\newtheorem*{Sol}{Решение}
\usepackage[dvipsnames]{xcolor}

\newcommand{\heart}{\begin{center}
\textcolor{RoyalPurple}{\ensuremath\heartsuit} 
\end{center}}
\usepackage{mathtools}
\usepackage{nicefrac}

%% Гиперссылки
\usepackage{xcolor}
\usepackage{hyperref}
\definecolor{linkcolor}{HTML}{8b00ff}
\hypersetup{colorlinks = true,
			linkcolor = linkcolor,
			urlcolor = linkcolor,
			citecolor = linkcolor}

%% Выравнивание
% \setlength{\parskip}{0.5em} % Расстояние между абзацами
\usepackage{geometry} % Поля
\geometry{
	a4paper,
	left=12mm,
	top=10mm,
	right=12mm}
\setlength{\parindent}{0cm} % Отступ (красная строка)
% \linespread{1.1} % Интерлиньяж
\usepackage[many]{tcolorbox}  

%% Оформление

% Код
\newcommand{\code}[1]{{\tt #1}}

\newenvironment{amatrix}[2]{%
    \left(\begin{array}{@{}*{#1}{c}|*{#2}{c}@{}}
}{%
    \end{array}\right)
}

\begin{document}

\begin{center}
\small
\noindent\makebox[\textwidth]{Линейная алгебра и геометрия \hfill ФКН НИУ ВШЭ, 2022/2023 учебный год, 1-й курс ОП ПМИ, группа 2212}
\end{center}

\large

\begin{center}
\textbf{Семинар 3 (23.09.2022)}
\end{center}

\textbf{Краткое содержание}
Повторили определение системы линейных уравнений (СЛУ) и разобрали число решений СЛУ в самом простом случае. Так,

$0 \cdot x = 1 \implies $ нет решений. \\
$1 \cdot x = 1 \implies $ 1 решение. \\
$0 \cdot x = 0 \implies \infty $ решений.

Решили СЛУ
$\begin{amatrix}{4}{1}
   1 & 1 & 2 & 3 & 4\\
   1 & 1 & -1 & 2 & 1
\end{amatrix}$. 
Привели ее к ул. ступ. виду 
$\begin{amatrix}{4}{1}
    1 & 1 & 0 & \nicefrac{7}{3} & 4\\
    0 & 0 & 1 & \nicefrac{1}{3} & 1
 \end{amatrix}$. Столбцы со ступеньками соответствуют главным переменным. Если переписывать это на лад уравнения, получаем

 \[
    \begin{cases}
        x_1 + x_2 + \frac{7}{3} x_4 = 4, \\
        x_3 + \frac{1}{3}x_4 = 1
    \end{cases}  
    \iff
    \begin{cases}
        x_1 = 4 - x_2 - \frac{7}{3} x_4, \\
        x_3 = 1 - \frac{1}{3}x_4
    \end{cases} 
 \]
 Можно заметить, что в каждом уравнении главная переменная только одна -- $x_1$ в первом, $x_3$ во втором. И каждая главная переменная выражается
 через свободные переменные и свободные члены. Решение СЛУ можно записать в трех видах
 \begin{itemize}
    \item $x_2 = a \in \R, x_4 = b\in \R$
    \[
       \left(x_1, x_2, x_3, x_4\right) = \left(4 - a - \tfrac{7}{3} b, a, 1 - \tfrac{1}{3}b, b\right)
    \]
    ite
 \end{itemize}


\begin{equation}
    \begin{pmatrix} A \mid b \end{pmatrix} = \begin{amatrix}{4}{1}
    a_{11} & a_{12} & \dots & a_{1n} & b_1 \\
    a_{21} & a_{22} & \dots & a_{2n} & b_2 \\
    \vdots & \vdots & \ddots & \vdots \\
    a_{m1} & a_{m2} & \dots & a_{mn} & b_m
    \end{amatrix}
\end{equation}

TBD


Разобрали элементарные преобразования строк матрицы и приведение матрицы к ступенчатому и улучшенному ступенчатому виду.

\heart

\textbf{Домашнее задание к семинару 4 (4.10.2022)}

Номера с пометкой П даны по задачнику Проскурякова, с пометкой К -- Кострикина.

\textbf{В заданиях 1. -- 6. требуется решить СЛУ методом Гаусса.}
\begin{enumerate}
    \item П76
    \item П83
    \item П85
    \item П567
    \item П578
    \item П580
    \item Найдите число решений СЛУ из номера П89 в зависимости от значений параметров.
    \item П715
    \item П718
    \item Докажите, что элементарное преобразование второго типа можно выразить через 
    преобразования первого и третьего типов.
    \item По мотивам обсуждения на семинаре про целочисленные вычисления в методе Гаусса.
    Докажите, что всякую целочисленную матрицу можно привести к ступенчатому виду целочисленными
    элементарными преобразованиями строк только первого типа.
    \end{enumerate}
\heart
    
\end{document}