\documentclass[10pt, a4paper]{extarticle}

%% Язык
\usepackage{cmap} % Поиск в PDF
\usepackage{mathtext} % Кириллица в формулах
\usepackage[T2A]{fontenc} % Кодировка
\usepackage[utf8]{inputenc} % Кодировка
\usepackage[english,russian]{babel} % Локализация, переносы

\pagestyle{empty} \textwidth=19.0cm \oddsidemargin=-1.3cm
\textheight=26cm \topmargin=-3.0cm

%% Математика
\usepackage{amsmath, amsfonts, amssymb, amsthm, mathtools}
\usepackage{icomma}

% Операторы
\DeclareMathOperator{\tr}{tr}
\renewcommand{\le}{\leqslant}
\renewcommand{\ge}{\geqslant}
\renewcommand{\leq}{\leqslant}
\renewcommand{\geq}{\geqslant}

% Множества
\def \R{\mathbb{R}}
\def \N{\mathbb{N}}
\def \Z{\mathbb{Z}}

\def \a{\alpha}
\def \be{\beta}

% Другое

\newcommand{\const}{\mathrm{const}}
\theoremstyle{definition}
\newtheorem{Task}{Задача}
%\newtheorem*{Taskn}{Задача {#1}}
\newtheorem*{Sol}{Решение}
\usepackage[dvipsnames]{xcolor}

\newcommand{\heart}{\begin{center}
\textcolor{RoyalPurple}{\ensuremath\heartsuit} 
\end{center}}
\usepackage{mathtools}

%% Гиперссылки
\usepackage{xcolor}
\usepackage{hyperref}
\definecolor{linkcolor}{HTML}{8b00ff}
\hypersetup{colorlinks = true,
			linkcolor = linkcolor,
			urlcolor = linkcolor,
			citecolor = linkcolor}

%% Выравнивание
% \setlength{\parskip}{0.5em} % Расстояние между абзацами
\usepackage{geometry} % Поля
\geometry{
	a4paper,
	left=12mm,
	top=10mm,
	right=12mm}
\setlength{\parindent}{0cm} % Отступ (красная строка)
% \linespread{1.1} % Интерлиньяж
\usepackage[many]{tcolorbox}  

%% Оформление

% Код
\newcommand{\code}[1]{{\tt #1}}

\newenvironment{amatrix}[2]{%
  \left(\begin{array}{@{}*{#1}{c}|{}*{#2}{c}}
}{%
  \end{array}\right)
}

\begin{document}

\begin{center}
\small
\noindent\makebox[\textwidth]{Линейная алгебра и геометрия \hfill ФКН НИУ ВШЭ, 2022/2023 учебный год, 1-й курс ОП ПМИ, группа 2212}
\end{center}

\large

\begin{center}
\textbf{Семинар 2 (20.09.2022)}
\end{center}

\textbf{Краткое содержание}

Поговорили про способы упрощения выражений с матрицами со степенями. Так, $A^m \cdot A^n = A^{m+n}$
и  $\left(A^m\right)^n = A^{m\cdot n}$.

Определили такую операцию как многочлен от матрицы. Если $f(t) = a_n t^n + \dots + a_1t + a_0 \in \R,$ 

$X \in Mat_{m}(\R)$, то 
$f(X) = a_n X^n + \dots + a_1X + a_0E \in Mat_m(\R)$.

Поговорили про последовательность Фибоначчи, ее рекурсивную запись в обычном и матричном формате. 
Матричный формат:
\[
    \begin{pmatrix}
        a_1 \\ a_0
    \end{pmatrix} =
    \begin{pmatrix}
        1 \\ 0
    \end{pmatrix} , 
    \begin{pmatrix}
        a_{n+1}\\ a_n
    \end{pmatrix} =
    \begin{pmatrix}
        1 & 1 \\ 1 & 0
    \end{pmatrix}
    \begin{pmatrix}
        a_n\\ a_{n-1}
    \end{pmatrix}, 
    \begin{pmatrix}
        a_{n+1}\\ a_n
    \end{pmatrix} =
    \begin{pmatrix}
        1 & 1 \\ 1 & 0
    \end{pmatrix}^n
    \begin{pmatrix}
        1\\ 0
    \end{pmatrix}
\]
Разобрав П808 и упомянув понятие обратной матрицы, выяснили, что задача возведения матрицы $A$
в произвольную степень сильно упрощается, если известно разложение вида $A = BCB^{-1}$, где $C$ --
диагональная матрица. В этом случае получается $A^k = BC^k B^{-1}$. Задачу нахождения такого разложения
(а также вопрос его существования) отложили до будущих времен. Для матрицы 
$\begin{pmatrix}
    1 & 1 \\ 1 & 0
\end{pmatrix}$, 
про которую мы говорили выше, это разложение существует и выглядит как
\[
    \begin{pmatrix}
        \frac{1-\sqrt{5}}{2} & \frac{1+\sqrt{5}}{2}\\ 1 & 1
    \end{pmatrix}
    \begin{pmatrix}
        \frac{1-\sqrt{5}}{2} & 0 \\ 0 & \frac{1+\sqrt{5}}{2}
    \end{pmatrix}
    \begin{pmatrix}
        -\frac{1}{\sqrt{5}} & \frac{1+\sqrt{5}}{2\sqrt{5}}\\ 
        \frac{1}{\sqrt{5}} & -\frac{1-\sqrt{5}}{2\sqrt{5}}
    \end{pmatrix}    
\]
Проговорили свойства основных операций над матрицами, обсудили, как оптимально посчитать матричное
выражение $2 A A^T - 3AB + 4 B^T A^T - 6B^TB$. Также обсудили, как решать матричное уравнение 
$XX^T - XA^T - AX^T + AA^T=0$, в котором $X$ -- неизвестная матрица заданного порядка, а 
$A$ -- известная матрциа с коэффициентами из $\R$.

Упомянули понятие следа квадратной матрицы, а также его основное свойство $\tr{AB} = \tr{BA}$.
Обсудили, что если $A, B$ -- квадратные матрицы одного порядка и $B$ обратима, то $\tr{BAB^{-1}} == \tr A$.
Вывели отсюда, что матричное уравнение $
B 
\begin{pmatrix}
    -1 & 1 \\ 1 & 0
\end{pmatrix}
B^{-1}= 
\begin{pmatrix}
    1 & 0 \\ 0 & 2
\end{pmatrix}$ корней не имеет.
Проговорили, как сдвигать три матрицы под следом. Поняли, что по сути можно брать только матрицы с краю 
и перемещать их в другой край (то есть, по циклу). Так, $\tr{ABC} == \tr{BCA} == \tr{CAB}$. Выяснили, что 
равенство $\tr{ABC} = \tr{ACB}$ не верно для всех матриц. Построение примера предложено в качестве
домашнего задания.

В конце поупражнялись со свойствами следа, упростив выражение \\
$\tr{[(6AB^T-3BA^T)^TC + C(2AB^T-4BA^T)]}$, 
где $A, B, C \in M_n, \quad C=C^T$. 

\heart

\textbf{Домашнее задание к семинару 3 (23.09.2022). Дедлайн 27.09.2022}

Номера с пометкой П даны по задачнику Проскурякова, с пометкой К -- Кострикина.
\begin{enumerate}
    \item K17.5
    \item П829
    \item П809
    \item П815
    \item П832
    \item Приведите пример трех квадратных (2x2)-матриц $A, B, C$, для которых $\tr{ABC} \neq \tr{ACB}$.
    \item Квадратная матрица $A$ называется \textit{верхнетреугольной}, если $a_{ij} = 0$ при всех $i > j$, то есть все ее элементы ниже главной диагонали равны нулю. Докажите, что произведение двух верхнетреугольных матриц снова будет верхнетреугольной матрицей.
    \item Назовем квадратную матрицу побочно-диагональной, если все ее элементы вне побочной диагонали равны нулю. Пусть $A, B$ -- две побочно-диагональные матрицы, причем у матрицы $A$ (соответственно $B$) на побочной диагонали стоят элементы $\lambda_1, \lambda_2, \dots, \lambda_n$ (соответственно $\mu_1, \mu_2, \dots, \mu_n$), если считать от правого верхнего угла к левому нижнему. Найдите произведения $AB$ и $BA$ и определите условия, при которых матрицы $A$ и $B$ коммутируют (то есть $AB=BA$).
\end{enumerate}
\heart
    
\end{document}