\documentclass[10pt, a4paper]{extarticle}

%% Язык
\usepackage{cmap} % Поиск в PDF
\usepackage{mathtext} % Кириллица в формулах
\usepackage[T2A]{fontenc} % Кодировка
\usepackage[utf8]{inputenc} % Кодировка
\usepackage[english,russian]{babel} % Локализация, переносы

\pagestyle{empty} \textwidth=19.0cm \oddsidemargin=-1.3cm
\textheight=26cm \topmargin=-3.0cm

%% Математика
\usepackage{amsmath, amsfonts, amssymb, amsthm, mathtools}
\usepackage{icomma}

% Операторы
\DeclareMathOperator{\tr}{tr}
\renewcommand{\le}{\leqslant}
\renewcommand{\ge}{\geqslant}
\renewcommand{\leq}{\leqslant}
\renewcommand{\geq}{\geqslant}

% Множества
\def \R{\mathbb{R}}
\def \N{\mathbb{N}}
\def \Z{\mathbb{Z}}
\newcommand{\rk}{\operatorname{\mathrm{rk}}}

\def \a{\alpha}
\def \be{\beta}

% Другое

\newcommand{\const}{\mathrm{const}}
\theoremstyle{definition}
\newtheorem{Task}{Задача}
%\newtheorem*{Taskn}{Задача {#1}}
\newtheorem*{Sol}{Решение}
\usepackage[dvipsnames]{xcolor}

\newcommand{\heart}{\begin{center}
\textcolor{RoyalPurple}{\ensuremath\heartsuit} 
\end{center}}
\usepackage{mathtools}
\usepackage{nicefrac}

%% Гиперссылки
\usepackage{xcolor}
\usepackage{hyperref}
\definecolor{linkcolor}{HTML}{8b00ff}
\hypersetup{colorlinks = true,
			linkcolor = linkcolor,
			urlcolor = linkcolor,
			citecolor = linkcolor}

%% Выравнивание
% \setlength{\parskip}{0.5em} % Расстояние между абзацами
\usepackage{geometry} % Поля
\geometry{
	a4paper,
	left=12mm,
	top=10mm,
	right=12mm}
% \setlength{\parindent}{0cm} % Отступ (красная строка)
% \linespread{1.1} % Интерлиньяж
\usepackage[many]{tcolorbox}  
\usepackage{enumitem}

%% Оформление

% Код
\newcommand{\code}[1]{{\tt #1}}

\newenvironment{amatrix}[2]{%
    \left(\begin{array}{@{}*{#1}{c}|*{#2}{c}@{}}
}{%
    \end{array}\right)
}

\begin{document}

\begin{center}
\small
\noindent\makebox[\textwidth]{Линейная алгебра и геометрия \hfill ФКН НИУ ВШЭ, 2022/2023 учебный год, 1-й курс ОП ПМИ, группа 2212}
\end{center}

\large

\begin{center}
\textbf{Семинар 15 (10.01.2023)}
\end{center}

\textbf{Краткое содержание}

Начали с напоминания о том, что всякое подпространство в $F^n$ может быть задано следующими двумя способами:

I: как линейная оболочка конечной системы векторов;

II: как множество решений некоторой ОСЛУ.

Вспомнили, как осуществляется переход между этими двумя способами задания: от II к I через нахождение ФСР ОСЛУ; от I к II --- алгоритм, разобранный на прошлом семинаре. 

Основная тема семинара --- нахождение базиса суммы и базиса пересечения двух подпространств в $F^n$, заданных тем или иным способом.

Если подпространства $U$ и $W$ в $F^n$ заданы способом I, то есть $U = \langle u_1,\dots, u_k \rangle$ и $W = \langle w_1,\dots, w_m \rangle$, 
то про сумму подпространств можно сразу сказать, что $U+W = \langle u_1,\dots, u_k, w_1,\dots, w_m \rangle$, поэтому в данной ситуации легко найти базис суммы.

Если подпространства $U$ и $W$ в $F^n$ заданы способом II, то сразу видно, что их пересечение $U \cap W$ есть множество решений большой ОСЛУ, 
получающейся объединением двух ОСЛУ, задающих $U$ и $W$, поэтому в данной ситуации легко найти базис пересечения.

Из предыдущего вытекают следующие базовые алгоритмы:
\begin{itemize}
\item
если $U$ и $W$ заданы способом I, то для нахождения базиса в $U \cap W$ нужно перейти к заданию $U$ и $W$ способом II, в котором требуемая задача решается легко;

\item
если $U$ и $W$ заданы способом II, то для нахождения базиса в $U + W$ нужно перейти к заданию $U$ и $W$ способом I, в котором требуемая задача решается легко.
\end{itemize}

Используя данные алгоритмы, нашли явно базис суммы и базис пересечения двух подпространств из номера К35.14(а).

Дальше упомянули теорему о том, что для любых двух подпространств $U,W$ справедливо соотношение $\dim (U \cap W) + \dim (U + W) = \dim U + \dim W$.
Обсудили, что если оба подпространства $U,W$ заданы способом I или оба заданы способом II, то среди чисел $\dim (U + W)$, $\dim (U \cap W)$, $\dim U$, $\dim W$ 
три из них легко находятся, а четвёртое уже определяется по теореме.

Следующий сюжет --- альтернативный способ нахождения базиса пересечения двух подпространств $U,W \subseteq F^n$, заданных способом I.
А именно, если $U = \langle u_1,\dots, u_k \rangle$ и $W = \langle w_1,\dots, w_m \rangle$, то для всякого вектора $v \in U \cap W$ найдутся скаляры $\lambda_i, \mu_j$,
для которых $v = \lambda_1u_1+ \dots + \lambda_k u_k$ и $v = \mu_1w_1+ \dots + \mu_m w_m$. Найдя общее решение уравнения $\lambda_1u_1 + \dots + \lambda_k u_k = \mu_1w_1 + \dots + \mu_m w_m$,
где $\lambda_i, \mu_j$ -- наши неизвестные, мы сможем описать все те векторы $v$, лежащие в пересечении подпространств. Вышеупомянутое уравнение можно переписать в более удобном формате:

\[
    (u_1 \mid u_2 \mid \dots \mid u_k) \begin{pmatrix}
        \lambda_1 \\ \lambda_2 \\ \dots \\ \lambda_k
    \end{pmatrix} - 
    (w_1 \mid w_2 \mid \dots \mid w_m) \begin{pmatrix}
        \mu_1 \\ \mu_2 \\ \dots \\ \mu_m
    \end{pmatrix} = 0
\]

Обозначив матрицу из $u_i$ как $A$, матрицу из $w_j$ как $B$, можно упростить выражение еще сильнее:

\[
    A \lambda - B \mu = 0 \iff (A \mid B) \begin{pmatrix}
        \lambda \\ -\mu
    \end{pmatrix} = 0
\]

Значит, чтобы найти все $\lambda_i, \mu_j$, удовлетворяющие уравнению, достаточно просто решить эту ОСЛУ.
Отсюда вытекает следующий алгоритм нахождения базиса в $U \cap W$:

\begin{enumerate}
    \item записываем векторы $u_1,\dots, u_k$ в столбцы матрицы $A$, а векторы $w_1,\dots, w_m$ --- в столбцы матрицы $B$;
    \item приводим матрицу $(A \mid B)$ к улучшенному ступенчатому виду. на разделительную черту внимания не обращаем, она имеет 
    чисто вспомогательный характер, приводим к у.с. виду до конца;
    \item по улучшенному ступенчатому виду выписываем ФСР для нашей ОСЛУ 
    \item для каждого элемента найденной ФСР находим вектор $v$ по одной из формул $v = \lambda_1u_1+ \dots + \lambda_k u_k$ или 
    $v = \mu_1w_1+ \dots + \mu_m w_m$. \textbf{замечание 1}: при нахождении ФСР не забывайте, что вы находите коэффициенты $\mu_j$ со знаком минус,
    и перед подстановкой в формулу, эти коэффициенты нужно умножить на -1. \textbf{замечание 2}: обе формулы должны давать один и тот же результат;
     если это не так, то вы где-то ошиблись!!;
    \item для всех полученных векторов $v$ выбираем базис их линейной оболочки, это и будет базис в $U \cap W$.

\end{enumerate}

Применили данный алгоритм к всё той же паре подпространств из К35.14(а) и нашли (другим способом) базис их пересечения.

Наблюдение: чтобы найти базис в $U+W$, нужно всё ту же матрицу $(A \mid B)$ привести к ступенчатому виду.
Таким образом, разобранный выше альтернативный способ даёт возможность найти базис в $U + W$ и $U\cap W$, работая с одной и той же матрицей $(A \mid B)$.

Вернёмся к разобранному выше альтернативному алгоритму нахождения базиса в $U \cap W$, где $U = \langle u_1,\dots, u_k \rangle$ и $W = \langle w_1,\dots, w_m \rangle$.

Утверждение $\diamond$: если каждый из двух наборов $u_1,\dots, u_k$ и $w_1,\dots, w_m$ линейно независим, то все полученные на шаге 4 алгоритма векторы $v$ автоматически образуют базис
 в $U \cap W$, так что в этой ситуации шаг 5 можно опустить. Доказательство остается в качестве бонуса к домашнему заданию.

\heart
\textbf{Домашнее задание к семинару 16. Дедлайн 17.01.2023}

Номера с пометкой П даны по задачнику Проскурякова, с пометкой К -- Кострикина.

\begin{enumerate}

    \item
    Подпространства $U$ и $W$ в $F^4$ заданы как множества решений ОСЛУ
    \begin{equation*} \label{systems}
    \begin{cases}
    x_1 + 2x_2 + x_4 = 0,\\
    x_1 + x_2 + x_3 = 0
    \end{cases}
    \qquad \text{и} \qquad
    \begin{cases}
    x_1 + x_3 = 0,\\
    x_1 + 3x_2 + x_4 = 0
    \end{cases}
    \end{equation*}
    соответственно.
    Найдите базис в $U \cap W$ и базис в $U + W$.

    \item
    Подпространства $U$ и $W$ в $F^5$ заданы как множества решений ОСЛУ
    \[
    \begin{cases}
    x_1 + x_2 - x_5 = 0,\\
    x_2 + x_3 + x_5 = 0,\\
    x_3 + x_4 + x_5 = 0
    \end{cases}
    \qquad \text{и} \qquad
    \begin{cases}
    x_1 + x_3 + x_5 = 0,\\
    2x_2 +x_3 + x_4 = 0,\\
    x_1 + 2x_2 + x_3 + 2x_4 - x_5 = 0
    \end{cases}
    \]
    соответственно.
    Найдите $\dim (U \cap W)$ и $\dim(U+W)$.

    \item К35.14(б,в)

    \item К35.15(а,в) (базис пересечения найти вторым (альтернативным) способом)

    \item
    Найдите базис пересечения подпространств из номера К35.15(в) через переход к заданию подпространств способом II.
    Сравните ответ с результатом предыдущей задачи и объясните напрямую, почему оба найденных базиса порождают одно и то же подпространство.

    \item К35.12(б)
    \item (бонус) Докажите утверждение $\diamond$ с семинара.


\end{enumerate}
\heart
    
\end{document}
