\documentclass[10pt, a4paper]{extarticle}

%% Язык
\usepackage{cmap} % Поиск в PDF
\usepackage{mathtext} % Кириллица в формулах
\usepackage[T2A]{fontenc} % Кодировка
\usepackage[utf8]{inputenc} % Кодировка
\usepackage[english,russian]{babel} % Локализация, переносы
\usepackage{bbold} % для ажурных буковок

\pagestyle{empty} \textwidth=19.0cm \oddsidemargin=-1.3cm
\textheight=26cm \topmargin=-3.0cm

%% Математика
\usepackage{amsmath, amsfonts, amssymb, amsthm, mathtools}
\usepackage{icomma}
\usepackage{algpseudocode}


% Операторы
\DeclareMathOperator{\tr}{tr}
\renewcommand{\le}{\leqslant}
\renewcommand{\ge}{\geqslant}
\renewcommand{\leq}{\leqslant}
\renewcommand{\geq}{\geqslant}

% Множества
\def \R{\mathbb{R}}
\def \N{\mathbb{N}}
\def \Z{\mathbb{Z}}
\newcommand{\rk}{\operatorname{\mathrm{rk}}}
\newcommand{\diag}{\operatorname{diag}}
\newcommand{\pr}{\mathop{\mathrm{pr}}}
\newcommand{\ort}{\mathop{\mathrm{ort}}}
\newcommand{\Ker}{\mathop{\mathrm{Ker}}}
\renewcommand{\Im}{\mathop{\mathrm{Im}}}
\newcommand{\Vol}{\mathop{\mathrm{Vol}}}

\def \a{\alpha}
\def \be{\beta}

% Другое

\newcommand{\const}{\mathrm{const}}
\theoremstyle{definition}
\newtheorem*{proposal}{Предложение}
\newtheorem{Task}{Задача}
%\newtheorem*{Taskn}{Задача {#1}}
\newtheorem*{Sol}{Решение}
\usepackage[dvipsnames]{xcolor}

\newcommand{\heart}{\begin{center}
\textcolor{RoyalPurple}{\ensuremath\heartsuit} 
\end{center}}
\usepackage{mathtools}
\usepackage{nicefrac}

%% Гиперссылки
\usepackage{xcolor}
\usepackage{hyperref}
\definecolor{linkcolor}{HTML}{8b00ff}
\hypersetup{colorlinks = true,
			linkcolor = linkcolor,
			urlcolor = linkcolor,
			citecolor = linkcolor}

%% Выравнивание
% \setlength{\parskip}{0.5em} % Расстояние между абзацами
\usepackage{geometry} % Поля
\geometry{
	a4paper,
	left=12mm,
	top=10mm,
    bottom=20mm,
	right=12mm}
% \setlength{\parindent}{0cm} % Отступ (красная строка)
% \linespread{1.1} % Интерлиньяж
\usepackage[many]{tcolorbox}  
\usepackage{enumitem}

%% Оформление

% Код
\newcommand{\code}[1]{{\tt #1}}

\newenvironment{amatrix}[2]{%
    \left(\begin{array}{@{}*{#1}{c}|*{#2}{c}@{}}
}{%
    \end{array}\right)
}

\begin{document}

\begin{center}
\small
\noindent\makebox[\textwidth]{Линейная алгебра и геометрия \hfill ФКН НИУ ВШЭ, 2022/2023 учебный год, 1-й курс ОП ПМИ, группа 2212}
\end{center}

\large

\begin{center}
\textbf{Семинар 24 (14.03.2023)}
\end{center}

\textbf{Краткое содержание}

Обсудили понятие расстояния в евклидовом пространстве и проговорили основную теорему о расстоянии от вектора до подпространства:

Пусть $S \subseteq \mathbb E$ --- подпространство и $x \in \mathbb E$.
Тогда $\rho(x,S) = | \mathrm{ort}_S x |$, причём $\mathrm{pr}_S x$ является ближайшим к $x$ вектором из $S$.


Дальше обсудили метод наименьших квадратов для несовместных систем линейных уравнений, а также явную формулу для псевдорешения в случае, когда столбцы матрицы коэффициентов линейно независимы.
Нашли псевдорешение для системы
\begin{equation} \label{eqn_SLE}
\begin{cases}
2x_1 + 5x_2 &= 3,\\
x_1 + 7x_2 &= -1,\\
2x_1 - 4x_2 &= -1.
\end{cases}
\end{equation}

Следующий сюжет --- $k$-мерный параллелепипед и его $k$-мерный объём.
Разобрали определение и формулу для объёма в терминах матрицы Грама системы векторов, задающих $k$-мерный параллелепипед, а также формулу для $n$-мерного объёма в $n$-мерном пространстве через определитель матрицы координат в ортонормированном базисе.
Нашли площадь параллелограмма в $\R^3$, натянутого на векторы $(2,1,2)$ и $(1,-2,1)$.

Дальше разобрали понятия ориентации и ориентированного объёма в евклидовом пространстве.

Новая тема --- векторные операции в пространстве $\R^3$ с фиксированной ориентацией.

Векторное произведение двух векторов $a,b \in \R^3$ можно определить как единственный вектор $[a,b] \in \R^3$, удовлетворяющий соотношению $([a,b],x) = \Vol(a,b,x)$ для всех $x \in \R^3$.
Разобрали геометрические свойства векторного произведения, которыми оно обычно определяется:

1) $[a,b]$ ортогонально каждому из векторов $a,b$;

2) длина вектора $[a,b]$ равна площади параллелограмма, натянутого на $a,b$;

3) $\Vol(a,b,[a,b]) \ge 0$.

Упомянули антикоммутативность, билинейность векторного произведения, а также критерий коллинеарности: два вектора $a,b \in \R^3$ коллинеарны ($=$ пропорциональны $=$ линейно зависимы) тогда и только тогда, когда $[a,b] = \vec 0$.
Также разобрали формулу для вычисления векторного произведения в координатах в положительно ориентированном ортонормированном базисе.

Смешанное произведение трёх векторов $a,b,c \in \R^3$ --- это величина $(a,b,c)$, равная попросту ориентированному объёму натянутого на них параллелепипеда.
Проговорили основные свойства смешанного произведения и формулу для его вычисления в координатах в положительно ориентированном ортонормированном базисе.
Упомянули критерий компланарности: три вектора $a,b,c \in \R^3$ компланарны ($=$ линейной зависимы) тогда и только тогда, когда $(a,b,c) = 0$.


Решили номера КК25.7, КК25.8.
Нашли ядро и образ линейного отображения $\R^3 \to \R^3$, $x \mapsto [x,a]$, где $a$ --- заданный ненулевой вектор.

\heart

\textbf{Домашнее задание к семинару 25. Дедлайн 21.03.2023}

Номера с пометкой П даны по задачнику Проскурякова, с пометкой К -- Кострикина, с пометкой КК -- Ким-Крицкова.

В обоих задачниках координаты векторов из $\R^n$ всегда записываются в строчку через запятую, однако нужно помнить, что мы всегда записываем эти координаты в столбец.


\begin{enumerate}

	\item К43.21(а)

	\item
	Рассмотрим евклидово пространство $\R[x]_{\leqslant 3}$ со скалярным произведением $(f,g) = \int\limits_{-1}^1 f(t)g(t)\,dt$.
	Найдите расстояние от вектора $x^3$ до подпространства $\langle 1,x,x^2 \rangle$.

	\item
	Найдите псевдорешение системы (\ref{eqn_SLE}) по явной формуле.
	
	\item
	Найдите псевдорешение для СЛУ из номера К43.30(а) двумя способами (через проекцию и по явной формуле).
	
	\item
	Найдите объём параллелепипеда в $\R^4$ (со стандартным скалярным произведением), натянутого на:
	
	(1) первые три вектора из номера К43.36(б);
	
	(2) все векторы из номера 43.36(б).
	
	\item
	Докажите, что $\operatorname{vol}P(a_1,\ldots,a_k) \leqslant |a_1|\cdot \ldots \cdot |a_k|$, то есть объём параллелепипеда не превосходит произведения длин его рёбер, выходящих из одной вершины.
	В каком случае в этом неравенстве достигается равенство?
	
	\item
	КК25.17
	
	\item
	КК25.18, 25.24(а)
	
	\item
	КК25.36

	

\end{enumerate}
\heart


\end{document}
