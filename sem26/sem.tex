\documentclass[10pt, a4paper]{extarticle}

%% Язык
\usepackage{cmap} % Поиск в PDF
\usepackage{mathtext} % Кириллица в формулах
\usepackage[T2A]{fontenc} % Кодировка
\usepackage[utf8]{inputenc} % Кодировка
\usepackage[english,russian]{babel} % Локализация, переносы
\usepackage{bbold} % для ажурных буковок

\pagestyle{empty} \textwidth=19.0cm \oddsidemargin=-1.3cm
\textheight=26cm \topmargin=-3.0cm

%% Математика
\usepackage{amsmath, amsfonts, amssymb, amsthm, mathtools}
\usepackage{icomma}
\usepackage{algpseudocode}


% Операторы
\DeclareMathOperator{\tr}{tr}
\renewcommand{\le}{\leqslant}
\renewcommand{\ge}{\geqslant}
\renewcommand{\leq}{\leqslant}
\renewcommand{\geq}{\geqslant}

% Множества
\def \R{\mathbb{R}}
\def \N{\mathbb{N}}
\def \Z{\mathbb{Z}}
\newcommand{\rk}{\operatorname{\mathrm{rk}}}
\newcommand{\diag}{\operatorname{diag}}
\newcommand{\pr}{\mathop{\mathrm{pr}}}
\newcommand{\ort}{\mathop{\mathrm{ort}}}
\newcommand{\Ker}{\mathop{\mathrm{Ker}}}
\renewcommand{\Im}{\mathop{\mathrm{Im}}}
\newcommand{\Vol}{\mathop{\mathrm{Vol}}}

\def \a{\alpha}
\def \be{\beta}

% Другое

\newcommand{\const}{\mathrm{const}}
\theoremstyle{definition}
\newtheorem*{proposal}{Предложение}
\newtheorem{Task}{Задача}
%\newtheorem*{Taskn}{Задача {#1}}
\newtheorem*{Sol}{Решение}
\usepackage[dvipsnames]{xcolor}

\newcommand{\heart}{\begin{center}
\textcolor{RoyalPurple}{\ensuremath\heartsuit} 
\end{center}}
\usepackage{mathtools}
\usepackage{nicefrac}

%% Гиперссылки
\usepackage{xcolor}
\usepackage{hyperref}
\definecolor{linkcolor}{HTML}{8b00ff}
\hypersetup{colorlinks = true,
			linkcolor = linkcolor,
			urlcolor = linkcolor,
			citecolor = linkcolor}

%% Выравнивание
% \setlength{\parskip}{0.5em} % Расстояние между абзацами
\usepackage{geometry} % Поля
\geometry{
	a4paper,
	left=12mm,
	top=10mm,
    bottom=20mm,
	right=12mm}
% \setlength{\parindent}{0cm} % Отступ (красная строка)
% \linespread{1.1} % Интерлиньяж
\usepackage[many]{tcolorbox}  
\usepackage{enumitem}

%% Оформление

% Код
\newcommand{\code}[1]{{\tt #1}}

\newenvironment{amatrix}[2]{%
    \left(\begin{array}{@{}*{#1}{c}|*{#2}{c}@{}}
}{%
    \end{array}\right)
}

\begin{document}

\begin{center}
\small
\noindent\makebox[\textwidth]{Линейная алгебра и геометрия \hfill ФКН НИУ ВШЭ, 2022/2023 учебный год, 1-й курс ОП ПМИ, группа 2212}
\end{center}

\large

\begin{center}
\textbf{Семинар 26 (11.04.2023)}
\end{center}

\textbf{Краткое содержание}

Разобрали случаи взаимного расположения двух плоскостей, прямой и плоскости, а также двух прямых в $\R^3$.

Дальше доказали, что подмножество в $\R^n$ является линейным многообразием тогда и только тогда, когда вместе с каждой парой своих различных точек оно содержит проходящую через них прямую.

Затем проговорили в общем виде, как решать задачи КК31.21--31.25.

Дальше обсудили формулы для нахождения расстояния от точки до прямой, от точки до плоскости, а также между двумя скрещивающимися прямыми.
Обсудили углы между двумя прямыми, между прямой и плоскостью, между двумя плоскостями.

Разобрали в общем виде следующую задачу: тетраэдр $ABCD$ задан координатами своих вершин; $AM$ --- медиана грани $ABD$, $BH$ --- высота, опущенная к грани $ACD$; найти угол и расстояние между прямыми $AM$ и $BH$, а также между прямыми $AM$ и $CH$.
Другой вариант: $AM$ --- биссектриса грани $ABD$, $BH$ --- высота грани $BCD$.

Новая тема --- линейные операторы.

Обсудили определение и матрицу линейного оператора в заданном базисе.

Нашли матрицу оператора
\begin{equation} 
x \mapsto [x, e_1+e_2+e_3]
\end{equation}
в $\R^3$ в положительно ориентированном ортонормированном базисе $(e_1,e_2,e_3)$.



\heart

\textbf{Домашнее задание к семинару 27. Дедлайн 19.04.2023}

Номера с пометкой П даны по задачнику Проскурякова, с пометкой К -- Кострикина, с пометкой КК -- Ким-Крицкова.

В обоих задачниках координаты векторов из $\R^n$ всегда записываются в строчку через запятую, однако нужно помнить, что мы всегда записываем эти координаты в столбец.


\begin{enumerate}

	
	\item
	КК27.32(2,3)

	\item
	КК31.18(2,3), КК31.19(2,3)

	\item
	КК31.13(1,3), КК31.15(1,2). Только определить взаимное расположение (составлять уравнения плоскостей и прочее не нужно).

	\item
	КК31.22

	\item
	КК31.23



\item
КК29.93

\item
КК32.35(1,2)

\item
КК32.37(1,2)

\item
КК32.28(1,2)

\item
КК32.30

\item
КК29.79

\item
В пространстве $\R^3$ со стандартным скалярным произведением задан тетраэдр с вершинами $A(2,-3,3)$, $B(20,-15,9)$, $C(2,-12,-6)$, $D(-18,10,8)$.
Пусть $BH$~--- высота грани $ABC$ и  $AM$~--- медиана грани $ACD$.
Найдите угол и расстояние между прямыми $BH$ и~$AM$.

	

\end{enumerate}
\heart


\end{document}
